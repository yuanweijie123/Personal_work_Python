
% Default to the notebook output style

    


% Inherit from the specified cell style.




    
\documentclass[11pt]{article}

    
    
    \usepackage[T1]{fontenc}
    % Nicer default font (+ math font) than Computer Modern for most use cases
    \usepackage{mathpazo}

    % Basic figure setup, for now with no caption control since it's done
    % automatically by Pandoc (which extracts ![](path) syntax from Markdown).
    \usepackage{graphicx}
    % We will generate all images so they have a width \maxwidth. This means
    % that they will get their normal width if they fit onto the page, but
    % are scaled down if they would overflow the margins.
    \makeatletter
    \def\maxwidth{\ifdim\Gin@nat@width>\linewidth\linewidth
    \else\Gin@nat@width\fi}
    \makeatother
    \let\Oldincludegraphics\includegraphics
    % Set max figure width to be 80% of text width, for now hardcoded.
    \renewcommand{\includegraphics}[1]{\Oldincludegraphics[width=.8\maxwidth]{#1}}
    % Ensure that by default, figures have no caption (until we provide a
    % proper Figure object with a Caption API and a way to capture that
    % in the conversion process - todo).
    \usepackage{caption}
    \DeclareCaptionLabelFormat{nolabel}{}
    \captionsetup{labelformat=nolabel}

    \usepackage{adjustbox} % Used to constrain images to a maximum size 
    \usepackage{xcolor} % Allow colors to be defined
    \usepackage{enumerate} % Needed for markdown enumerations to work
    \usepackage{geometry} % Used to adjust the document margins
    \usepackage{amsmath} % Equations
    \usepackage{amssymb} % Equations
    \usepackage{textcomp} % defines textquotesingle
    % Hack from http://tex.stackexchange.com/a/47451/13684:
    \AtBeginDocument{%
        \def\PYZsq{\textquotesingle}% Upright quotes in Pygmentized code
    }
    \usepackage{upquote} % Upright quotes for verbatim code
    \usepackage{eurosym} % defines \euro
    \usepackage[mathletters]{ucs} % Extended unicode (utf-8) support
    \usepackage[utf8x]{inputenc} % Allow utf-8 characters in the tex document
    \usepackage{fancyvrb} % verbatim replacement that allows latex
    \usepackage{grffile} % extends the file name processing of package graphics 
                         % to support a larger range 
    % The hyperref package gives us a pdf with properly built
    % internal navigation ('pdf bookmarks' for the table of contents,
    % internal cross-reference links, web links for URLs, etc.)
    \usepackage{hyperref}
    \usepackage{longtable} % longtable support required by pandoc >1.10
    \usepackage{booktabs}  % table support for pandoc > 1.12.2
    \usepackage[inline]{enumitem} % IRkernel/repr support (it uses the enumerate* environment)
    \usepackage[normalem]{ulem} % ulem is needed to support strikethroughs (\sout)
                                % normalem makes italics be italics, not underlines
    

    
    
    % Colors for the hyperref package
    \definecolor{urlcolor}{rgb}{0,.145,.698}
    \definecolor{linkcolor}{rgb}{.71,0.21,0.01}
    \definecolor{citecolor}{rgb}{.12,.54,.11}

    % ANSI colors
    \definecolor{ansi-black}{HTML}{3E424D}
    \definecolor{ansi-black-intense}{HTML}{282C36}
    \definecolor{ansi-red}{HTML}{E75C58}
    \definecolor{ansi-red-intense}{HTML}{B22B31}
    \definecolor{ansi-green}{HTML}{00A250}
    \definecolor{ansi-green-intense}{HTML}{007427}
    \definecolor{ansi-yellow}{HTML}{DDB62B}
    \definecolor{ansi-yellow-intense}{HTML}{B27D12}
    \definecolor{ansi-blue}{HTML}{208FFB}
    \definecolor{ansi-blue-intense}{HTML}{0065CA}
    \definecolor{ansi-magenta}{HTML}{D160C4}
    \definecolor{ansi-magenta-intense}{HTML}{A03196}
    \definecolor{ansi-cyan}{HTML}{60C6C8}
    \definecolor{ansi-cyan-intense}{HTML}{258F8F}
    \definecolor{ansi-white}{HTML}{C5C1B4}
    \definecolor{ansi-white-intense}{HTML}{A1A6B2}

    % commands and environments needed by pandoc snippets
    % extracted from the output of `pandoc -s`
    \providecommand{\tightlist}{%
      \setlength{\itemsep}{0pt}\setlength{\parskip}{0pt}}
    \DefineVerbatimEnvironment{Highlighting}{Verbatim}{commandchars=\\\{\}}
    % Add ',fontsize=\small' for more characters per line
    \newenvironment{Shaded}{}{}
    \newcommand{\KeywordTok}[1]{\textcolor[rgb]{0.00,0.44,0.13}{\textbf{{#1}}}}
    \newcommand{\DataTypeTok}[1]{\textcolor[rgb]{0.56,0.13,0.00}{{#1}}}
    \newcommand{\DecValTok}[1]{\textcolor[rgb]{0.25,0.63,0.44}{{#1}}}
    \newcommand{\BaseNTok}[1]{\textcolor[rgb]{0.25,0.63,0.44}{{#1}}}
    \newcommand{\FloatTok}[1]{\textcolor[rgb]{0.25,0.63,0.44}{{#1}}}
    \newcommand{\CharTok}[1]{\textcolor[rgb]{0.25,0.44,0.63}{{#1}}}
    \newcommand{\StringTok}[1]{\textcolor[rgb]{0.25,0.44,0.63}{{#1}}}
    \newcommand{\CommentTok}[1]{\textcolor[rgb]{0.38,0.63,0.69}{\textit{{#1}}}}
    \newcommand{\OtherTok}[1]{\textcolor[rgb]{0.00,0.44,0.13}{{#1}}}
    \newcommand{\AlertTok}[1]{\textcolor[rgb]{1.00,0.00,0.00}{\textbf{{#1}}}}
    \newcommand{\FunctionTok}[1]{\textcolor[rgb]{0.02,0.16,0.49}{{#1}}}
    \newcommand{\RegionMarkerTok}[1]{{#1}}
    \newcommand{\ErrorTok}[1]{\textcolor[rgb]{1.00,0.00,0.00}{\textbf{{#1}}}}
    \newcommand{\NormalTok}[1]{{#1}}
    
    % Additional commands for more recent versions of Pandoc
    \newcommand{\ConstantTok}[1]{\textcolor[rgb]{0.53,0.00,0.00}{{#1}}}
    \newcommand{\SpecialCharTok}[1]{\textcolor[rgb]{0.25,0.44,0.63}{{#1}}}
    \newcommand{\VerbatimStringTok}[1]{\textcolor[rgb]{0.25,0.44,0.63}{{#1}}}
    \newcommand{\SpecialStringTok}[1]{\textcolor[rgb]{0.73,0.40,0.53}{{#1}}}
    \newcommand{\ImportTok}[1]{{#1}}
    \newcommand{\DocumentationTok}[1]{\textcolor[rgb]{0.73,0.13,0.13}{\textit{{#1}}}}
    \newcommand{\AnnotationTok}[1]{\textcolor[rgb]{0.38,0.63,0.69}{\textbf{\textit{{#1}}}}}
    \newcommand{\CommentVarTok}[1]{\textcolor[rgb]{0.38,0.63,0.69}{\textbf{\textit{{#1}}}}}
    \newcommand{\VariableTok}[1]{\textcolor[rgb]{0.10,0.09,0.49}{{#1}}}
    \newcommand{\ControlFlowTok}[1]{\textcolor[rgb]{0.00,0.44,0.13}{\textbf{{#1}}}}
    \newcommand{\OperatorTok}[1]{\textcolor[rgb]{0.40,0.40,0.40}{{#1}}}
    \newcommand{\BuiltInTok}[1]{{#1}}
    \newcommand{\ExtensionTok}[1]{{#1}}
    \newcommand{\PreprocessorTok}[1]{\textcolor[rgb]{0.74,0.48,0.00}{{#1}}}
    \newcommand{\AttributeTok}[1]{\textcolor[rgb]{0.49,0.56,0.16}{{#1}}}
    \newcommand{\InformationTok}[1]{\textcolor[rgb]{0.38,0.63,0.69}{\textbf{\textit{{#1}}}}}
    \newcommand{\WarningTok}[1]{\textcolor[rgb]{0.38,0.63,0.69}{\textbf{\textit{{#1}}}}}
    
    
    % Define a nice break command that doesn't care if a line doesn't already
    % exist.
    \def\br{\hspace*{\fill} \\* }
    % Math Jax compatability definitions
    \def\gt{>}
    \def\lt{<}
    % Document parameters
    \title{hw0}
    
    
    

    % Pygments definitions
    
\makeatletter
\def\PY@reset{\let\PY@it=\relax \let\PY@bf=\relax%
    \let\PY@ul=\relax \let\PY@tc=\relax%
    \let\PY@bc=\relax \let\PY@ff=\relax}
\def\PY@tok#1{\csname PY@tok@#1\endcsname}
\def\PY@toks#1+{\ifx\relax#1\empty\else%
    \PY@tok{#1}\expandafter\PY@toks\fi}
\def\PY@do#1{\PY@bc{\PY@tc{\PY@ul{%
    \PY@it{\PY@bf{\PY@ff{#1}}}}}}}
\def\PY#1#2{\PY@reset\PY@toks#1+\relax+\PY@do{#2}}

\expandafter\def\csname PY@tok@w\endcsname{\def\PY@tc##1{\textcolor[rgb]{0.73,0.73,0.73}{##1}}}
\expandafter\def\csname PY@tok@c\endcsname{\let\PY@it=\textit\def\PY@tc##1{\textcolor[rgb]{0.25,0.50,0.50}{##1}}}
\expandafter\def\csname PY@tok@cp\endcsname{\def\PY@tc##1{\textcolor[rgb]{0.74,0.48,0.00}{##1}}}
\expandafter\def\csname PY@tok@k\endcsname{\let\PY@bf=\textbf\def\PY@tc##1{\textcolor[rgb]{0.00,0.50,0.00}{##1}}}
\expandafter\def\csname PY@tok@kp\endcsname{\def\PY@tc##1{\textcolor[rgb]{0.00,0.50,0.00}{##1}}}
\expandafter\def\csname PY@tok@kt\endcsname{\def\PY@tc##1{\textcolor[rgb]{0.69,0.00,0.25}{##1}}}
\expandafter\def\csname PY@tok@o\endcsname{\def\PY@tc##1{\textcolor[rgb]{0.40,0.40,0.40}{##1}}}
\expandafter\def\csname PY@tok@ow\endcsname{\let\PY@bf=\textbf\def\PY@tc##1{\textcolor[rgb]{0.67,0.13,1.00}{##1}}}
\expandafter\def\csname PY@tok@nb\endcsname{\def\PY@tc##1{\textcolor[rgb]{0.00,0.50,0.00}{##1}}}
\expandafter\def\csname PY@tok@nf\endcsname{\def\PY@tc##1{\textcolor[rgb]{0.00,0.00,1.00}{##1}}}
\expandafter\def\csname PY@tok@nc\endcsname{\let\PY@bf=\textbf\def\PY@tc##1{\textcolor[rgb]{0.00,0.00,1.00}{##1}}}
\expandafter\def\csname PY@tok@nn\endcsname{\let\PY@bf=\textbf\def\PY@tc##1{\textcolor[rgb]{0.00,0.00,1.00}{##1}}}
\expandafter\def\csname PY@tok@ne\endcsname{\let\PY@bf=\textbf\def\PY@tc##1{\textcolor[rgb]{0.82,0.25,0.23}{##1}}}
\expandafter\def\csname PY@tok@nv\endcsname{\def\PY@tc##1{\textcolor[rgb]{0.10,0.09,0.49}{##1}}}
\expandafter\def\csname PY@tok@no\endcsname{\def\PY@tc##1{\textcolor[rgb]{0.53,0.00,0.00}{##1}}}
\expandafter\def\csname PY@tok@nl\endcsname{\def\PY@tc##1{\textcolor[rgb]{0.63,0.63,0.00}{##1}}}
\expandafter\def\csname PY@tok@ni\endcsname{\let\PY@bf=\textbf\def\PY@tc##1{\textcolor[rgb]{0.60,0.60,0.60}{##1}}}
\expandafter\def\csname PY@tok@na\endcsname{\def\PY@tc##1{\textcolor[rgb]{0.49,0.56,0.16}{##1}}}
\expandafter\def\csname PY@tok@nt\endcsname{\let\PY@bf=\textbf\def\PY@tc##1{\textcolor[rgb]{0.00,0.50,0.00}{##1}}}
\expandafter\def\csname PY@tok@nd\endcsname{\def\PY@tc##1{\textcolor[rgb]{0.67,0.13,1.00}{##1}}}
\expandafter\def\csname PY@tok@s\endcsname{\def\PY@tc##1{\textcolor[rgb]{0.73,0.13,0.13}{##1}}}
\expandafter\def\csname PY@tok@sd\endcsname{\let\PY@it=\textit\def\PY@tc##1{\textcolor[rgb]{0.73,0.13,0.13}{##1}}}
\expandafter\def\csname PY@tok@si\endcsname{\let\PY@bf=\textbf\def\PY@tc##1{\textcolor[rgb]{0.73,0.40,0.53}{##1}}}
\expandafter\def\csname PY@tok@se\endcsname{\let\PY@bf=\textbf\def\PY@tc##1{\textcolor[rgb]{0.73,0.40,0.13}{##1}}}
\expandafter\def\csname PY@tok@sr\endcsname{\def\PY@tc##1{\textcolor[rgb]{0.73,0.40,0.53}{##1}}}
\expandafter\def\csname PY@tok@ss\endcsname{\def\PY@tc##1{\textcolor[rgb]{0.10,0.09,0.49}{##1}}}
\expandafter\def\csname PY@tok@sx\endcsname{\def\PY@tc##1{\textcolor[rgb]{0.00,0.50,0.00}{##1}}}
\expandafter\def\csname PY@tok@m\endcsname{\def\PY@tc##1{\textcolor[rgb]{0.40,0.40,0.40}{##1}}}
\expandafter\def\csname PY@tok@gh\endcsname{\let\PY@bf=\textbf\def\PY@tc##1{\textcolor[rgb]{0.00,0.00,0.50}{##1}}}
\expandafter\def\csname PY@tok@gu\endcsname{\let\PY@bf=\textbf\def\PY@tc##1{\textcolor[rgb]{0.50,0.00,0.50}{##1}}}
\expandafter\def\csname PY@tok@gd\endcsname{\def\PY@tc##1{\textcolor[rgb]{0.63,0.00,0.00}{##1}}}
\expandafter\def\csname PY@tok@gi\endcsname{\def\PY@tc##1{\textcolor[rgb]{0.00,0.63,0.00}{##1}}}
\expandafter\def\csname PY@tok@gr\endcsname{\def\PY@tc##1{\textcolor[rgb]{1.00,0.00,0.00}{##1}}}
\expandafter\def\csname PY@tok@ge\endcsname{\let\PY@it=\textit}
\expandafter\def\csname PY@tok@gs\endcsname{\let\PY@bf=\textbf}
\expandafter\def\csname PY@tok@gp\endcsname{\let\PY@bf=\textbf\def\PY@tc##1{\textcolor[rgb]{0.00,0.00,0.50}{##1}}}
\expandafter\def\csname PY@tok@go\endcsname{\def\PY@tc##1{\textcolor[rgb]{0.53,0.53,0.53}{##1}}}
\expandafter\def\csname PY@tok@gt\endcsname{\def\PY@tc##1{\textcolor[rgb]{0.00,0.27,0.87}{##1}}}
\expandafter\def\csname PY@tok@err\endcsname{\def\PY@bc##1{\setlength{\fboxsep}{0pt}\fcolorbox[rgb]{1.00,0.00,0.00}{1,1,1}{\strut ##1}}}
\expandafter\def\csname PY@tok@kc\endcsname{\let\PY@bf=\textbf\def\PY@tc##1{\textcolor[rgb]{0.00,0.50,0.00}{##1}}}
\expandafter\def\csname PY@tok@kd\endcsname{\let\PY@bf=\textbf\def\PY@tc##1{\textcolor[rgb]{0.00,0.50,0.00}{##1}}}
\expandafter\def\csname PY@tok@kn\endcsname{\let\PY@bf=\textbf\def\PY@tc##1{\textcolor[rgb]{0.00,0.50,0.00}{##1}}}
\expandafter\def\csname PY@tok@kr\endcsname{\let\PY@bf=\textbf\def\PY@tc##1{\textcolor[rgb]{0.00,0.50,0.00}{##1}}}
\expandafter\def\csname PY@tok@bp\endcsname{\def\PY@tc##1{\textcolor[rgb]{0.00,0.50,0.00}{##1}}}
\expandafter\def\csname PY@tok@fm\endcsname{\def\PY@tc##1{\textcolor[rgb]{0.00,0.00,1.00}{##1}}}
\expandafter\def\csname PY@tok@vc\endcsname{\def\PY@tc##1{\textcolor[rgb]{0.10,0.09,0.49}{##1}}}
\expandafter\def\csname PY@tok@vg\endcsname{\def\PY@tc##1{\textcolor[rgb]{0.10,0.09,0.49}{##1}}}
\expandafter\def\csname PY@tok@vi\endcsname{\def\PY@tc##1{\textcolor[rgb]{0.10,0.09,0.49}{##1}}}
\expandafter\def\csname PY@tok@vm\endcsname{\def\PY@tc##1{\textcolor[rgb]{0.10,0.09,0.49}{##1}}}
\expandafter\def\csname PY@tok@sa\endcsname{\def\PY@tc##1{\textcolor[rgb]{0.73,0.13,0.13}{##1}}}
\expandafter\def\csname PY@tok@sb\endcsname{\def\PY@tc##1{\textcolor[rgb]{0.73,0.13,0.13}{##1}}}
\expandafter\def\csname PY@tok@sc\endcsname{\def\PY@tc##1{\textcolor[rgb]{0.73,0.13,0.13}{##1}}}
\expandafter\def\csname PY@tok@dl\endcsname{\def\PY@tc##1{\textcolor[rgb]{0.73,0.13,0.13}{##1}}}
\expandafter\def\csname PY@tok@s2\endcsname{\def\PY@tc##1{\textcolor[rgb]{0.73,0.13,0.13}{##1}}}
\expandafter\def\csname PY@tok@sh\endcsname{\def\PY@tc##1{\textcolor[rgb]{0.73,0.13,0.13}{##1}}}
\expandafter\def\csname PY@tok@s1\endcsname{\def\PY@tc##1{\textcolor[rgb]{0.73,0.13,0.13}{##1}}}
\expandafter\def\csname PY@tok@mb\endcsname{\def\PY@tc##1{\textcolor[rgb]{0.40,0.40,0.40}{##1}}}
\expandafter\def\csname PY@tok@mf\endcsname{\def\PY@tc##1{\textcolor[rgb]{0.40,0.40,0.40}{##1}}}
\expandafter\def\csname PY@tok@mh\endcsname{\def\PY@tc##1{\textcolor[rgb]{0.40,0.40,0.40}{##1}}}
\expandafter\def\csname PY@tok@mi\endcsname{\def\PY@tc##1{\textcolor[rgb]{0.40,0.40,0.40}{##1}}}
\expandafter\def\csname PY@tok@il\endcsname{\def\PY@tc##1{\textcolor[rgb]{0.40,0.40,0.40}{##1}}}
\expandafter\def\csname PY@tok@mo\endcsname{\def\PY@tc##1{\textcolor[rgb]{0.40,0.40,0.40}{##1}}}
\expandafter\def\csname PY@tok@ch\endcsname{\let\PY@it=\textit\def\PY@tc##1{\textcolor[rgb]{0.25,0.50,0.50}{##1}}}
\expandafter\def\csname PY@tok@cm\endcsname{\let\PY@it=\textit\def\PY@tc##1{\textcolor[rgb]{0.25,0.50,0.50}{##1}}}
\expandafter\def\csname PY@tok@cpf\endcsname{\let\PY@it=\textit\def\PY@tc##1{\textcolor[rgb]{0.25,0.50,0.50}{##1}}}
\expandafter\def\csname PY@tok@c1\endcsname{\let\PY@it=\textit\def\PY@tc##1{\textcolor[rgb]{0.25,0.50,0.50}{##1}}}
\expandafter\def\csname PY@tok@cs\endcsname{\let\PY@it=\textit\def\PY@tc##1{\textcolor[rgb]{0.25,0.50,0.50}{##1}}}

\def\PYZbs{\char`\\}
\def\PYZus{\char`\_}
\def\PYZob{\char`\{}
\def\PYZcb{\char`\}}
\def\PYZca{\char`\^}
\def\PYZam{\char`\&}
\def\PYZlt{\char`\<}
\def\PYZgt{\char`\>}
\def\PYZsh{\char`\#}
\def\PYZpc{\char`\%}
\def\PYZdl{\char`\$}
\def\PYZhy{\char`\-}
\def\PYZsq{\char`\'}
\def\PYZdq{\char`\"}
\def\PYZti{\char`\~}
% for compatibility with earlier versions
\def\PYZat{@}
\def\PYZlb{[}
\def\PYZrb{]}
\makeatother


    % Exact colors from NB
    \definecolor{incolor}{rgb}{0.0, 0.0, 0.5}
    \definecolor{outcolor}{rgb}{0.545, 0.0, 0.0}



    
    % Prevent overflowing lines due to hard-to-break entities
    \sloppy 
    % Setup hyperref package
    \hypersetup{
      breaklinks=true,  % so long urls are correctly broken across lines
      colorlinks=true,
      urlcolor=urlcolor,
      linkcolor=linkcolor,
      citecolor=citecolor,
      }
    % Slightly bigger margins than the latex defaults
    
    \geometry{verbose,tmargin=1in,bmargin=1in,lmargin=1in,rmargin=1in}
    
    

    \begin{document}
    
    
    \maketitle
    
    

    
    Before you turn this problem in, make sure everything runs as expected.
First, \textbf{restart the kernel} (in the menubar, select
Kernel\(\rightarrow\)Restart) and then \textbf{run all cells} (in the
menubar, select Cell\(\rightarrow\)Run All).

Make sure you fill in any place that says \texttt{YOUR\ CODE\ HERE} or
"YOUR ANSWER HERE", as well as your name and collaborators below:

    \begin{Verbatim}[commandchars=\\\{\}]
{\color{incolor}In [{\color{incolor}1}]:} \PY{n}{NAME} \PY{o}{=} \PY{l+s+s2}{\PYZdq{}}\PY{l+s+s2}{Weijie Yuan}\PY{l+s+s2}{\PYZdq{}}
        \PY{n}{COLLABORATORS} \PY{o}{=} \PY{l+s+s2}{\PYZdq{}}\PY{l+s+s2}{N/A}\PY{l+s+s2}{\PYZdq{}}
\end{Verbatim}


    \begin{center}\rule{0.5\linewidth}{\linethickness}\end{center}

    \section{HW0: Introductions}\label{hw0-introductions}

\subsection{Setup, Prerequisites, and
Classification}\label{setup-prerequisites-and-classification}

\subsection{Due Date: Monday 9/3,
11:59PM}\label{due-date-monday-93-1159pm}

\subsection{Course Policies}\label{course-policies}

Here are some important course policies. These are also located at
http://www.ds100.org/fa18/.

\textbf{Collaboration Policy}

Data science is a collaborative activity. While you may talk with others
about the homework, we ask that you \textbf{write your solutions
individually}. If you do discuss the assignments with others please
\textbf{include their names} at the top of your notebook.

\subsection{This Assignment}\label{this-assignment}

Welcome to DS100! Before we work our way through the data science
lifecycle, we need to ground ourselves in some of the foundations of
data science. This assignment will cover your working environment, test
prerequisite understanding and help you create your (first?)
classification model.

One of the purposes of this homework is to help you diagnose your
preparedness for the course. You should ask yourself if the math
questions are too challenging, or if the coding is confusing. The rest
of this course will rely heavily on these foundations and this homework
should not be too difficult to complete. If it is, consider reviewing
supplementary material such as the DS100 textbook.

As you work through this assignment, you will learn:

\begin{itemize}
\tightlist
\item
  Python basics, like defining functions.
\item
  How to use the \texttt{numpy} library to compute with arrays of
  numbers.
\item
  Debugging in python and notebook using pdb
\item
  Linear algebra, calculus, probability
\item
  Basics of classification using KNN
\end{itemize}

\subsection{Score breakdown}\label{score-breakdown}

\begin{longtable}[]{@{}ll@{}}
\toprule
Question & Points\tabularnewline
\midrule
\endhead
1a & 1\tabularnewline
1b & 1\tabularnewline
1c & 1\tabularnewline
1d & 1\tabularnewline
2a & 0.33\tabularnewline
2b & 0.33\tabularnewline
2c & 0.33\tabularnewline
2d & 1\tabularnewline
3a & 1\tabularnewline
3b & 1\tabularnewline
3c & 2\tabularnewline
4a & 4\tabularnewline
4b & 1\tabularnewline
4c & 2\tabularnewline
4d & 6\tabularnewline
4e & 4\tabularnewline
4f & 2\tabularnewline
4g & 1\tabularnewline
4h & 2\tabularnewline
5 & 6\tabularnewline
Total & 38\tabularnewline
\bottomrule
\end{longtable}

    \subsubsection{Running a Cell}\label{running-a-cell}

Try running the following cell. If you are unfamiliar with Jupyter
Notebooks consider skimming
\href{http://nbviewer.jupyter.org/github/jupyter/notebook/blob/master/docs/source/examples/Notebook/Notebook\%20Basics.ipynb}{this
tutorial} or selecting \textbf{Help -\textgreater{} User Interface Tour}
in the menu above.

    \begin{Verbatim}[commandchars=\\\{\}]
{\color{incolor}In [{\color{incolor}2}]:} \PY{n+nb}{print}\PY{p}{(}\PY{l+s+s2}{\PYZdq{}}\PY{l+s+s2}{Hello World!}\PY{l+s+s2}{\PYZdq{}}\PY{p}{)}
\end{Verbatim}


    \begin{Verbatim}[commandchars=\\\{\}]
Hello World!

    \end{Verbatim}

    Even if you are familiar with Jupyter, we strongly encourage you to
become proficient with keyboard shortcuts (this will save you time in
the future). To learn about keyboard shortcuts go to \textbf{Help
-\textgreater{} Keyboard Shortcuts} in the menu above.

Here are a few we like: 1. \texttt{ctrl}+\texttt{return} :
\emph{Evaluate the current cell} 1. \texttt{shift}+\texttt{return}:
\emph{Evaluate the current cell and move to the next} 1. \texttt{esc} :
\emph{command mode} (may need to press before using any of the commands
below) 1. \texttt{a} : \emph{create a cell above} 1. \texttt{b} :
\emph{create a cell below} 1. \texttt{dd} : \emph{delete a cell} 1.
\texttt{m} : \emph{convert a cell to markdown} 1. \texttt{y} :
\emph{convert a cell to code}

    \subsubsection{Testing your Setup}\label{testing-your-setup}

This cell should run without problems:

    \begin{Verbatim}[commandchars=\\\{\}]
{\color{incolor}In [{\color{incolor}3}]:} \PY{k+kn}{import} \PY{n+nn}{math}
        \PY{k+kn}{import} \PY{n+nn}{numpy} \PY{k}{as} \PY{n+nn}{np}
        \PY{k+kn}{import} \PY{n+nn}{matplotlib}
        \PY{o}{\PYZpc{}}\PY{k}{matplotlib} inline
        \PY{k+kn}{import} \PY{n+nn}{matplotlib}\PY{n+nn}{.}\PY{n+nn}{pyplot} \PY{k}{as} \PY{n+nn}{plt}
        \PY{n}{plt}\PY{o}{.}\PY{n}{style}\PY{o}{.}\PY{n}{use}\PY{p}{(}\PY{l+s+s1}{\PYZsq{}}\PY{l+s+s1}{fivethirtyeight}\PY{l+s+s1}{\PYZsq{}}\PY{p}{)}
        \PY{k+kn}{import} \PY{n+nn}{pandas} \PY{k}{as} \PY{n+nn}{pd}
        \PY{k+kn}{import} \PY{n+nn}{skimage}
        \PY{k+kn}{import} \PY{n+nn}{skimage}\PY{n+nn}{.}\PY{n+nn}{io}
        \PY{k+kn}{import} \PY{n+nn}{skimage}\PY{n+nn}{.}\PY{n+nn}{filters}
\end{Verbatim}


    \begin{center}\rule{0.5\linewidth}{\linethickness}\end{center}

\subsection{1: Prerequisites}\label{prerequisites}

    \subsubsection{Python}\label{python}

Python is the main programming language we'll use in the course. We
expect that you've taken CS61A, CS8, or an equivalent class, so you
should be able to explain the following cells. Run them and make sure
you understand what is happening in each.

If this seems difficult, please review one or more of the following
materials.

\begin{itemize}
\tightlist
\item
  \textbf{\href{https://docs.python.org/3.5/tutorial/}{Python
  Tutorial}}: Introduction to Python from the creators of Python.
\item
  \textbf{\href{http://composingprograms.com/pages/11-getting-started.html}{Composing
  Programs Chapter 1}}: This is more of a introduction to programming
  with Python.
\item
  \textbf{\href{http://cs231n.github.io/python-numpy-tutorial/}{Advanced
  Crash Course}}: A fast crash course which assumes some programming
  background.
\end{itemize}

    \paragraph{Mathematical Expressions}\label{mathematical-expressions}

Note that the rocket icon indicates that you should just run the
following cells.

    \begin{Verbatim}[commandchars=\\\{\}]
{\color{incolor}In [{\color{incolor}4}]:} \PY{c+c1}{\PYZsh{} This is a comment.}
        \PY{c+c1}{\PYZsh{} In Python, the ** operator performs exponentiation.}
        \PY{n}{math}\PY{o}{.}\PY{n}{sqrt}\PY{p}{(}\PY{n}{math}\PY{o}{.}\PY{n}{e} \PY{o}{*}\PY{o}{*} \PY{p}{(}\PY{o}{\PYZhy{}}\PY{n}{math}\PY{o}{.}\PY{n}{pi} \PY{o}{+} \PY{l+m+mi}{1}\PY{p}{)}\PY{p}{)}
\end{Verbatim}


\begin{Verbatim}[commandchars=\\\{\}]
{\color{outcolor}Out[{\color{outcolor}4}]:} 0.3427354792736325
\end{Verbatim}
            
    \paragraph{Output and Printing}\label{output-and-printing}

    \begin{Verbatim}[commandchars=\\\{\}]
{\color{incolor}In [{\color{incolor}5}]:} \PY{l+s+s2}{\PYZdq{}}\PY{l+s+s2}{Why didn}\PY{l+s+s2}{\PYZsq{}}\PY{l+s+s2}{t this line print?}\PY{l+s+s2}{\PYZdq{}}
        
        \PY{n+nb}{print}\PY{p}{(}\PY{l+s+s2}{\PYZdq{}}\PY{l+s+s2}{Hello}\PY{l+s+s2}{\PYZdq{}} \PY{o}{+} \PY{l+s+s2}{\PYZdq{}}\PY{l+s+s2}{,}\PY{l+s+s2}{\PYZdq{}}\PY{p}{,} \PY{l+s+s2}{\PYZdq{}}\PY{l+s+s2}{world!}\PY{l+s+s2}{\PYZdq{}}\PY{p}{)}
        
        \PY{l+s+s2}{\PYZdq{}}\PY{l+s+s2}{Hello, cell}\PY{l+s+s2}{\PYZdq{}} \PY{o}{+} \PY{l+s+s2}{\PYZdq{}}\PY{l+s+s2}{ output!}\PY{l+s+s2}{\PYZdq{}}
\end{Verbatim}


    \begin{Verbatim}[commandchars=\\\{\}]
Hello, world!

    \end{Verbatim}

\begin{Verbatim}[commandchars=\\\{\}]
{\color{outcolor}Out[{\color{outcolor}5}]:} 'Hello, cell output!'
\end{Verbatim}
            
    \paragraph{For Loops}\label{for-loops}

    \begin{Verbatim}[commandchars=\\\{\}]
{\color{incolor}In [{\color{incolor}6}]:} \PY{c+c1}{\PYZsh{} A for loop repeats a block of code once for each}
        \PY{c+c1}{\PYZsh{} element in a given collection.}
        \PY{k}{for} \PY{n}{i} \PY{o+ow}{in} \PY{n+nb}{range}\PY{p}{(}\PY{l+m+mi}{5}\PY{p}{)}\PY{p}{:}
            \PY{k}{if} \PY{n}{i} \PY{o}{\PYZpc{}} \PY{l+m+mi}{2} \PY{o}{==} \PY{l+m+mi}{0}\PY{p}{:}
                \PY{n+nb}{print}\PY{p}{(}\PY{l+m+mi}{2}\PY{o}{*}\PY{o}{*}\PY{n}{i}\PY{p}{)}
            \PY{k}{else}\PY{p}{:}
                \PY{n+nb}{print}\PY{p}{(}\PY{l+s+s2}{\PYZdq{}}\PY{l+s+s2}{Odd power of 2}\PY{l+s+s2}{\PYZdq{}}\PY{p}{)}
\end{Verbatim}


    \begin{Verbatim}[commandchars=\\\{\}]
1
Odd power of 2
4
Odd power of 2
16

    \end{Verbatim}

    \paragraph{List Comprehension}\label{list-comprehension}

    \begin{Verbatim}[commandchars=\\\{\}]
{\color{incolor}In [{\color{incolor}7}]:} \PY{p}{[}\PY{n+nb}{str}\PY{p}{(}\PY{n}{i}\PY{p}{)} \PY{o}{+} \PY{l+s+s2}{\PYZdq{}}\PY{l+s+s2}{ sheep.}\PY{l+s+s2}{\PYZdq{}} \PY{k}{for} \PY{n}{i} \PY{o+ow}{in} \PY{n+nb}{range}\PY{p}{(}\PY{l+m+mi}{1}\PY{p}{,}\PY{l+m+mi}{5}\PY{p}{)}\PY{p}{]} 
\end{Verbatim}


\begin{Verbatim}[commandchars=\\\{\}]
{\color{outcolor}Out[{\color{outcolor}7}]:} ['1 sheep.', '2 sheep.', '3 sheep.', '4 sheep.']
\end{Verbatim}
            
    \begin{Verbatim}[commandchars=\\\{\}]
{\color{incolor}In [{\color{incolor}8}]:} \PY{p}{[}\PY{n}{i} \PY{k}{for} \PY{n}{i} \PY{o+ow}{in} \PY{n+nb}{range}\PY{p}{(}\PY{l+m+mi}{10}\PY{p}{)} \PY{k}{if} \PY{n}{i} \PY{o}{\PYZpc{}} \PY{l+m+mi}{2} \PY{o}{==} \PY{l+m+mi}{0}\PY{p}{]}
\end{Verbatim}


\begin{Verbatim}[commandchars=\\\{\}]
{\color{outcolor}Out[{\color{outcolor}8}]:} [0, 2, 4, 6, 8]
\end{Verbatim}
            
    \paragraph{Defining Functions}\label{defining-functions}

    \begin{Verbatim}[commandchars=\\\{\}]
{\color{incolor}In [{\color{incolor}9}]:} \PY{k}{def} \PY{n+nf}{add2}\PY{p}{(}\PY{n}{x}\PY{p}{)}\PY{p}{:}
            \PY{l+s+sd}{\PYZdq{}\PYZdq{}\PYZdq{}This docstring explains what this function does: it adds 2 to a number.\PYZdq{}\PYZdq{}\PYZdq{}}
            \PY{k}{return} \PY{n}{x} \PY{o}{+} \PY{l+m+mi}{2}
\end{Verbatim}


    \paragraph{Getting Help}\label{getting-help}

    \begin{Verbatim}[commandchars=\\\{\}]
{\color{incolor}In [{\color{incolor}10}]:} \PY{n}{help}\PY{p}{(}\PY{n}{add2}\PY{p}{)}
\end{Verbatim}


    \begin{Verbatim}[commandchars=\\\{\}]
Help on function add2 in module \_\_main\_\_:

add2(x)
    This docstring explains what this function does: it adds 2 to a number.


    \end{Verbatim}

    \paragraph{Passing Functions as
Values}\label{passing-functions-as-values}

    \begin{Verbatim}[commandchars=\\\{\}]
{\color{incolor}In [{\color{incolor}11}]:} \PY{k}{def} \PY{n+nf}{makeAdder}\PY{p}{(}\PY{n}{amount}\PY{p}{)}\PY{p}{:}
             \PY{l+s+sd}{\PYZdq{}\PYZdq{}\PYZdq{}Make a function that adds the given amount to a number.\PYZdq{}\PYZdq{}\PYZdq{}}
             \PY{k}{def} \PY{n+nf}{addAmount}\PY{p}{(}\PY{n}{x}\PY{p}{)}\PY{p}{:}
                 \PY{k}{return} \PY{n}{x} \PY{o}{+} \PY{n}{amount}
             \PY{k}{return} \PY{n}{addAmount}
         
         \PY{n}{add3} \PY{o}{=} \PY{n}{makeAdder}\PY{p}{(}\PY{l+m+mi}{3}\PY{p}{)}
         \PY{n}{add3}\PY{p}{(}\PY{l+m+mi}{4}\PY{p}{)}
\end{Verbatim}


\begin{Verbatim}[commandchars=\\\{\}]
{\color{outcolor}Out[{\color{outcolor}11}]:} 7
\end{Verbatim}
            
    \begin{Verbatim}[commandchars=\\\{\}]
{\color{incolor}In [{\color{incolor}12}]:} \PY{n}{makeAdder}\PY{p}{(}\PY{l+m+mi}{3}\PY{p}{)}\PY{p}{(}\PY{l+m+mi}{4}\PY{p}{)}
\end{Verbatim}


\begin{Verbatim}[commandchars=\\\{\}]
{\color{outcolor}Out[{\color{outcolor}12}]:} 7
\end{Verbatim}
            
    \paragraph{Anonymous Functions and
Lambdas}\label{anonymous-functions-and-lambdas}

    \begin{Verbatim}[commandchars=\\\{\}]
{\color{incolor}In [{\color{incolor}13}]:} \PY{c+c1}{\PYZsh{} add4 is very similar to add2, but it\PYZsq{}s been created using a lambda expression.}
         \PY{n}{add4} \PY{o}{=} \PY{k}{lambda} \PY{n}{x}\PY{p}{:} \PY{n}{x} \PY{o}{+} \PY{l+m+mi}{4}
         \PY{n}{add4}\PY{p}{(}\PY{l+m+mi}{5}\PY{p}{)}
\end{Verbatim}


\begin{Verbatim}[commandchars=\\\{\}]
{\color{outcolor}Out[{\color{outcolor}13}]:} 9
\end{Verbatim}
            
    \paragraph{Recursion}\label{recursion}

    \begin{Verbatim}[commandchars=\\\{\}]
{\color{incolor}In [{\color{incolor}14}]:} \PY{k}{def} \PY{n+nf}{fib}\PY{p}{(}\PY{n}{n}\PY{p}{)}\PY{p}{:}
             \PY{k}{if} \PY{n}{n} \PY{o}{\PYZlt{}}\PY{o}{=} \PY{l+m+mi}{1}\PY{p}{:}
                 \PY{k}{return} \PY{l+m+mi}{1}
             \PY{k}{else}\PY{p}{:}
                 \PY{c+c1}{\PYZsh{} Functions can call themselves recursively.}
                 \PY{k}{return} \PY{n}{fib}\PY{p}{(}\PY{n}{n}\PY{o}{\PYZhy{}}\PY{l+m+mi}{1}\PY{p}{)} \PY{o}{+} \PY{n}{fib}\PY{p}{(}\PY{n}{n}\PY{o}{\PYZhy{}}\PY{l+m+mi}{2}\PY{p}{)}
         
         \PY{n}{fib}\PY{p}{(}\PY{l+m+mi}{6}\PY{p}{)}
\end{Verbatim}


\begin{Verbatim}[commandchars=\\\{\}]
{\color{outcolor}Out[{\color{outcolor}14}]:} 13
\end{Verbatim}
            
    \subsubsection{Question 1}\label{question-1}

\paragraph{Question 1a}\label{question-1a}

Write a function \texttt{nums\_reversed} that takes in a positive
integer \texttt{n} and returns a string containing the numbers 1 through
\texttt{n} including \texttt{n} in reverse order, separated by spaces.
For example:

\begin{verbatim}
>>> nums_reversed(5)
'5 4 3 2 1'
\end{verbatim}

\textbf{\emph{Note:}} The line \texttt{raise\ NotImplementedError()}
indicates that the implementation still needs to be added. This is an
exception derived from \texttt{RuntimeError}. Please comment out that
line when you have implemented the function.

 The code icon indicates that you should complete the following block of
code.

    \begin{Verbatim}[commandchars=\\\{\}]
{\color{incolor}In [{\color{incolor}15}]:} \PY{k}{def} \PY{n+nf}{nums\PYZus{}reversed}\PY{p}{(}\PY{n}{n}\PY{p}{)}\PY{p}{:}
             \PY{n+nb}{list} \PY{o}{=} \PY{p}{[}\PY{n}{i} \PY{k}{for} \PY{n}{i} \PY{o+ow}{in} \PY{n+nb}{range}\PY{p}{(}\PY{l+m+mi}{1}\PY{p}{,}\PY{n}{n}\PY{o}{+}\PY{l+m+mi}{1}\PY{p}{)}\PY{p}{]}
             \PY{n+nb}{list}\PY{o}{.}\PY{n}{reverse}\PY{p}{(}\PY{p}{)}
             \PY{k}{return} \PY{l+s+s1}{\PYZsq{}}\PY{l+s+s1}{ }\PY{l+s+s1}{\PYZsq{}}\PY{o}{.}\PY{n}{join}\PY{p}{(}\PY{n+nb}{map}\PY{p}{(}\PY{n+nb}{str}\PY{p}{,}\PY{n+nb}{list}\PY{p}{)}\PY{p}{)}
\end{Verbatim}


    \begin{Verbatim}[commandchars=\\\{\}]
{\color{incolor}In [{\color{incolor}16}]:} \PY{n}{nums\PYZus{}reversed}\PY{p}{(}\PY{l+m+mi}{5}\PY{p}{)}
\end{Verbatim}


\begin{Verbatim}[commandchars=\\\{\}]
{\color{outcolor}Out[{\color{outcolor}16}]:} '5 4 3 2 1'
\end{Verbatim}
            
    Test your code in the cell below.

    \begin{Verbatim}[commandchars=\\\{\}]
{\color{incolor}In [{\color{incolor}17}]:} \PY{k}{assert} \PY{n}{nums\PYZus{}reversed}\PY{p}{(}\PY{l+m+mi}{5}\PY{p}{)} \PY{o}{==} \PY{l+s+s1}{\PYZsq{}}\PY{l+s+s1}{5 4 3 2 1}\PY{l+s+s1}{\PYZsq{}}
         \PY{k}{assert} \PY{n}{nums\PYZus{}reversed}\PY{p}{(}\PY{l+m+mi}{1}\PY{p}{)} \PY{o}{==} \PY{l+s+s1}{\PYZsq{}}\PY{l+s+s1}{1}\PY{l+s+s1}{\PYZsq{}}
         \PY{k}{assert} \PY{n}{nums\PYZus{}reversed}\PY{p}{(}\PY{l+m+mi}{3}\PY{p}{)} \PY{o}{==}  \PY{l+s+s1}{\PYZsq{}}\PY{l+s+s1}{3 2 1}\PY{l+s+s1}{\PYZsq{}}
\end{Verbatim}


    \paragraph{Question 1b}\label{question-1b}

Write a function \texttt{string\_splosion} that takes in a non-empty
string like \texttt{"Code"} and returns a long string containing every
prefix of the input. For example:

\begin{verbatim}
>>> string_splosion('Code')
'CCoCodCode'
>>> string_splosion('data!')
'ddadatdatadata!'
>>> string_splosion('hi')
'hhi'
\end{verbatim}

\textbf{Hint:} Try to use recursion. Think about how you might answering
the following two questions: 1. \textbf{{[}Base Case{]}} What is the
\texttt{string\_splosion} of the empty string? 1. \textbf{{[}Inductive
Step{]}} If you had a \texttt{string\_splosion} function for the first
\(n-1\) characters of your string how could you extend it to the
\(n^{th}\) character? For example,
\texttt{string\_splosion("Cod")\ =\ "CCoCod"} becomes
\texttt{string\_splosion("Code")\ =\ "CCoCodCode"}.

    \begin{Verbatim}[commandchars=\\\{\}]
{\color{incolor}In [{\color{incolor}18}]:} \PY{k}{def} \PY{n+nf}{string\PYZus{}splosion}\PY{p}{(}\PY{n}{string}\PY{p}{)}\PY{p}{:}
             \PY{n}{sep} \PY{o}{=} \PY{p}{[}\PY{n}{string}\PY{p}{[}\PY{l+m+mi}{0}\PY{p}{:}\PY{n}{i}\PY{p}{]} \PY{k}{for} \PY{n}{i} \PY{o+ow}{in} \PY{n+nb}{range}\PY{p}{(}\PY{l+m+mi}{1}\PY{p}{,}\PY{n+nb}{len}\PY{p}{(}\PY{n}{string}\PY{p}{)}\PY{o}{+}\PY{l+m+mi}{1}\PY{p}{)}\PY{p}{]}
             \PY{k}{return} \PY{l+s+s1}{\PYZsq{}}\PY{l+s+s1}{\PYZsq{}}\PY{o}{.}\PY{n}{join}\PY{p}{(}\PY{n}{sep}\PY{p}{)}
\end{Verbatim}


    \begin{Verbatim}[commandchars=\\\{\}]
{\color{incolor}In [{\color{incolor}19}]:} \PY{n}{string\PYZus{}splosion}\PY{p}{(}\PY{l+s+s2}{\PYZdq{}}\PY{l+s+s2}{Cod}\PY{l+s+s2}{\PYZdq{}}\PY{p}{)}
\end{Verbatim}


\begin{Verbatim}[commandchars=\\\{\}]
{\color{outcolor}Out[{\color{outcolor}19}]:} 'CCoCod'
\end{Verbatim}
            
    Test your code in the cell below.

    \begin{Verbatim}[commandchars=\\\{\}]
{\color{incolor}In [{\color{incolor}20}]:} \PY{k}{assert} \PY{n}{string\PYZus{}splosion}\PY{p}{(}\PY{l+s+s1}{\PYZsq{}}\PY{l+s+s1}{Code}\PY{l+s+s1}{\PYZsq{}}\PY{p}{)} \PY{o}{==} \PY{l+s+s1}{\PYZsq{}}\PY{l+s+s1}{CCoCodCode}\PY{l+s+s1}{\PYZsq{}}
         \PY{k}{assert} \PY{n}{string\PYZus{}splosion}\PY{p}{(}\PY{l+s+s1}{\PYZsq{}}\PY{l+s+s1}{fade}\PY{l+s+s1}{\PYZsq{}}\PY{p}{)} \PY{o}{==} \PY{l+s+s1}{\PYZsq{}}\PY{l+s+s1}{ffafadfade}\PY{l+s+s1}{\PYZsq{}}
         \PY{k}{assert} \PY{n}{string\PYZus{}splosion}\PY{p}{(}\PY{l+s+s1}{\PYZsq{}}\PY{l+s+s1}{Kitten}\PY{l+s+s1}{\PYZsq{}}\PY{p}{)} \PY{o}{==} \PY{l+s+s1}{\PYZsq{}}\PY{l+s+s1}{KKiKitKittKitteKitten}\PY{l+s+s1}{\PYZsq{}}
         \PY{k}{assert} \PY{n}{string\PYZus{}splosion}\PY{p}{(}\PY{l+s+s1}{\PYZsq{}}\PY{l+s+s1}{data!}\PY{l+s+s1}{\PYZsq{}}\PY{p}{)} \PY{o}{==} \PY{l+s+s1}{\PYZsq{}}\PY{l+s+s1}{ddadatdatadata!}\PY{l+s+s1}{\PYZsq{}}
\end{Verbatim}


    \paragraph{Question 1c}\label{question-1c}

Write a function \texttt{double100} that takes in a list of integers and
returns \texttt{True} only if the list has two \texttt{100}s next to
each other.

\begin{verbatim}
>>> double100([100, 2, 3, 100])
False
>>> double100([2, 3, 100, 100, 5])
True
\end{verbatim}

    \begin{Verbatim}[commandchars=\\\{\}]
{\color{incolor}In [{\color{incolor}21}]:} \PY{k}{def} \PY{n+nf}{double100}\PY{p}{(}\PY{n}{nums}\PY{p}{)}\PY{p}{:}
             \PY{n}{flag}\PY{o}{=}\PY{k+kc}{False}
             \PY{k}{if} \PY{l+m+mi}{100} \PY{o+ow}{in} \PY{n}{nums}\PY{p}{:}
                 \PY{n}{index1}\PY{o}{=}\PY{n}{nums}\PY{o}{.}\PY{n}{index}\PY{p}{(}\PY{l+m+mi}{100}\PY{p}{)}
                 \PY{k}{if} \PY{n}{nums}\PY{p}{[}\PY{n}{index1}\PY{o}{+}\PY{l+m+mi}{1}\PY{p}{]}\PY{o}{==}\PY{l+m+mi}{100}\PY{p}{:}
                     \PY{n}{flag}\PY{o}{=}\PY{k+kc}{True}
             \PY{k}{return} \PY{n}{flag}
\end{Verbatim}


    \begin{Verbatim}[commandchars=\\\{\}]
{\color{incolor}In [{\color{incolor}22}]:} \PY{n}{double100}\PY{p}{(}\PY{p}{[}\PY{l+m+mi}{100}\PY{p}{,} \PY{l+m+mi}{2}\PY{p}{,} \PY{l+m+mi}{3}\PY{p}{,} \PY{l+m+mi}{100}\PY{p}{]}\PY{p}{)}
\end{Verbatim}


\begin{Verbatim}[commandchars=\\\{\}]
{\color{outcolor}Out[{\color{outcolor}22}]:} False
\end{Verbatim}
            
    \begin{Verbatim}[commandchars=\\\{\}]
{\color{incolor}In [{\color{incolor}23}]:} \PY{k}{assert} \PY{n}{double100}\PY{p}{(}\PY{p}{[}\PY{l+m+mi}{3}\PY{p}{,} \PY{l+m+mi}{3}\PY{p}{,} \PY{l+m+mi}{100}\PY{p}{,} \PY{l+m+mi}{100}\PY{p}{]}\PY{p}{)} \PY{o}{==} \PY{k+kc}{True}
         \PY{k}{assert} \PY{n}{double100}\PY{p}{(}\PY{p}{[}\PY{l+m+mi}{5}\PY{p}{,} \PY{l+m+mi}{2}\PY{p}{,} \PY{l+m+mi}{5}\PY{p}{,} \PY{l+m+mi}{2}\PY{p}{]}\PY{p}{)} \PY{o}{==} \PY{k+kc}{False}
         \PY{k}{assert} \PY{n}{double100}\PY{p}{(}\PY{p}{[}\PY{l+m+mi}{4}\PY{p}{,} \PY{l+m+mi}{2}\PY{p}{,} \PY{l+m+mi}{4}\PY{p}{,} \PY{l+m+mi}{100}\PY{p}{,} \PY{l+m+mi}{100}\PY{p}{,} \PY{l+m+mi}{5}\PY{p}{]}\PY{p}{)} \PY{o}{==} \PY{k+kc}{True}
         \PY{k}{assert} \PY{n}{double100}\PY{p}{(}\PY{p}{[}\PY{l+m+mi}{4}\PY{p}{,} \PY{l+m+mi}{2}\PY{p}{,} \PY{l+m+mi}{4}\PY{p}{,} \PY{l+m+mi}{10}\PY{p}{,} \PY{l+m+mi}{10}\PY{p}{,} \PY{l+m+mi}{5}\PY{p}{]}\PY{p}{)} \PY{o}{==} \PY{k+kc}{False}
\end{Verbatim}


    \paragraph{Question 1d}\label{question-1d}

Recall the formula for population variance below:

\[\sigma^2 = \frac{\sum_{i=1}^N (x_i - \mu)^2}{N}\]

In this question, we'll ask you to compute the population variance of a
given list by completing the functions below. For this portion, do not
use built in numpy functions; we will use numpy to verify our code.

    \begin{Verbatim}[commandchars=\\\{\}]
{\color{incolor}In [{\color{incolor}24}]:} \PY{k}{def} \PY{n+nf}{mean}\PY{p}{(}\PY{n}{population}\PY{p}{)}\PY{p}{:}
             \PY{k}{return} \PY{n+nb}{sum}\PY{p}{(}\PY{n}{population}\PY{p}{)}\PY{o}{/}\PY{n+nb}{len}\PY{p}{(}\PY{n}{population}\PY{p}{)}
         
         \PY{k}{def} \PY{n+nf}{variance}\PY{p}{(}\PY{n}{population}\PY{p}{)}\PY{p}{:}
             \PY{k}{return} \PY{n+nb}{sum}\PY{p}{(}\PY{p}{(}\PY{n}{population}\PY{o}{\PYZhy{}}\PY{n}{mean}\PY{p}{(}\PY{n}{population}\PY{p}{)}\PY{p}{)}\PY{o}{*}\PY{o}{*}\PY{l+m+mi}{2}\PY{p}{)}\PY{o}{/}\PY{n+nb}{len}\PY{p}{(}\PY{n}{population}\PY{p}{)}
\end{Verbatim}


    \begin{Verbatim}[commandchars=\\\{\}]
{\color{incolor}In [{\color{incolor}25}]:} \PY{k+kn}{from} \PY{n+nn}{math} \PY{k}{import} \PY{n}{isclose}
         
         \PY{n}{population\PYZus{}0} \PY{o}{=} \PY{n}{np}\PY{o}{.}\PY{n}{random}\PY{o}{.}\PY{n}{randn}\PY{p}{(}\PY{l+m+mi}{100}\PY{p}{)}
         \PY{n}{population\PYZus{}1} \PY{o}{=} \PY{l+m+mi}{3} \PY{o}{*} \PY{n}{np}\PY{o}{.}\PY{n}{random}\PY{o}{.}\PY{n}{randn}\PY{p}{(}\PY{l+m+mi}{100}\PY{p}{)} \PY{o}{+} \PY{l+m+mi}{5}
         
         \PY{k}{assert} \PY{n}{isclose}\PY{p}{(}\PY{n}{mean}\PY{p}{(}\PY{n}{population\PYZus{}0}\PY{p}{)}\PY{p}{,} \PY{n}{np}\PY{o}{.}\PY{n}{mean}\PY{p}{(}\PY{n}{population\PYZus{}0}\PY{p}{)}\PY{p}{,} \PY{n}{abs\PYZus{}tol}\PY{o}{=}\PY{l+m+mf}{1e\PYZhy{}6}\PY{p}{)}
         \PY{k}{assert} \PY{n}{isclose}\PY{p}{(}\PY{n}{mean}\PY{p}{(}\PY{n}{population\PYZus{}1}\PY{p}{)}\PY{p}{,} \PY{n}{np}\PY{o}{.}\PY{n}{mean}\PY{p}{(}\PY{n}{population\PYZus{}1}\PY{p}{)}\PY{p}{,} \PY{n}{abs\PYZus{}tol}\PY{o}{=}\PY{l+m+mf}{1e\PYZhy{}6}\PY{p}{)}
         \PY{k}{assert} \PY{n}{isclose}\PY{p}{(}\PY{n}{variance}\PY{p}{(}\PY{n}{population\PYZus{}0}\PY{p}{)}\PY{p}{,} \PY{n}{np}\PY{o}{.}\PY{n}{var}\PY{p}{(}\PY{n}{population\PYZus{}0}\PY{p}{)}\PY{p}{,} \PY{n}{abs\PYZus{}tol}\PY{o}{=}\PY{l+m+mf}{1e\PYZhy{}6}\PY{p}{)}
         \PY{k}{assert} \PY{n}{isclose}\PY{p}{(}\PY{n}{variance}\PY{p}{(}\PY{n}{population\PYZus{}1}\PY{p}{)}\PY{p}{,} \PY{n}{np}\PY{o}{.}\PY{n}{var}\PY{p}{(}\PY{n}{population\PYZus{}1}\PY{p}{)}\PY{p}{,} \PY{n}{abs\PYZus{}tol}\PY{o}{=}\PY{l+m+mf}{1e\PYZhy{}6}\PY{p}{)}
\end{Verbatim}


    \begin{center}\rule{0.5\linewidth}{\linethickness}\end{center}

\subsection{2: NumPy}\label{numpy}

The \texttt{NumPy} library lets us do fast, simple computing with
numbers in Python. We assume you have already taken Data 8, so you
should already be familiar with Numpy. We will not teach Numpy in the
course.

You should be able to understand the code in the following cells. If
not, review the following:

\begin{itemize}
\tightlist
\item
  \href{http://ds100.org/fa17/assets/notebooks/numpy/Numpy_Review.html}{DS100
  Numpy Review}
\item
  \href{http://cs231n.github.io/python-numpy-tutorial/\#numpy}{Condensed
  Numpy Review}
\item
  \href{https://docs.scipy.org/doc/numpy/user/quickstart.html}{The
  Official Numpy Tutorial}
\item
  \href{https://www.inferentialthinking.com/chapters/05/1/Arrays}{The
  Data 8 Textbook Chapter on Numpy}
\end{itemize}

    \textbf{Jupyter pro-tip}: Pull up the docs for any function in Jupyter
by running a cell with the function name and a \texttt{?} at the end:

    \begin{Verbatim}[commandchars=\\\{\}]
{\color{incolor}In [{\color{incolor}26}]:} np.arange\PY{o}{?}
\end{Verbatim}


    You can close the window at the bottom by pressing \texttt{esc} several
times.

    \textbf{Another Jupyter pro-tip}: Pull up the docs for any function in
Jupyter by typing the function name, then
\texttt{\textless{}Shift\textgreater{}-\textless{}Tab\textgreater{}} on
your keyboard. Super convenient when you forget the order of the
arguments to a function. You can press
\texttt{\textless{}Tab\textgreater{}} multiple times to expand the docs.

Try it on the function below:

    \begin{Verbatim}[commandchars=\\\{\}]
{\color{incolor}In [{\color{incolor}27}]:} \PY{n}{np}\PY{o}{.}\PY{n}{linspace}
\end{Verbatim}


\begin{Verbatim}[commandchars=\\\{\}]
{\color{outcolor}Out[{\color{outcolor}27}]:} <function numpy.core.function\_base.linspace>
\end{Verbatim}
            
    You can use the tips above to help you decipher the following code.

    \begin{Verbatim}[commandchars=\\\{\}]
{\color{incolor}In [{\color{incolor}28}]:} \PY{c+c1}{\PYZsh{} Let\PYZsq{}s take a 20\PYZhy{}sided die...}
         \PY{n}{NUM\PYZus{}FACES} \PY{o}{=} \PY{l+m+mi}{20}
         
         \PY{c+c1}{\PYZsh{} ...and roll it 4 times}
         \PY{n}{rolls} \PY{o}{=} \PY{l+m+mi}{4}
         
         \PY{c+c1}{\PYZsh{} What\PYZsq{}s the probability that all 4 rolls are different? It\PYZsq{}s:}
         \PY{c+c1}{\PYZsh{} 20/20 * 19/20 * 18/20 * 17/20}
         \PY{n}{prob\PYZus{}diff} \PY{o}{=} \PY{n}{np}\PY{o}{.}\PY{n}{prod}\PY{p}{(}\PY{p}{(}\PY{n}{NUM\PYZus{}FACES} \PY{o}{\PYZhy{}} \PY{n}{np}\PY{o}{.}\PY{n}{arange}\PY{p}{(}\PY{n}{rolls}\PY{p}{)}\PY{p}{)}
                             \PY{o}{/} \PY{n}{NUM\PYZus{}FACES}\PY{p}{)}
         \PY{n}{prob\PYZus{}diff}
\end{Verbatim}


\begin{Verbatim}[commandchars=\\\{\}]
{\color{outcolor}Out[{\color{outcolor}28}]:} 0.72675000000000001
\end{Verbatim}
            
    \begin{Verbatim}[commandchars=\\\{\}]
{\color{incolor}In [{\color{incolor}29}]:} \PY{c+c1}{\PYZsh{} Let\PYZsq{}s compute that probability for 1 roll, 2 rolls, ..., 20 rolls.}
         \PY{c+c1}{\PYZsh{} The array ys will contain:}
         \PY{c+c1}{\PYZsh{} }
         \PY{c+c1}{\PYZsh{} 20/20}
         \PY{c+c1}{\PYZsh{} 20/20 * 19/20}
         \PY{c+c1}{\PYZsh{} 20/20 * 18/20}
         \PY{c+c1}{\PYZsh{} ...}
         \PY{c+c1}{\PYZsh{} 20/20 * 19/20 * ... * 1/20}
         
         \PY{n}{xs} \PY{o}{=} \PY{n}{np}\PY{o}{.}\PY{n}{arange}\PY{p}{(}\PY{l+m+mi}{20}\PY{p}{)}
         \PY{n}{ys} \PY{o}{=} \PY{n}{np}\PY{o}{.}\PY{n}{cumprod}\PY{p}{(}\PY{p}{(}\PY{n}{NUM\PYZus{}FACES} \PY{o}{\PYZhy{}} \PY{n}{xs}\PY{p}{)} \PY{o}{/} \PY{n}{NUM\PYZus{}FACES}\PY{p}{)}
         
         \PY{c+c1}{\PYZsh{} Python slicing works on arrays too}
         \PY{n}{ys}\PY{p}{[}\PY{p}{:}\PY{l+m+mi}{5}\PY{p}{]}
\end{Verbatim}


\begin{Verbatim}[commandchars=\\\{\}]
{\color{outcolor}Out[{\color{outcolor}29}]:} array([ 1.     ,  0.95   ,  0.855  ,  0.72675,  0.5814 ])
\end{Verbatim}
            
    \begin{Verbatim}[commandchars=\\\{\}]
{\color{incolor}In [{\color{incolor}30}]:} \PY{c+c1}{\PYZsh{} plt is a data plotting library that we will discuss in a later lecture.}
         \PY{n}{plt}\PY{o}{.}\PY{n}{plot}\PY{p}{(}\PY{n}{xs}\PY{p}{,} \PY{n}{ys}\PY{p}{,} \PY{l+s+s1}{\PYZsq{}}\PY{l+s+s1}{o\PYZhy{}}\PY{l+s+s1}{\PYZsq{}}\PY{p}{)}
         \PY{n}{plt}\PY{o}{.}\PY{n}{xlabel}\PY{p}{(}\PY{l+s+s2}{\PYZdq{}}\PY{l+s+s2}{Num Rolls}\PY{l+s+s2}{\PYZdq{}}\PY{p}{)}
         \PY{n}{plt}\PY{o}{.}\PY{n}{ylabel}\PY{p}{(}\PY{l+s+s1}{\PYZsq{}}\PY{l+s+s1}{P(all different)}\PY{l+s+s1}{\PYZsq{}}\PY{p}{)}
\end{Verbatim}


\begin{Verbatim}[commandchars=\\\{\}]
{\color{outcolor}Out[{\color{outcolor}30}]:} Text(0,0.5,'P(all different)')
\end{Verbatim}
            
    \begin{center}
    \adjustimage{max size={0.9\linewidth}{0.9\paperheight}}{output_57_1.png}
    \end{center}
    { \hspace*{\fill} \\}
    
    \begin{Verbatim}[commandchars=\\\{\}]
{\color{incolor}In [{\color{incolor}31}]:} \PY{c+c1}{\PYZsh{} Mysterious...}
         \PY{n}{mystery} \PY{o}{=} \PY{n}{np}\PY{o}{.}\PY{n}{exp}\PY{p}{(}\PY{o}{\PYZhy{}}\PY{n}{xs} \PY{o}{*}\PY{o}{*} \PY{l+m+mi}{2} \PY{o}{/} \PY{p}{(}\PY{l+m+mi}{2} \PY{o}{*} \PY{n}{NUM\PYZus{}FACES}\PY{p}{)}\PY{p}{)}
         \PY{n}{mystery}
\end{Verbatim}


\begin{Verbatim}[commandchars=\\\{\}]
{\color{outcolor}Out[{\color{outcolor}31}]:} array([  1.00000000e+00,   9.75309912e-01,   9.04837418e-01,
                  7.98516219e-01,   6.70320046e-01,   5.35261429e-01,
                  4.06569660e-01,   2.93757700e-01,   2.01896518e-01,
                  1.31993843e-01,   8.20849986e-02,   4.85578213e-02,
                  2.73237224e-02,   1.46253347e-02,   7.44658307e-03,
                  3.60656314e-03,   1.66155727e-03,   7.28152539e-04,
                  3.03539138e-04,   1.20362805e-04])
\end{Verbatim}
            
    \begin{Verbatim}[commandchars=\\\{\}]
{\color{incolor}In [{\color{incolor}32}]:} \PY{c+c1}{\PYZsh{} If you\PYZsq{}re curious, this is the exponential approximation for our probability:}
         \PY{c+c1}{\PYZsh{} https://textbook.prob140.org/notebooks\PYZhy{}md/15\PYZus{}04\PYZus{}Exponential\PYZus{}Distribution.html}
         \PY{n}{plt}\PY{o}{.}\PY{n}{plot}\PY{p}{(}\PY{n}{xs}\PY{p}{,} \PY{n}{ys}\PY{p}{,} \PY{l+s+s1}{\PYZsq{}}\PY{l+s+s1}{o\PYZhy{}}\PY{l+s+s1}{\PYZsq{}}\PY{p}{,} \PY{n}{label}\PY{o}{=}\PY{l+s+s2}{\PYZdq{}}\PY{l+s+s2}{All Different}\PY{l+s+s2}{\PYZdq{}}\PY{p}{)}
         \PY{n}{plt}\PY{o}{.}\PY{n}{plot}\PY{p}{(}\PY{n}{xs}\PY{p}{,} \PY{n}{mystery}\PY{p}{,} \PY{l+s+s1}{\PYZsq{}}\PY{l+s+s1}{o\PYZhy{}}\PY{l+s+s1}{\PYZsq{}}\PY{p}{,} \PY{n}{label}\PY{o}{=}\PY{l+s+s2}{\PYZdq{}}\PY{l+s+s2}{Mystery}\PY{l+s+s2}{\PYZdq{}}\PY{p}{)}
         \PY{n}{plt}\PY{o}{.}\PY{n}{xlabel}\PY{p}{(}\PY{l+s+s2}{\PYZdq{}}\PY{l+s+s2}{Num Rolls}\PY{l+s+s2}{\PYZdq{}}\PY{p}{)}
         \PY{n}{plt}\PY{o}{.}\PY{n}{ylabel}\PY{p}{(}\PY{l+s+s1}{\PYZsq{}}\PY{l+s+s1}{P(all different)}\PY{l+s+s1}{\PYZsq{}}\PY{p}{)}
         \PY{n}{plt}\PY{o}{.}\PY{n}{legend}\PY{p}{(}\PY{p}{)}
\end{Verbatim}


\begin{Verbatim}[commandchars=\\\{\}]
{\color{outcolor}Out[{\color{outcolor}32}]:} <matplotlib.legend.Legend at 0x7fcf345b2da0>
\end{Verbatim}
            
    \begin{center}
    \adjustimage{max size={0.9\linewidth}{0.9\paperheight}}{output_59_1.png}
    \end{center}
    { \hspace*{\fill} \\}
    
    \subsubsection{Question 2}\label{question-2}

To test your understanding of Numpy we will work through some basic
image exercises. In the process we will explore visual perception and
color.

Images are 2-dimensional grids of pixels. Each pixel contains 3 values
between 0 and 1 that specify how much red, green, and blue go into each
pixel.

We can create images in NumPy:

    \begin{Verbatim}[commandchars=\\\{\}]
{\color{incolor}In [{\color{incolor}33}]:} \PY{n}{simple\PYZus{}image} \PY{o}{=} \PY{n}{np}\PY{o}{.}\PY{n}{array}\PY{p}{(}\PY{p}{[}
             \PY{p}{[}\PY{p}{[}  \PY{l+m+mi}{0}\PY{p}{,}   \PY{l+m+mi}{0}\PY{p}{,} \PY{l+m+mi}{0}\PY{p}{]}\PY{p}{,} \PY{p}{[}\PY{l+m+mf}{0.5}\PY{p}{,} \PY{l+m+mf}{0.5}\PY{p}{,} \PY{l+m+mf}{0.5}\PY{p}{]}\PY{p}{,} \PY{p}{[}\PY{l+m+mf}{1.0}\PY{p}{,} \PY{l+m+mf}{1.0}\PY{p}{,} \PY{l+m+mf}{1.0}\PY{p}{]}\PY{p}{]}\PY{p}{,} \PY{c+c1}{\PYZsh{} Grayscale pixels}
             \PY{p}{[}\PY{p}{[}\PY{l+m+mf}{1.0}\PY{p}{,}   \PY{l+m+mi}{0}\PY{p}{,} \PY{l+m+mi}{0}\PY{p}{]}\PY{p}{,} \PY{p}{[}  \PY{l+m+mi}{0}\PY{p}{,} \PY{l+m+mf}{1.0}\PY{p}{,}   \PY{l+m+mi}{0}\PY{p}{]}\PY{p}{,} \PY{p}{[}  \PY{l+m+mi}{0}\PY{p}{,}   \PY{l+m+mi}{0}\PY{p}{,} \PY{l+m+mf}{1.0}\PY{p}{]}\PY{p}{]}\PY{p}{,} \PY{c+c1}{\PYZsh{} Pure RGB pixels}
             \PY{p}{[}\PY{p}{[}\PY{l+m+mf}{0.5}\PY{p}{,} \PY{l+m+mf}{0.5}\PY{p}{,} \PY{l+m+mi}{0}\PY{p}{]}\PY{p}{,} \PY{p}{[}\PY{l+m+mf}{0.5}\PY{p}{,}   \PY{l+m+mi}{0}\PY{p}{,} \PY{l+m+mf}{0.5}\PY{p}{]}\PY{p}{,} \PY{p}{[}  \PY{l+m+mi}{0}\PY{p}{,} \PY{l+m+mf}{0.5}\PY{p}{,} \PY{l+m+mf}{0.5}\PY{p}{]}\PY{p}{]}\PY{p}{,} \PY{c+c1}{\PYZsh{} Blend of 2 colors}
         \PY{p}{]}\PY{p}{)}
         \PY{n}{simple\PYZus{}image}
\end{Verbatim}


\begin{Verbatim}[commandchars=\\\{\}]
{\color{outcolor}Out[{\color{outcolor}33}]:} array([[[ 0. ,  0. ,  0. ],
                 [ 0.5,  0.5,  0.5],
                 [ 1. ,  1. ,  1. ]],
         
                [[ 1. ,  0. ,  0. ],
                 [ 0. ,  1. ,  0. ],
                 [ 0. ,  0. ,  1. ]],
         
                [[ 0.5,  0.5,  0. ],
                 [ 0.5,  0. ,  0.5],
                 [ 0. ,  0.5,  0.5]]])
\end{Verbatim}
            
    We can then use the
\href{http://scikit-image.org/}{\texttt{scikit-image}} library to
display an image:

    \begin{Verbatim}[commandchars=\\\{\}]
{\color{incolor}In [{\color{incolor}34}]:} \PY{c+c1}{\PYZsh{} Curious how this method works? Try using skimage.io.imshow? to find out.}
         \PY{c+c1}{\PYZsh{} Or, you can always look at the docs for the method.}
         \PY{n}{skimage}\PY{o}{.}\PY{n}{io}\PY{o}{.}\PY{n}{imshow}\PY{p}{(}\PY{n}{simple\PYZus{}image}\PY{p}{)}
         \PY{n}{plt}\PY{o}{.}\PY{n}{grid}\PY{p}{(}\PY{k+kc}{False}\PY{p}{)} \PY{c+c1}{\PYZsh{} Disable matplotlib\PYZsq{}s grid lines}
\end{Verbatim}


    \begin{center}
    \adjustimage{max size={0.9\linewidth}{0.9\paperheight}}{output_63_0.png}
    \end{center}
    { \hspace*{\fill} \\}
    
    We can read in image files using the \texttt{skimage.io.imread} method.

\emph{Note that in many image formats (e.g., JPEG) image values are
numbers between 0 and 255 corresponding to a byte. Therefore we divide
each pixel value by 255 to obtain numbers between 0 and 1.}

    \begin{Verbatim}[commandchars=\\\{\}]
{\color{incolor}In [{\color{incolor}35}]:} \PY{n}{plt}\PY{o}{.}\PY{n}{figure}\PY{p}{(}\PY{n}{figsize}\PY{o}{=}\PY{p}{(}\PY{l+m+mi}{20}\PY{p}{,}\PY{l+m+mi}{10}\PY{p}{)}\PY{p}{)}
         
         \PY{c+c1}{\PYZsh{} Some image files (including .jpg files) have pixel values in between}
         \PY{c+c1}{\PYZsh{} 0 and 255 when read. We divide by 255 to scale the values between 0 and 1:}
         \PY{n}{pic} \PY{o}{=} \PY{n}{skimage}\PY{o}{.}\PY{n}{io}\PY{o}{.}\PY{n}{imread}\PY{p}{(}\PY{l+s+s1}{\PYZsq{}}\PY{l+s+s1}{target.jpg}\PY{l+s+s1}{\PYZsq{}}\PY{p}{)}\PY{o}{/}\PY{l+m+mi}{255}
         
         
         \PY{n}{skimage}\PY{o}{.}\PY{n}{io}\PY{o}{.}\PY{n}{imshow}\PY{p}{(}\PY{n}{pic}\PY{p}{)}
         \PY{n}{plt}\PY{o}{.}\PY{n}{grid}\PY{p}{(}\PY{k+kc}{False}\PY{p}{)} \PY{c+c1}{\PYZsh{} Disable matplotlib\PYZsq{}s grid lines}
\end{Verbatim}


    \begin{center}
    \adjustimage{max size={0.9\linewidth}{0.9\paperheight}}{output_65_0.png}
    \end{center}
    { \hspace*{\fill} \\}
    
    \paragraph{Question 2a}\label{question-2a}

Complete the following block of code to plot the Red, Green, and Blue
color channels separately. The resulting images should appear in black
and white.

\begin{itemize}
\tightlist
\item
  \textbf{Hint:} \texttt{pic{[}:,\ :,\ 0{]}} will slice the image to
  extract the red color channel. Plotting the resulting matrix will
  generate a black and white picture.*
\end{itemize}

    \begin{Verbatim}[commandchars=\\\{\}]
{\color{incolor}In [{\color{incolor}36}]:} \PY{n}{plt}\PY{o}{.}\PY{n}{figure}\PY{p}{(}\PY{n}{figsize}\PY{o}{=}\PY{p}{(}\PY{l+m+mi}{20}\PY{p}{,}\PY{l+m+mi}{10}\PY{p}{)}\PY{p}{)} 
         \PY{n}{channel\PYZus{}names} \PY{o}{=} \PY{p}{[}\PY{l+s+s2}{\PYZdq{}}\PY{l+s+s2}{Red}\PY{l+s+s2}{\PYZdq{}}\PY{p}{,} \PY{l+s+s2}{\PYZdq{}}\PY{l+s+s2}{Green}\PY{l+s+s2}{\PYZdq{}}\PY{p}{,} \PY{l+s+s2}{\PYZdq{}}\PY{l+s+s2}{Blue}\PY{l+s+s2}{\PYZdq{}}\PY{p}{]}
         
         \PY{c+c1}{\PYZsh{} Loop through index of each channel}
         \PY{k}{for} \PY{n}{channel} \PY{o+ow}{in} \PY{n+nb}{range}\PY{p}{(}\PY{l+m+mi}{3}\PY{p}{)}\PY{p}{:}
             \PY{c+c1}{\PYZsh{} Make a subplot}
             \PY{n}{plt}\PY{o}{.}\PY{n}{subplot}\PY{p}{(}\PY{l+m+mi}{1}\PY{p}{,}\PY{l+m+mi}{3}\PY{p}{,}\PY{n}{channel}\PY{o}{+}\PY{l+m+mi}{1}\PY{p}{)}
             \PY{c+c1}{\PYZsh{} FINISH THE CODE}
             \PY{c+c1}{\PYZsh{} Hint: you should write one line of code that calls}
             \PY{c+c1}{\PYZsh{} `skimage.io.imshow`}
             \PY{n}{pic} \PY{o}{=} \PY{n}{skimage}\PY{o}{.}\PY{n}{io}\PY{o}{.}\PY{n}{imread}\PY{p}{(}\PY{l+s+s1}{\PYZsq{}}\PY{l+s+s1}{target.jpg}\PY{l+s+s1}{\PYZsq{}}\PY{p}{)}
             \PY{n}{skimage}\PY{o}{.}\PY{n}{io}\PY{o}{.}\PY{n}{imshow}\PY{p}{(}\PY{n}{pic}\PY{p}{[}\PY{p}{:}\PY{p}{,}\PY{p}{:}\PY{p}{,}\PY{n}{channel}\PY{p}{]}\PY{p}{)}
             \PY{n}{plt}\PY{o}{.}\PY{n}{grid}\PY{p}{(}\PY{k+kc}{False}\PY{p}{)}
             \PY{n}{plt}\PY{o}{.}\PY{n}{title}\PY{p}{(}\PY{n}{channel\PYZus{}names}\PY{p}{[}\PY{n}{channel}\PY{p}{]}\PY{p}{)}
             \PY{n}{plt}\PY{o}{.}\PY{n}{suptitle}\PY{p}{(}\PY{l+s+s2}{\PYZdq{}}\PY{l+s+s2}{Separate channels of an image}\PY{l+s+s2}{\PYZdq{}}\PY{p}{)}
\end{Verbatim}


    \begin{center}
    \adjustimage{max size={0.9\linewidth}{0.9\paperheight}}{output_67_0.png}
    \end{center}
    { \hspace*{\fill} \\}
    
    \paragraph{Question 2b}\label{question-2b}

Surprisingly the human eye doesn't see all colors equally. To
demonstrate this we will study how blurring color channels affects image
appearance. First, we will try to blur each color channel individually.
Complete the following block of code using the
\texttt{skimage.filters.gaussian} blurring function (read the docs) to
render a blurry black and white image for each of the color channels.
You should set the standard deviation of the Gaussian blurring kernel
\texttt{sigma} to 10.

    \begin{Verbatim}[commandchars=\\\{\}]
{\color{incolor}In [{\color{incolor}37}]:} \PY{n}{plt}\PY{o}{.}\PY{n}{figure}\PY{p}{(}\PY{n}{figsize}\PY{o}{=}\PY{p}{(}\PY{l+m+mi}{20}\PY{p}{,}\PY{l+m+mi}{10}\PY{p}{)}\PY{p}{)}
         
         \PY{n}{sigma} \PY{o}{=} \PY{l+m+mi}{10}
         
         \PY{c+c1}{\PYZsh{} Loop through index of each channel}
         \PY{k}{for} \PY{n}{channel} \PY{o+ow}{in} \PY{n+nb}{range}\PY{p}{(}\PY{l+m+mi}{3}\PY{p}{)}\PY{p}{:}
             \PY{c+c1}{\PYZsh{} Make a subplot}
             \PY{n}{plt}\PY{o}{.}\PY{n}{subplot}\PY{p}{(}\PY{l+m+mi}{1}\PY{p}{,}\PY{l+m+mi}{3}\PY{p}{,}\PY{n}{channel}\PY{o}{+}\PY{l+m+mi}{1}\PY{p}{)}
             \PY{c+c1}{\PYZsh{} FINISH THE CODE }
             \PY{n}{pic} \PY{o}{=} \PY{n}{skimage}\PY{o}{.}\PY{n}{io}\PY{o}{.}\PY{n}{imread}\PY{p}{(}\PY{l+s+s1}{\PYZsq{}}\PY{l+s+s1}{target.jpg}\PY{l+s+s1}{\PYZsq{}}\PY{p}{)}
             \PY{n}{pic\PYZus{}filter}\PY{o}{=}\PY{n}{skimage}\PY{o}{.}\PY{n}{filters}\PY{o}{.}\PY{n}{gaussian}\PY{p}{(}\PY{n}{pic}\PY{p}{[}\PY{p}{:}\PY{p}{,}\PY{p}{:}\PY{p}{,}\PY{n}{channel}\PY{p}{]}\PY{p}{,}\PY{n}{sigma}\PY{o}{=}\PY{l+m+mi}{10}\PY{p}{)}
             \PY{n}{skimage}\PY{o}{.}\PY{n}{io}\PY{o}{.}\PY{n}{imshow}\PY{p}{(}\PY{n}{pic\PYZus{}filter}\PY{p}{)}
             \PY{n}{plt}\PY{o}{.}\PY{n}{grid}\PY{p}{(}\PY{k+kc}{False}\PY{p}{)}
             \PY{n}{plt}\PY{o}{.}\PY{n}{title}\PY{p}{(}\PY{n}{channel\PYZus{}names}\PY{p}{[}\PY{n}{channel}\PY{p}{]}\PY{p}{)}
             \PY{n}{plt}\PY{o}{.}\PY{n}{suptitle}\PY{p}{(}\PY{l+s+s2}{\PYZdq{}}\PY{l+s+s2}{Blurred channels of an image}\PY{l+s+s2}{\PYZdq{}}\PY{p}{)}
\end{Verbatim}


    \begin{center}
    \adjustimage{max size={0.9\linewidth}{0.9\paperheight}}{output_69_0.png}
    \end{center}
    { \hspace*{\fill} \\}
    
    \paragraph{Question 2c}\label{question-2c}

Using the following block of code:

\begin{Shaded}
\begin{Highlighting}[]
\NormalTok{pic_copy }\OperatorTok{=}\NormalTok{ pic.copy()}
\NormalTok{pic_copy[:, :, channel] }\OperatorTok{=}\NormalTok{ ...}
\NormalTok{skimage.io.imshow(pic_copy)}
\end{Highlighting}
\end{Shaded}

we can replace a color channel with a different black and white image.
Complete the following block of code to render three different versions
of the full color image with just one of the channels blurred.

    \begin{Verbatim}[commandchars=\\\{\}]
{\color{incolor}In [{\color{incolor}38}]:} \PY{n}{plt}\PY{o}{.}\PY{n}{figure}\PY{p}{(}\PY{n}{figsize}\PY{o}{=}\PY{p}{(}\PY{l+m+mi}{20}\PY{p}{,}\PY{l+m+mi}{10}\PY{p}{)}\PY{p}{)}
         
         \PY{n}{sigma} \PY{o}{=} \PY{l+m+mi}{10}
         
         \PY{c+c1}{\PYZsh{} Loop through index of each channel}
         \PY{k}{for} \PY{n}{channel} \PY{o+ow}{in} \PY{n+nb}{range}\PY{p}{(}\PY{l+m+mi}{3}\PY{p}{)}\PY{p}{:}
             \PY{c+c1}{\PYZsh{} Make a subplot}
             \PY{n}{plt}\PY{o}{.}\PY{n}{subplot}\PY{p}{(}\PY{l+m+mi}{1}\PY{p}{,}\PY{l+m+mi}{3}\PY{p}{,}\PY{n}{channel}\PY{o}{+}\PY{l+m+mi}{1}\PY{p}{)}
             \PY{c+c1}{\PYZsh{} FINISH THE CODE}
             \PY{n}{pic\PYZus{}copy} \PY{o}{=} \PY{n}{pic}\PY{o}{.}\PY{n}{copy}\PY{p}{(}\PY{p}{)}
             \PY{n}{pic\PYZus{}copy}\PY{p}{[}\PY{p}{:}\PY{p}{,} \PY{p}{:}\PY{p}{,} \PY{n}{channel}\PY{p}{]} \PY{o}{=} \PY{n}{skimage}\PY{o}{.}\PY{n}{filters}\PY{o}{.}\PY{n}{gaussian}\PY{p}{(}\PY{n}{pic}\PY{p}{[}\PY{p}{:}\PY{p}{,}\PY{p}{:}\PY{p}{,}\PY{n}{channel}\PY{p}{]}\PY{p}{,}\PY{n}{sigma}\PY{o}{=}\PY{l+m+mi}{10}\PY{p}{)}
             \PY{n}{skimage}\PY{o}{.}\PY{n}{io}\PY{o}{.}\PY{n}{imshow}\PY{p}{(}\PY{n}{pic\PYZus{}copy}\PY{p}{)}
             
             \PY{n}{plt}\PY{o}{.}\PY{n}{grid}\PY{p}{(}\PY{k+kc}{False}\PY{p}{)}
             \PY{n}{plt}\PY{o}{.}\PY{n}{title}\PY{p}{(}\PY{n}{channel\PYZus{}names}\PY{p}{[}\PY{n}{channel}\PY{p}{]}\PY{p}{)}
             \PY{n}{plt}\PY{o}{.}\PY{n}{suptitle}\PY{p}{(}\PY{l+s+s2}{\PYZdq{}}\PY{l+s+s2}{Blurred channels of an image}\PY{l+s+s2}{\PYZdq{}}\PY{p}{)}
\end{Verbatim}


    \begin{center}
    \adjustimage{max size={0.9\linewidth}{0.9\paperheight}}{output_71_0.png}
    \end{center}
    { \hspace*{\fill} \\}
    
    \begin{Verbatim}[commandchars=\\\{\}]
{\color{incolor}In [{\color{incolor}39}]:} \PY{k}{assert} \PY{k+kc}{True} \PY{c+c1}{\PYZsh{} We\PYZsq{}re giving you this free point for 2a, 2b, 2c, but please do the question}
\end{Verbatim}


    \paragraph{Question 2d}\label{question-2d}

Each image should appear slightly different. Which one is the blurriest
and which is the sharpest? Write a short description of what you see in
the cell below.

 *This icon means you will need to write in text response in the cell
below using English.

\textbf{Hint:} I observe ... . On possible explanation for this is ... .

    I observe that the second image with green channel blurred is blurriest
and the first image with red channel blurred is sharpest. One possible
explanation for this is eyes are most sensitive to details of green and
least sensitive to details of red.

    \subsection{3: Debugging and Testing}\label{debugging-and-testing}

In this course, you will write a significant amount of code and
inevitably encounter some bugs in the process. You should feel
comfortable reading stack traces, debugging erroneous code, and testing
the code you write.

Regarding debugging methods, we will be using \texttt{pdb} (Python
Debugger), linked here: https://docs.python.org/3/library/pdb.html.
\texttt{pdb} provides an interactive debugger for python programs and
works within a notebook. It has a lot of useful features such as setting
conditional breakpoints and single stepping at the source line level.

We will only be enforcing \texttt{pdb} on this first assignment. You can
still use print statements or other methods of your choice on other
assignments, although when the code becomes increasingly complex you may
find it harder to debug without \texttt{pdb}.

Please walk through the \texttt{debugging.ipynb} before proceeding.

    \subsubsection{Question 3}\label{question-3}

The following blocks of code contains some errors. Find the bug and fix
it in the same block so that the function does what it is expected to
do. Feel free to create new cells to debug and test your code. Once you
are confident that your code works, change the \_debug to \_fixed to run
our tests.

\begin{itemize}
\tightlist
\item
  \textbf{Hint:} Try to use array operations to simplify your code.
\end{itemize}

    \paragraph{Question 3a}\label{question-3a}

    \begin{Verbatim}[commandchars=\\\{\}]
{\color{incolor}In [{\color{incolor}40}]:} \PY{k}{def} \PY{n+nf}{multiply\PYZus{}list\PYZus{}debug}\PY{p}{(}\PY{n}{some\PYZus{}list}\PY{p}{,} \PY{n}{multiplier}\PY{p}{)}\PY{p}{:}
             \PY{l+s+sd}{\PYZdq{}\PYZdq{}\PYZdq{}}
         \PY{l+s+sd}{    Given a list and a multiplier, return a list containing each element in the list multiplied by }
         \PY{l+s+sd}{    the multiplier. }
         \PY{l+s+sd}{    You may assume that the lists are well formed and not nested more than once, }
         \PY{l+s+sd}{    ie: }
         \PY{l+s+sd}{        [[1,2,3],[4,5,6]] is valid since it is nested once and each nested list }
         \PY{l+s+sd}{            is of the same type and length,}
         \PY{l+s+sd}{        [[1,2,3,[4,5,6]] is not valid  since it is nested more than once and }
         \PY{l+s+sd}{            each nested list is not of same type}
         \PY{l+s+sd}{    Args:}
         \PY{l+s+sd}{        some\PYZus{}list: a list of integers that may or may not be nested}
         \PY{l+s+sd}{    Returns:}
         \PY{l+s+sd}{        A list (or hint: array) containing each element in the list multipled by the multiplier}
         \PY{l+s+sd}{    e.g.}
         \PY{l+s+sd}{    [1,2,3], 5 should return [5,10,15]}
         \PY{l+s+sd}{    [[1,2,3], [4,5,6], [7,8,9]], 0.5 should return [[0.5,1,1.5],[2,2.5,3],[3.5,4,4.5]]}
         \PY{l+s+sd}{    \PYZdq{}\PYZdq{}\PYZdq{}}
             \PY{n}{some\PYZus{}list\PYZus{}multiplied} \PY{o}{=} \PY{n}{some\PYZus{}list} \PY{o}{*} \PY{n}{multiplier}
             \PY{k}{return} \PY{n}{some\PYZus{}list\PYZus{}multiplied}
\end{Verbatim}


    \begin{Verbatim}[commandchars=\\\{\}]
{\color{incolor}In [{\color{incolor}41}]:} \PY{k}{def} \PY{n+nf}{multiply\PYZus{}list\PYZus{}fixed}\PY{p}{(}\PY{n}{some\PYZus{}list}\PY{p}{,} \PY{n}{multiplier}\PY{p}{)}\PY{p}{:}
             \PY{k}{return} \PY{n}{np}\PY{o}{.}\PY{n}{array}\PY{p}{(}\PY{n}{some\PYZus{}list}\PY{p}{)}\PY{o}{*}\PY{n}{multiplier}
\end{Verbatim}


    \begin{Verbatim}[commandchars=\\\{\}]
{\color{incolor}In [{\color{incolor}42}]:} \PY{k}{assert} \PY{n}{np}\PY{o}{.}\PY{n}{array\PYZus{}equal}\PY{p}{(}\PY{n}{multiply\PYZus{}list\PYZus{}fixed}\PY{p}{(}\PY{p}{[}\PY{l+m+mi}{1}\PY{p}{,}\PY{l+m+mi}{2}\PY{p}{,}\PY{l+m+mi}{3}\PY{p}{]}\PY{p}{,}\PY{l+m+mi}{5}\PY{p}{)}\PY{p}{,} \PY{p}{[}\PY{l+m+mi}{5}\PY{p}{,}\PY{l+m+mi}{10}\PY{p}{,}\PY{l+m+mi}{15}\PY{p}{]}\PY{p}{)}
         \PY{k}{assert} \PY{n}{np}\PY{o}{.}\PY{n}{array\PYZus{}equal}\PY{p}{(}\PY{n}{multiply\PYZus{}list\PYZus{}fixed}\PY{p}{(}\PY{p}{[}\PY{p}{[}\PY{l+m+mi}{1}\PY{p}{,}\PY{l+m+mi}{2}\PY{p}{,}\PY{l+m+mi}{3}\PY{p}{]}\PY{p}{,} \PY{p}{[}\PY{l+m+mi}{4}\PY{p}{,}\PY{l+m+mi}{5}\PY{p}{,}\PY{l+m+mi}{6}\PY{p}{]}\PY{p}{,} \PY{p}{[}\PY{l+m+mi}{7}\PY{p}{,}\PY{l+m+mi}{8}\PY{p}{,}\PY{l+m+mi}{9}\PY{p}{]}\PY{p}{]}\PY{p}{,} \PY{l+m+mf}{0.5}\PY{p}{)}\PY{p}{,} \PY{p}{[}\PY{p}{[}\PY{l+m+mf}{0.5}\PY{p}{,}\PY{l+m+mi}{1}\PY{p}{,}\PY{l+m+mf}{1.5}\PY{p}{]}\PY{p}{,}\PY{p}{[}\PY{l+m+mi}{2}\PY{p}{,}\PY{l+m+mf}{2.5}\PY{p}{,}\PY{l+m+mi}{3}\PY{p}{]}\PY{p}{,}\PY{p}{[}\PY{l+m+mf}{3.5}\PY{p}{,}\PY{l+m+mi}{4}\PY{p}{,}\PY{l+m+mf}{4.5}\PY{p}{]}\PY{p}{]}\PY{p}{)}
\end{Verbatim}


    \paragraph{Question 3b}\label{question-3b}

    \begin{Verbatim}[commandchars=\\\{\}]
{\color{incolor}In [{\color{incolor}43}]:} \PY{k}{def} \PY{n+nf}{find\PYZus{}all\PYZus{}under\PYZus{}10\PYZus{}debug}\PY{p}{(}\PY{n}{some\PYZus{}list}\PY{p}{)}\PY{p}{:}
             \PY{l+s+sd}{\PYZdq{}\PYZdq{}\PYZdq{}}
         \PY{l+s+sd}{    Given a list, return a list containing all elements that are strictly less than 10.}
         \PY{l+s+sd}{    You may assume that the lists are well formed and not nested more than once, }
         \PY{l+s+sd}{    ie: }
         \PY{l+s+sd}{        [[1,2,3],[4,5,6]] is valid since it is nested once and each nested list }
         \PY{l+s+sd}{            is of the same type, }
         \PY{l+s+sd}{        [[1,2,3,[4,5,6]] is not valid  since it is nested more than once and }
         \PY{l+s+sd}{            each nested list is not of same type}
         \PY{l+s+sd}{    Args:}
         \PY{l+s+sd}{        some\PYZus{}list: a list of integers that may or may not be nested}
         \PY{l+s+sd}{    Returns:}
         \PY{l+s+sd}{        A list (or array) of all elements that are less than 10}
         \PY{l+s+sd}{    e.g.}
         \PY{l+s+sd}{    [1,2,...,20] should return [1,2,...,9]}
         \PY{l+s+sd}{    [[1,2,3], [4,5,6], [20,21,22]] should return [1,2,3,4,5,6]}
         \PY{l+s+sd}{    \PYZdq{}\PYZdq{}\PYZdq{}}
             \PY{n}{all\PYZus{}under\PYZus{}10} \PY{o}{=} \PY{n}{np}\PY{o}{.}\PY{n}{array}\PY{p}{(}\PY{p}{[}\PY{p}{]}\PY{p}{)}
             \PY{k}{for} \PY{n}{item} \PY{o+ow}{in} \PY{n}{some\PYZus{}list}\PY{p}{:}
                 \PY{k}{if} \PY{n}{item} \PY{o}{\PYZlt{}} \PY{l+m+mi}{10}\PY{p}{:}
                     \PY{n}{np}\PY{o}{.}\PY{n}{append}\PY{p}{(}\PY{n}{all\PYZus{}under\PYZus{}10}\PY{p}{,}\PY{n}{item}\PY{p}{)}
             \PY{k}{return} \PY{n}{all\PYZus{}under\PYZus{}10}
\end{Verbatim}


    \begin{Verbatim}[commandchars=\\\{\}]
{\color{incolor}In [{\color{incolor}44}]:} \PY{k}{def} \PY{n+nf}{find\PYZus{}all\PYZus{}under\PYZus{}10\PYZus{}fixed}\PY{p}{(}\PY{n}{some\PYZus{}list}\PY{p}{)}\PY{p}{:}
             \PY{n}{all\PYZus{}under\PYZus{}10} \PY{o}{=} \PY{n}{np}\PY{o}{.}\PY{n}{array}\PY{p}{(}\PY{p}{[}\PY{p}{]}\PY{p}{)}
             \PY{k}{for} \PY{n}{item} \PY{o+ow}{in} \PY{n}{some\PYZus{}list}\PY{p}{:}
                 \PY{n}{item}\PY{o}{=}\PY{n}{np}\PY{o}{.}\PY{n}{array}\PY{p}{(}\PY{n}{item}\PY{p}{)}
                 \PY{k}{if} \PY{p}{(}\PY{n}{item} \PY{o}{\PYZlt{}} \PY{l+m+mi}{10}\PY{p}{)}\PY{o}{.}\PY{n}{any}\PY{p}{(}\PY{p}{)}\PY{p}{:}
                     \PY{n}{all\PYZus{}under\PYZus{}10} \PY{o}{=} \PY{n}{np}\PY{o}{.}\PY{n}{append}\PY{p}{(}\PY{n}{all\PYZus{}under\PYZus{}10}\PY{p}{,}\PY{n}{item}\PY{p}{[}\PY{n}{item} \PY{o}{\PYZlt{}} \PY{l+m+mi}{10}\PY{p}{]}\PY{p}{)}
             \PY{n}{all\PYZus{}under\PYZus{}10} \PY{o}{=} \PY{p}{[}\PY{n}{i} \PY{k}{for} \PY{n}{i} \PY{o+ow}{in} \PY{n}{all\PYZus{}under\PYZus{}10}\PY{p}{]}
             \PY{k}{return} \PY{n}{all\PYZus{}under\PYZus{}10}
\end{Verbatim}


    \begin{Verbatim}[commandchars=\\\{\}]
{\color{incolor}In [{\color{incolor}45}]:} \PY{k}{assert} \PY{n}{np}\PY{o}{.}\PY{n}{array\PYZus{}equal}\PY{p}{(}\PY{n}{find\PYZus{}all\PYZus{}under\PYZus{}10\PYZus{}fixed}\PY{p}{(}\PY{n}{np}\PY{o}{.}\PY{n}{arange}\PY{p}{(}\PY{l+m+mi}{20}\PY{p}{)}\PY{p}{)}\PY{p}{,} \PY{n}{np}\PY{o}{.}\PY{n}{arange}\PY{p}{(}\PY{l+m+mi}{10}\PY{p}{)}\PY{p}{)}
         \PY{k}{assert} \PY{n}{np}\PY{o}{.}\PY{n}{array\PYZus{}equal}\PY{p}{(}\PY{n}{find\PYZus{}all\PYZus{}under\PYZus{}10\PYZus{}fixed}\PY{p}{(}\PY{p}{[}\PY{p}{[}\PY{l+m+mi}{1}\PY{p}{,}\PY{l+m+mi}{2}\PY{p}{,}\PY{l+m+mi}{3}\PY{p}{]}\PY{p}{,} \PY{p}{[}\PY{l+m+mi}{4}\PY{p}{,}\PY{l+m+mi}{5}\PY{p}{,}\PY{l+m+mi}{6}\PY{p}{]}\PY{p}{,} \PY{p}{[}\PY{l+m+mi}{20}\PY{p}{,}\PY{l+m+mi}{21}\PY{p}{,}\PY{l+m+mi}{22}\PY{p}{]}\PY{p}{]}\PY{p}{)}\PY{p}{,}\PY{p}{[}\PY{l+m+mi}{1}\PY{p}{,}\PY{l+m+mi}{2}\PY{p}{,}\PY{l+m+mi}{3}\PY{p}{,}\PY{l+m+mi}{4}\PY{p}{,}\PY{l+m+mi}{5}\PY{p}{,}\PY{l+m+mi}{6}\PY{p}{]}\PY{p}{)}
\end{Verbatim}


    \paragraph{Question 3c}\label{question-3c}

    \begin{Verbatim}[commandchars=\\\{\}]
{\color{incolor}In [{\color{incolor}46}]:} \PY{k}{def} \PY{n+nf}{cat\PYZus{}debug}\PY{p}{(}\PY{n}{cat\PYZus{}data}\PY{p}{)}\PY{p}{:}
             \PY{l+s+sd}{\PYZdq{}\PYZdq{}\PYZdq{}}
         \PY{l+s+sd}{    Given a 2d array containing cat names and weights, find the cats that weigh less than 10 lbs }
         \PY{l+s+sd}{    and return a 2d array of those cats after adding 3 lbs to their weights.}
         \PY{l+s+sd}{    Args:}
         \PY{l+s+sd}{        cat\PYZus{}data: a 2d array containing cat names and their corresponding weights}
         \PY{l+s+sd}{    Returns:}
         \PY{l+s+sd}{        A 2d list (or array) of cats less than 10 lbs and their weights + 3 lbs}
         \PY{l+s+sd}{    e.g.}
         \PY{l+s+sd}{    [[\PYZsq{}Jordan\PYZsq{},8],[\PYZsq{}Michael\PYZsq{},11],[\PYZsq{}Peter\PYZsq{}, 7]] should return [[\PYZsq{}Jordan\PYZsq{},11],[\PYZsq{}Peter\PYZsq{}, 10]]}
         \PY{l+s+sd}{    }
         \PY{l+s+sd}{    Hint: What happens when a list or array contains data of multiple types?}
         \PY{l+s+sd}{    Hint: checkout arr.astype for type casting}
         \PY{l+s+sd}{    \PYZdq{}\PYZdq{}\PYZdq{}}
             \PY{n}{list\PYZus{}of\PYZus{}cats} \PY{o}{=} \PY{p}{[}\PY{p}{]}
             \PY{k}{for} \PY{n}{cat} \PY{o+ow}{in} \PY{n}{cat\PYZus{}data}\PY{p}{:}
                 \PY{k}{if} \PY{n}{cat}\PY{p}{[}\PY{l+m+mi}{1}\PY{p}{]} \PY{o}{\PYZlt{}} \PY{l+m+mi}{10}\PY{p}{:}
                     \PY{n}{list\PYZus{}of\PYZus{}cats}\PY{o}{.}\PY{n}{append}\PY{p}{(}\PY{n}{cat}\PY{p}{)}
             \PY{k}{for} \PY{n}{cat} \PY{o+ow}{in} \PY{n}{list\PYZus{}of\PYZus{}cats}\PY{p}{:}
                 \PY{n}{cat} \PY{o}{=} \PY{n}{cat} \PY{o}{+} \PY{l+m+mi}{3}
             \PY{k}{return} \PY{n}{list\PYZus{}of\PYZus{}cats}
\end{Verbatim}


    \begin{Verbatim}[commandchars=\\\{\}]
{\color{incolor}In [{\color{incolor}47}]:} \PY{k}{def} \PY{n+nf}{cat\PYZus{}fixed}\PY{p}{(}\PY{n}{cat\PYZus{}data}\PY{p}{)}\PY{p}{:}
             \PY{n}{list\PYZus{}of\PYZus{}cats} \PY{o}{=} \PY{p}{[}\PY{p}{]}
             \PY{k}{for} \PY{n}{cat} \PY{o+ow}{in} \PY{n}{cat\PYZus{}data}\PY{p}{:}
                 \PY{k}{if} \PY{n+nb}{int}\PY{p}{(}\PY{n}{cat}\PY{p}{[}\PY{l+m+mi}{1}\PY{p}{]}\PY{p}{)}\PY{o}{\PYZlt{}} \PY{l+m+mi}{10}\PY{p}{:}
                     \PY{n}{list\PYZus{}of\PYZus{}cats}\PY{o}{.}\PY{n}{append}\PY{p}{(}\PY{n}{cat}\PY{p}{)}
             \PY{k}{for} \PY{n}{cat} \PY{o+ow}{in} \PY{n}{list\PYZus{}of\PYZus{}cats}\PY{p}{:}
                 \PY{n}{cat}\PY{p}{[}\PY{l+m+mi}{1}\PY{p}{]} \PY{o}{=} \PY{n+nb}{int}\PY{p}{(}\PY{n}{cat}\PY{p}{[}\PY{l+m+mi}{1}\PY{p}{]}\PY{p}{)} \PY{o}{+} \PY{l+m+mi}{3}
             \PY{k}{return} \PY{n}{list\PYZus{}of\PYZus{}cats}
\end{Verbatim}


    \begin{Verbatim}[commandchars=\\\{\}]
{\color{incolor}In [{\color{incolor}48}]:} \PY{n}{cat\PYZus{}data} \PY{o}{=} \PY{n}{np}\PY{o}{.}\PY{n}{array}\PY{p}{(}\PY{p}{[}\PY{p}{[}\PY{l+s+s1}{\PYZsq{}}\PY{l+s+s1}{Jordan}\PY{l+s+s1}{\PYZsq{}}\PY{p}{,}\PY{l+m+mi}{8}\PY{p}{]}\PY{p}{,}\PY{p}{[}\PY{l+s+s1}{\PYZsq{}}\PY{l+s+s1}{Michael}\PY{l+s+s1}{\PYZsq{}}\PY{p}{,}\PY{l+m+mi}{11}\PY{p}{]}\PY{p}{,}\PY{p}{[}\PY{l+s+s1}{\PYZsq{}}\PY{l+s+s1}{Peter}\PY{l+s+s1}{\PYZsq{}}\PY{p}{,} \PY{l+m+mi}{7}\PY{p}{]}\PY{p}{]}\PY{p}{)}
         \PY{k}{assert} \PY{n}{np}\PY{o}{.}\PY{n}{array\PYZus{}equal}\PY{p}{(}\PY{n}{cat\PYZus{}fixed}\PY{p}{(}\PY{n}{cat\PYZus{}data}\PY{p}{)}\PY{p}{,} \PY{p}{[}\PY{p}{[}\PY{l+s+s1}{\PYZsq{}}\PY{l+s+s1}{Jordan}\PY{l+s+s1}{\PYZsq{}}\PY{p}{,}\PY{l+m+mi}{11}\PY{p}{]}\PY{p}{,}\PY{p}{[}\PY{l+s+s1}{\PYZsq{}}\PY{l+s+s1}{Peter}\PY{l+s+s1}{\PYZsq{}}\PY{p}{,} \PY{l+m+mi}{10}\PY{p}{]}\PY{p}{]}\PY{p}{)}
\end{Verbatim}


    \begin{center}\rule{0.5\linewidth}{\linethickness}\end{center}

\subsection{4: Multivariable Calculus, Linear Algebra, and
Probability}\label{multivariable-calculus-linear-algebra-and-probability}

The following questions ask you to recall your knowledge of
multivariable calculus, linear algebra, and probability. We will use
some of the most fundamental concepts from each discipline in this
class, so the following problems should at least seem familiar to you.

For the following problems, you should use LaTeX to format your answer.
If you aren't familiar with LaTeX, not to worry. It's not hard to use in
a Jupyter notebook. Just place your math in between dollar signs:

\textbackslash{}\$ f(x) = 2x \textbackslash{}\$ becomes \$ f(x) = 2x \$.

If you have a longer equation, use double dollar signs:

\textbackslash{}\(\\\) \sum\_\{i=0\}\^{}n i\^{}2 \textbackslash{}\(\\\)
becomes:

\[ \sum_{i=0}^n i^2 \].

Here are some handy notation:

\begin{longtable}[]{@{}ll@{}}
\toprule
\begin{minipage}[b]{0.05\columnwidth}\raggedright\strut
Output\strut
\end{minipage} & \begin{minipage}[b]{0.05\columnwidth}\raggedright\strut
Latex\strut
\end{minipage}\tabularnewline
\midrule
\endhead
\begin{minipage}[t]{0.05\columnwidth}\raggedright\strut
\[x^{a + b}\]\strut
\end{minipage} & \begin{minipage}[t]{0.05\columnwidth}\raggedright\strut
\texttt{x\^{}\{a\ +\ b\}}\strut
\end{minipage}\tabularnewline
\begin{minipage}[t]{0.05\columnwidth}\raggedright\strut
\[x_{a + b}\]\strut
\end{minipage} & \begin{minipage}[t]{0.05\columnwidth}\raggedright\strut
\texttt{x\_\{a\ +\ b\}}\strut
\end{minipage}\tabularnewline
\begin{minipage}[t]{0.05\columnwidth}\raggedright\strut
\[\frac{a}{b}\]\strut
\end{minipage} & \begin{minipage}[t]{0.05\columnwidth}\raggedright\strut
\texttt{\textbackslash{}frac\{a\}\{b\}}\strut
\end{minipage}\tabularnewline
\begin{minipage}[t]{0.05\columnwidth}\raggedright\strut
\[\sqrt{a + b}\]\strut
\end{minipage} & \begin{minipage}[t]{0.05\columnwidth}\raggedright\strut
\texttt{\textbackslash{}sqrt\{a\ +\ b\}}\strut
\end{minipage}\tabularnewline
\begin{minipage}[t]{0.05\columnwidth}\raggedright\strut
\[\{ \alpha, \beta, \gamma, \pi, \mu, \sigma^2  \}\]\strut
\end{minipage} & \begin{minipage}[t]{0.05\columnwidth}\raggedright\strut
\texttt{\textbackslash{}\{\ \textbackslash{}alpha,\ \textbackslash{}beta,\ \textbackslash{}gamma,\ \textbackslash{}pi,\ \textbackslash{}mu,\ \textbackslash{}sigma\^{}2\ \ \textbackslash{}\}}\strut
\end{minipage}\tabularnewline
\begin{minipage}[t]{0.05\columnwidth}\raggedright\strut
\[\sum_{x=1}^{100}\]\strut
\end{minipage} & \begin{minipage}[t]{0.05\columnwidth}\raggedright\strut
\texttt{\textbackslash{}sum\_\{x=1\}\^{}\{100\}}\strut
\end{minipage}\tabularnewline
\begin{minipage}[t]{0.05\columnwidth}\raggedright\strut
\[\frac{\partial}{\partial x} \]\strut
\end{minipage} & \begin{minipage}[t]{0.05\columnwidth}\raggedright\strut
\texttt{\textbackslash{}frac\{\textbackslash{}partial\}\{\textbackslash{}partial\ x\}}\strut
\end{minipage}\tabularnewline
\begin{minipage}[t]{0.05\columnwidth}\raggedright\strut
\[\begin{bmatrix} 2x + 4y \\ 4x + 6y^2 \\ \end{bmatrix}\]\strut
\end{minipage} & \begin{minipage}[t]{0.05\columnwidth}\raggedright\strut
\texttt{\textbackslash{}begin\{bmatrix\}\ 2x\ +\ 4y\ \textbackslash{}\textbackslash{}\ 4x\ +\ 6y\^{}2\ \textbackslash{}\textbackslash{}\ \textbackslash{}end\{bmatrix\}}\strut
\end{minipage}\tabularnewline
\bottomrule
\end{longtable}

\href{https://www.sharelatex.com/learn/Mathematical_expressions}{For
more about basic LaTeX formatting, you can read this article.}

If you have trouble with these topics, we suggest reviewing:

\begin{itemize}
\tightlist
\item
  \href{https://www.khanacademy.org/math/multivariable-calculus}{Khan
  Academy's Multivariable Calculus}
\item
  \href{https://www.khanacademy.org/math/linear-algebra}{Khan Academy's
  Linear Algebra}
\item
  \href{https://www.khanacademy.org/math/statistics-probability}{Khan
  Academy's Statistics and Probability}
\end{itemize}

    \subsubsection{Question 4}\label{question-4}

\paragraph{Question 4a}\label{question-4a}

Recall that summation (or sigma notation) is a way of expressing a long
sum in a concise way. Let \(a_1, a_2, ..., a_n \in \mathbb{R}\) and
\(x_1, x_2, ..., x_n \in \mathbb{R}\) be collections of real numbers.
When you see \(x_i\), you can think of the \(i\) as an index for the
\(i^{th}\) \(x\). For example \(x_2\) is the second \(x\) value in the
list \(x_1, x_2, ..., x_n\). We define sigma notation as follows:

\[ \sum_{i=1}^n a_i x_i = a_1 x_1 + a_2 x_2 + ... + a_n x_n \]

We commonly use sigma notation to compactly write the definition of the
arithmetic mean (commonly known as the \texttt{average}):

\[ \bar{x} = \dfrac{1}{n} \left(x_1 + x_2 + ... + x_n \right) = \dfrac{1}{n} \sum_{i=1}^{n} x_i \]

\textbf{For each of the statements below, either prove that it is true
by using definitions or show that it is false by providing a
counterexample.}

\textbf{Note:} *This icon means you will need to write in text response
in the cell below using English + \(\LaTeX\).

    \paragraph{Statement I}\label{statement-i}

\(\dfrac{\sum_{i=1}^{n} a_i x_i}{\sum_{i=1}^{n} a_i} = \sum_{i=1}^n x_i\)

    False. Counterexample: \(\frac{1*1+1*1}{1+1}\neq{1+1}\)

    \paragraph{Statement II}\label{statement-ii}

\(\sum_{i=1}^{n} x_1 = nx_1\)

    True. Prove: \(\sum_{i=1}^nx_1=x_1+x_1+...+x_1=nx_1\). Sum of n
\(x_1\)'s and \(x_1\) is a constant is this equation.

    \paragraph{Statement III}\label{statement-iii}

\$\sum\_\{i=1\}\^{}\{n\} a\_3 x\_i = n a\_3 \bar\{x\} \$

    True.
Prove:\(\sum_{i=1}^na_3x_i=a_3\sum_{i=1}^nx_i=a_3*\frac{x_1+x_2+...+x_n}{n}*n=na_3\bar x\)

    \paragraph{Statement IV}\label{statement-iv}

\$\sum\_\{i=1\}\^{}\{n\} a\_i x\_i = n \bar\{a\} \bar\{x\} \$

    False. Counterexample: \(1*1+3*3 \neq 2*2*2\)

    \textbf{Note:}

We can also generalize the summation concepts above to multiple indices:
consider an array of values \(x_{ij}\)

\[ \begin{bmatrix} x_{1,1} & x_{1, 2} & ... & x_{1, n} \\ 
x_{2,1} & x_{2, 2} & ... & x_{2, n} \\ 
\vdots  & \vdots   & \ddots & \vdots   \\ 
x_{n,1} & x_{n, 2} & ... & x_{n, n} \\ 
\end{bmatrix} \]

By convention, the first index refers to the row and the second index
references the column. e.g. \(x_{2, 4}\) is the value in the second row
and the fourth column. For multi-indexed arrays like this, we can write
down the sum of all the values by evoking sigma notation multiple times:

\begin{align*} 
    \sum_{i=1}^{n} \sum_{j=1}^{n} x_{i,j}
    &= \sum_{i=1}^{n} \left(\sum_{j=1}^{n} x_{i,j} \right) \\
    &= \sum_{i=1}^{n} \left(x_{i,1} + x_{i,2} + ... + x_{i,n}\right) \\
    &= \sum_{i=1}^{n} x_{i,1} + \sum_{i=1}^{n} x_{i,2} + ... + \sum_{i=1}^{n} x_{i,n} \\
    &= \left(x_{1,1} + x_{1,2} + ... + x_{1,n}\right) + \left(x_{2,1} + x_{2,2} + ... + x_{2,n}\right) + ... +  \left(x_{n,1} + x_{n,2} + ... + x_{n,n}\right) \\
    &= x_{1,1} + x_{1,2} + ... + x_{1,n} + x_{2,1} + x_{2,2} + ... + x_{2,n} + ... + ... + x_{n,1} + x_{n,2} + ... + x_{n,n}
\end{align*}

    \paragraph{Question 4b}\label{question-4b}

Manipulate the following expression to show that it can be written as a
sum of powers of \(x\) (e.g. \(3x + 5x^2 + 8x^3\)):

\[
    \ln \left( 3 e^{2  x} e^{\frac{1}{x^2}} \right)
\]

    \(ln(3e^{2x}e^{\frac{1}{x^2}})=ln(3)+ln(e^{2x})+ln(e^{\frac{1} {x^2}})=ln(3)+2ln(e)x+ln(e)x^{-2}=ln(3)+2x+x^{-2}\)

    \paragraph{Question 4c}\label{question-4c}

Suppose we have the following scalar-valued function on \(x\) and \(y\):

\[ f(x, y) = x^2 + 4xy + 2y^3 + e^{-3y} + \ln(2y) \]

Compute the partial derivative with respect to \(x\).

    \(\frac{\partial}{\partial x}f(x,y)=2x+4y+2y^3+e^{-3y}+ln(2y)\)

    Now compute the partial derivative of \(f(x,y)\) with respect to \(y\):

    \(\frac{\partial}{\partial y}f(x,y)=x^2+4x+6y^2-3e^{-3y}+\frac{1}{y}\)

    \paragraph{Question 4d}\label{question-4d}

Find the value(s) of \(x\) which minimize the expressions below. Justify
why it is the minimum.

    \paragraph{Part A}\label{part-a}

\(\sum_{i=1}^{10} (i - x)^2\)

    x = 5.5 Justify:
\[\frac{d}{dx}\sum_{i=1}^{10}(i-x)^2=\sum_{i=1}^{10}(2x-2i)=\sum_{i=1}^{10}(2x)-\sum_{i=1}^{10}(2i)=20x-110=0\],
when \(x=5.5\)
\[\frac{d^2}{dx^2}\sum_{i=1}^{10}(i-x)^2=\sum_{i=1}^{10}2=20>0\], so
\(x=5.5\)will minimize the expression.

    \paragraph{Part B}\label{part-b}

\(\sin\left(\frac{x}{2}\right)\cos\left(\frac{x}{2}\right)\)

    \(x=\frac{3}{2}\pi+2k\pi, k\in \mathbb{Z}\)
Justify:\[sin(\frac{x}{2})cos(\frac{x}{2})=\frac{1}{2}sin(x)\] of which
graphic show the values of x which minimize the origin expression.

    \paragraph{Part C}\label{part-c}

\$ \vert 5 - x \vert + \vert 3 - x \vert + \vert 1 - x \vert \$

    \begin{equation}
origin\ expression=\left\{
\begin{aligned}
3x-9,\quad & x>5 \\
x+1,\quad & 3<x\leq5 \\
-x+7,\quad & 1<x\leq3 \\
-3x+9,\quad & x\leq1
\end{aligned}
\right.
\end{equation}

The graphic show the value of x which minimize the expression is \(x=3\)

    \paragraph{Question 4e}\label{question-4e}

Let \$ \sigma(x) = \dfrac{1}{1+e^{-x}} \$

    \paragraph{Part A}\label{part-a}

Show that \(\sigma(-x) = 1 - \sigma(x)\) where
\(\sigma(x) = \frac{1}{1+e^{-x}}\).

    \(\sigma(-x)=\frac{1}{1+e^x}=\frac{1+e^x-e^x}{1+e^x}=1-\frac{e^x}{1+e^x}=1-\frac{1}{1+e^{-x}}=1-\sigma(x)\)

    \paragraph{Part B}\label{part-b}

Show that the derivative can be written as:

\[\frac{d}{dx}\sigma(x) = \sigma(x)(1 - \sigma(x)) \]

    \(\frac{d}{dx}\sigma(x)=\frac{e^{-x}}{(1+e^{-x})^2}=\frac{1}{1+e^{-x}}\frac{e^{-x}}{1+e^{-x}}=\sigma(x)*\sigma(-x)= \sigma(x)(1 - \sigma(x))\)

    \paragraph{Question 4f}\label{question-4f}

Write code to plot the function \(f(x) = x^2\) and the derivative of
\(f\) evaluated at \(x=8\) and \(x=0\).

Set the range of the x-axis to (-15, 15) and the range of the y axis to
(-100, 300) and the figure size to (8,8).

Your resulting plot should look like this:

You should use the \texttt{plt.plot} function to plot lines. You may
find the following functions useful:

\begin{itemize}
\tightlist
\item
  \href{https://matplotlib.org/api/_as_gen/matplotlib.pyplot.plot.html}{\texttt{plt.plot}}
\item
  \href{https://stackoverflow.com/questions/332289/how-do-you-change-the-size-of-figures-drawn-with-matplotlib}{\texttt{plt.figure(figsize=..)}}
\item
  \href{https://matplotlib.org/api/_as_gen/matplotlib.pyplot.ylim.html}{\texttt{plt.ylim(..)}}
\item
  \href{https://matplotlib.org/api/_as_gen/matplotlib.pyplot.hlines.html}{\texttt{plt.axhline(..)}}
\end{itemize}

    

    \begin{Verbatim}[commandchars=\\\{\}]
{\color{incolor}In [{\color{incolor}49}]:} \PY{k}{def} \PY{n+nf}{f}\PY{p}{(}\PY{n}{x}\PY{p}{)}\PY{p}{:}
             \PY{k}{return} \PY{n}{x}\PY{o}{*}\PY{o}{*}\PY{l+m+mi}{2}
             
         \PY{k}{def} \PY{n+nf}{df}\PY{p}{(}\PY{n}{x}\PY{p}{)}\PY{p}{:}
             \PY{k}{return} \PY{l+m+mi}{2}\PY{o}{*}\PY{n}{x}
         
         \PY{k}{def} \PY{n+nf}{plot}\PY{p}{(}\PY{n}{f}\PY{p}{,} \PY{n}{df}\PY{p}{)}\PY{p}{:}
             \PY{n}{plt}\PY{o}{.}\PY{n}{figure}\PY{p}{(}\PY{n}{figsize}\PY{o}{=}\PY{p}{(}\PY{l+m+mi}{8}\PY{p}{,}\PY{l+m+mi}{8}\PY{p}{)}\PY{p}{)}
             \PY{n}{plt}\PY{o}{.}\PY{n}{ylim}\PY{p}{(}\PY{p}{(}\PY{o}{\PYZhy{}}\PY{l+m+mi}{100}\PY{p}{,}\PY{l+m+mi}{300}\PY{p}{)}\PY{p}{)}
             \PY{n}{plt}\PY{o}{.}\PY{n}{xlim}\PY{p}{(}\PY{p}{(}\PY{o}{\PYZhy{}}\PY{l+m+mi}{15}\PY{p}{,}\PY{l+m+mi}{15}\PY{p}{)}\PY{p}{)}
             
             \PY{n}{x2}\PY{o}{=}\PY{n}{np}\PY{o}{.}\PY{n}{arange}\PY{p}{(}\PY{o}{\PYZhy{}}\PY{l+m+mi}{15}\PY{p}{,}\PY{l+m+mi}{15}\PY{p}{,}\PY{l+m+mf}{0.01}\PY{p}{)}
             \PY{n}{y2}\PY{o}{=}\PY{n}{f}\PY{p}{(}\PY{n}{x2}\PY{p}{)}
             \PY{n}{plt}\PY{o}{.}\PY{n}{plot}\PY{p}{(}\PY{n}{x2}\PY{p}{,}\PY{n}{y2}\PY{p}{,}\PY{l+s+s2}{\PYZdq{}}\PY{l+s+s2}{blue}\PY{l+s+s2}{\PYZdq{}}\PY{p}{,}\PY{n}{linewidth}\PY{o}{=}\PY{l+m+mi}{3}\PY{p}{)}
             
             \PY{n}{x1}\PY{o}{=}\PY{n}{np}\PY{o}{.}\PY{n}{arange}\PY{p}{(}\PY{o}{\PYZhy{}}\PY{l+m+mi}{15}\PY{p}{,}\PY{l+m+mi}{15}\PY{p}{,}\PY{l+m+mf}{0.01}\PY{p}{)}
             \PY{n}{y1}\PY{o}{=}\PY{n}{df}\PY{p}{(}\PY{n}{x1}\PY{p}{)}
             \PY{n}{plt}\PY{o}{.}\PY{n}{plot}\PY{p}{(}\PY{n}{x1}\PY{p}{,}\PY{n}{y1}\PY{p}{,}\PY{l+s+s2}{\PYZdq{}}\PY{l+s+s2}{yellow}\PY{l+s+s2}{\PYZdq{}}\PY{p}{,}\PY{n}{linewidth}\PY{o}{=}\PY{l+m+mi}{3}\PY{p}{)}
             
             \PY{n}{plt}\PY{o}{.}\PY{n}{axhline}\PY{p}{(}\PY{n+nb}{min}\PY{p}{(}\PY{n}{y2}\PY{p}{)}\PY{p}{,}\PY{n}{xmin}\PY{o}{=}\PY{l+m+mi}{0}\PY{p}{,}\PY{n}{xmax}\PY{o}{=}\PY{l+m+mi}{1}\PY{p}{,}\PY{n}{color}\PY{o}{=}\PY{l+s+s1}{\PYZsq{}}\PY{l+s+s1}{red}\PY{l+s+s1}{\PYZsq{}}\PY{p}{,}\PY{n}{linewidth}\PY{o}{=}\PY{l+m+mi}{3}\PY{p}{)}
             
             \PY{n}{plt}\PY{o}{.}\PY{n}{show}
             
         \PY{n}{plot}\PY{p}{(}\PY{n}{f}\PY{p}{,} \PY{n}{df}\PY{p}{)}
\end{Verbatim}


    \begin{center}
    \adjustimage{max size={0.9\linewidth}{0.9\paperheight}}{output_120_0.png}
    \end{center}
    { \hspace*{\fill} \\}
    
    \paragraph{Question 4g}\label{question-4g}

How can we interpret what the derivative of a function is using this
plot?

    We can see that when a function is decresing, the derivative of the
function is below 0. And when a function is increasing, the derivative
of function is above 0. Therefore, we can interpret the derivative is a
index which suggest the trend and rate of change of function.

    \paragraph{Question 4h}\label{question-4h}

Consider the following scenario:

Only \(1\%\) of 40-year-old women who participate in a routine
mammography test have breast cancer. \(80\%\) of women who have breast
cancer will test positive, but \(9.6\%\) of women who don't have breast
cancer will also get positive tests.

Suppose we know that a woman of this age tested positive in a routine
screening. What is the probability that she actually has breast cancer?

\textbf{Hint:} Try to use Bayes' rule.

    \(P(C|+)=\frac{P(+|C)*P(C)}{P(+|C)P(C)+P(+|C^c)P(C^c)}\approx 7.764\%\)

    \begin{center}\rule{0.5\linewidth}{\linethickness}\end{center}

\subsection{5: Basic Classification}\label{basic-classification}

Suppose we have a dataset containing two different classes, class A and
class B, visualized below as purple dots and yellow dots. Let's
implement basic K nearest neighbors classifier that classifies a point
of interest by finding the K nearest neighbors. Recall the K-NN
algorithm:

Given an example point to classify and the population we are querying
from,

\begin{enumerate}
\def\labelenumi{\arabic{enumi}.}
\item
  Find the distance between the example point and each point in the data
  set
\item
  Augment the data table with a column containing all the distances to
  the example point
\item
  Sort the augmented table in increasing order of the distances to the
  example point
\item
  Take the top k rows of the sorted table to get the K nearest neighbors
\end{enumerate}

    

    \begin{Verbatim}[commandchars=\\\{\}]
{\color{incolor}In [{\color{incolor}50}]:} \PY{c+c1}{\PYZsh{} Code to visualize the dataset}
         \PY{n}{class\PYZus{}A} \PY{o}{=} \PY{l+m+mf}{0.5} \PY{o}{*} \PY{n}{np}\PY{o}{.}\PY{n}{random}\PY{o}{.}\PY{n}{randn}\PY{p}{(}\PY{l+m+mi}{100}\PY{p}{,} \PY{l+m+mi}{2}\PY{p}{)}
         \PY{n}{class\PYZus{}B} \PY{o}{=} \PY{l+m+mf}{0.5} \PY{o}{*} \PY{n}{np}\PY{o}{.}\PY{n}{random}\PY{o}{.}\PY{n}{randn}\PY{p}{(}\PY{l+m+mi}{100}\PY{p}{,} \PY{l+m+mi}{2}\PY{p}{)} \PY{o}{+} \PY{p}{(}\PY{l+m+mi}{2}\PY{p}{,} \PY{l+m+mi}{2}\PY{p}{)}
         \PY{n}{class\PYZus{}A} \PY{o}{=} \PY{n}{np}\PY{o}{.}\PY{n}{append}\PY{p}{(}\PY{n}{class\PYZus{}A}\PY{p}{,} \PY{n}{np}\PY{o}{.}\PY{n}{zeros}\PY{p}{(}\PY{p}{(}\PY{l+m+mi}{100}\PY{p}{,} \PY{l+m+mi}{1}\PY{p}{)}\PY{p}{)}\PY{p}{,} \PY{n}{axis}\PY{o}{=}\PY{l+m+mi}{1}\PY{p}{)}
         \PY{n}{class\PYZus{}B} \PY{o}{=} \PY{n}{np}\PY{o}{.}\PY{n}{append}\PY{p}{(}\PY{n}{class\PYZus{}B}\PY{p}{,} \PY{n}{np}\PY{o}{.}\PY{n}{ones}\PY{p}{(}\PY{p}{(}\PY{l+m+mi}{100}\PY{p}{,}\PY{l+m+mi}{1}\PY{p}{)}\PY{p}{)}\PY{p}{,} \PY{n}{axis}\PY{o}{=}\PY{l+m+mi}{1}\PY{p}{)}
         \PY{n}{population} \PY{o}{=} \PY{n}{np}\PY{o}{.}\PY{n}{append}\PY{p}{(}\PY{n}{class\PYZus{}A}\PY{p}{,} \PY{n}{class\PYZus{}B}\PY{p}{,} \PY{n}{axis}\PY{o}{=}\PY{l+m+mi}{0}\PY{p}{)}
         \PY{n}{plt}\PY{o}{.}\PY{n}{figure}\PY{p}{(}\PY{n}{figsize}\PY{o}{=}\PY{p}{(}\PY{l+m+mi}{10}\PY{p}{,}\PY{l+m+mi}{10}\PY{p}{)}\PY{p}{)}
         \PY{n}{plt}\PY{o}{.}\PY{n}{axis}\PY{p}{(}\PY{l+s+s1}{\PYZsq{}}\PY{l+s+s1}{equal}\PY{l+s+s1}{\PYZsq{}}\PY{p}{)}
         \PY{n}{plt}\PY{o}{.}\PY{n}{scatter}\PY{p}{(}\PY{n}{population}\PY{p}{[}\PY{p}{:}\PY{p}{,}\PY{l+m+mi}{0}\PY{p}{]}\PY{p}{,} \PY{n}{population}\PY{p}{[}\PY{p}{:}\PY{p}{,}\PY{l+m+mi}{1}\PY{p}{]}\PY{p}{,} \PY{n}{c}\PY{o}{=}\PY{n}{population}\PY{p}{[}\PY{p}{:}\PY{p}{,}\PY{l+m+mi}{2}\PY{p}{]}\PY{p}{)}
         \PY{n}{plt}\PY{o}{.}\PY{n}{title}\PY{p}{(}\PY{l+s+s1}{\PYZsq{}}\PY{l+s+s1}{Two Class Dataset}\PY{l+s+s1}{\PYZsq{}}\PY{p}{)}
\end{Verbatim}


\begin{Verbatim}[commandchars=\\\{\}]
{\color{outcolor}Out[{\color{outcolor}50}]:} Text(0.5,1,'Two Class Dataset')
\end{Verbatim}
            
    \begin{center}
    \adjustimage{max size={0.9\linewidth}{0.9\paperheight}}{output_127_1.png}
    \end{center}
    { \hspace*{\fill} \\}
    
    \subsubsection{Question 5}\label{question-5}

Fill in the functions \texttt{distance} and \texttt{getNeighbors} below,
and test your code against the sklearn KNN classifier (write tests to
see that your \texttt{getNeighbors} finds the same neighbors as the
sklearn version). You must write at least 2 test cases with one example
for each class.

Sklearn KNN:
http://scikit-learn.org/stable/modules/generated/sklearn.neighbors.KNeighborsClassifier.html

    \begin{Verbatim}[commandchars=\\\{\}]
{\color{incolor}In [{\color{incolor}51}]:} \PY{k+kn}{import} \PY{n+nn}{operator}
         \PY{k+kn}{from} \PY{n+nn}{sklearn}\PY{n+nn}{.}\PY{n+nn}{neighbors} \PY{k}{import} \PY{n}{KNeighborsClassifier}
         \PY{k+kn}{from} \PY{n+nn}{typing} \PY{k}{import} \PY{n}{Union}
         
         \PY{k}{def} \PY{n+nf}{distance}\PY{p}{(}\PY{n}{p1}\PY{p}{:} \PY{n}{np}\PY{o}{.}\PY{n}{ndarray}\PY{p}{,} \PY{n}{p2}\PY{p}{:} \PY{n}{np}\PY{o}{.}\PY{n}{ndarray}\PY{p}{)} \PY{o}{\PYZhy{}}\PY{o}{\PYZgt{}} \PY{n}{Union}\PY{p}{[}\PY{n+nb}{float}\PY{p}{,} \PY{n+nb}{int}\PY{p}{]}\PY{p}{:}
             \PY{l+s+sd}{\PYZdq{}\PYZdq{}\PYZdq{}}
         \PY{l+s+sd}{    Return the euclidean distance between two points p1 and p2.}
         \PY{l+s+sd}{    }
         \PY{l+s+sd}{    Shape:}
         \PY{l+s+sd}{        \PYZhy{} p1 and p2 are expected to have shape (1,2)}
         \PY{l+s+sd}{    }
         \PY{l+s+sd}{    \PYZgt{}\PYZgt{}\PYZgt{} p1 = np.array([1.2, 2])}
         \PY{l+s+sd}{    \PYZgt{}\PYZgt{}\PYZgt{} p2 = np.array([2, 2])}
         \PY{l+s+sd}{    \PYZgt{}\PYZgt{}\PYZgt{} distance(p1, p2)}
         \PY{l+s+sd}{    0.8}
         \PY{l+s+sd}{    \PYZgt{}\PYZgt{}\PYZgt{} distance(p1, p1)}
         \PY{l+s+sd}{    0 }
         \PY{l+s+sd}{    \PYZdq{}\PYZdq{}\PYZdq{}}
             \PY{k}{return} \PY{n}{np}\PY{o}{.}\PY{n}{sqrt}\PY{p}{(}\PY{n+nb}{sum}\PY{p}{(}\PY{n+nb}{abs}\PY{p}{(}\PY{n}{p1}\PY{o}{\PYZhy{}}\PY{n}{p2}\PY{p}{)}\PY{o}{*}\PY{o}{*}\PY{l+m+mi}{2}\PY{p}{)}\PY{p}{)}
         
         \PY{k}{def} \PY{n+nf}{get\PYZus{}neighbors}\PY{p}{(}\PY{n}{example}\PY{p}{:} \PY{n}{np}\PY{o}{.}\PY{n}{ndarray}\PY{p}{,} \PY{n}{population}\PY{p}{:} \PY{n}{np}\PY{o}{.}\PY{n}{ndarray}\PY{p}{,} \PY{n}{k}\PY{p}{:} \PY{n+nb}{int}\PY{o}{=}\PY{l+m+mi}{5}\PY{p}{)}\PY{p}{:}
             \PY{l+s+sd}{\PYZdq{}\PYZdq{}\PYZdq{}}
         \PY{l+s+sd}{    For a given example, return the k nearest neighbors in the given population as a np.array.}
         \PY{l+s+sd}{    Each element of the population array should have the x\PYZhy{}coordinate, the y\PYZhy{}coordinate, and the class.}
         \PY{l+s+sd}{    }
         \PY{l+s+sd}{    Shape:}
         \PY{l+s+sd}{        \PYZhy{} example is a point, with shape (1,1,2)}
         \PY{l+s+sd}{        \PYZhy{} population is a 2d array with shape (N, 3). N is the total number of points.}
         \PY{l+s+sd}{          The first two column represent the x and y coordinates of the points. }
         \PY{l+s+sd}{          The third column represent the class of each point. }
         \PY{l+s+sd}{    }
         \PY{l+s+sd}{    \PYZgt{}\PYZgt{}\PYZgt{} example = np.array([[1, 1]])}
         \PY{l+s+sd}{    \PYZgt{}\PYZgt{}\PYZgt{} population = np.array([}
         \PY{l+s+sd}{            [0, 0, 0], \PYZsh{} point at coordinate (0, 0) belongs to class 0 }
         \PY{l+s+sd}{            [100, 100, 1] \PYZsh{} point at coordinate (100, 100) belongs to class 1}
         \PY{l+s+sd}{        ])}
         \PY{l+s+sd}{    \PYZgt{}\PYZgt{}\PYZgt{} get\PYZus{}neighbors(example, population, k=1)}
         \PY{l+s+sd}{    np.array([}
         \PY{l+s+sd}{        [0, 0, 0] \PYZsh{} point at coordinate (0, 0) is closet to example at (1, 1)}
         \PY{l+s+sd}{    ])}
         \PY{l+s+sd}{    \PYZdq{}\PYZdq{}\PYZdq{}}
             \PY{n}{dis} \PY{o}{=} \PY{n}{np}\PY{o}{.}\PY{n}{array}\PY{p}{(}\PY{p}{[}\PY{n}{distance}\PY{p}{(}\PY{n}{example}\PY{p}{[}\PY{l+m+mi}{0}\PY{p}{]}\PY{p}{,}\PY{n}{pop}\PY{p}{[}\PY{l+m+mi}{0}\PY{p}{:}\PY{l+m+mi}{2}\PY{p}{]}\PY{p}{)} \PY{k}{for} \PY{n}{pop} \PY{o+ow}{in} \PY{n}{population}\PY{p}{]}\PY{p}{)}
             \PY{n}{add} \PY{o}{=} \PY{n}{np}\PY{o}{.}\PY{n}{c\PYZus{}}\PY{p}{[}\PY{n}{population}\PY{p}{,}\PY{n}{dis}\PY{p}{]}
             \PY{k}{return} \PY{n}{add}\PY{p}{[}\PY{p}{(}\PY{n}{add}\PY{p}{[}\PY{p}{:}\PY{p}{,}\PY{l+m+mi}{3}\PY{p}{]}\PY{p}{)}\PY{o}{.}\PY{n}{argsort}\PY{p}{(}\PY{p}{)}\PY{p}{]}\PY{p}{[}\PY{l+m+mi}{0}\PY{p}{:}\PY{n}{k}\PY{p}{]}\PY{p}{[}\PY{p}{:}\PY{p}{,}\PY{l+m+mi}{0}\PY{p}{:}\PY{l+m+mi}{2}\PY{p}{]}
             
         
         \PY{k}{def} \PY{n+nf}{test}\PY{p}{(}\PY{n}{example}\PY{p}{,} \PY{n}{population}\PY{p}{,} \PY{n}{k}\PY{p}{)}\PY{p}{:}
             \PY{l+s+sd}{\PYZdq{}\PYZdq{}\PYZdq{}}
         \PY{l+s+sd}{    Compare your results against the sklearn KNN classifier.}
         \PY{l+s+sd}{    }
         \PY{l+s+sd}{    This function should create a sklearn KNeighborsClassifier and verify}
         \PY{l+s+sd}{    that get\PYZus{}neighbors method returns the same result as the KNeighborsClassifier.}
         \PY{l+s+sd}{    }
         \PY{l+s+sd}{    \PYZgt{}\PYZgt{}\PYZgt{} example = np.array([[1, 1]])}
         \PY{l+s+sd}{    \PYZgt{}\PYZgt{}\PYZgt{} population = np.array([}
         \PY{l+s+sd}{            [0, 0, 0], \PYZsh{} point at coordinate (0, 0) belongs to class 0 }
         \PY{l+s+sd}{            [100, 100, 1] \PYZsh{} point at coordinate (100, 100) belongs to class 1}
         \PY{l+s+sd}{        ])}
         \PY{l+s+sd}{    \PYZgt{}\PYZgt{}\PYZgt{} \PYZsh{} Provided get\PYZus{}neighbors implemented correctly}
         \PY{l+s+sd}{    \PYZgt{}\PYZgt{}\PYZgt{} test(example, population, k=1)}
         \PY{l+s+sd}{    True}
         \PY{l+s+sd}{    \PYZdq{}\PYZdq{}\PYZdq{}}
             \PY{n}{neigh} \PY{o}{=} \PY{n}{KNeighborsClassifier}\PY{p}{(}\PY{n}{n\PYZus{}neighbors}\PY{o}{=}\PY{n}{k}\PY{p}{)}
             \PY{n}{neigh}\PY{o}{.}\PY{n}{fit}\PY{p}{(}\PY{n}{population}\PY{p}{[}\PY{p}{:}\PY{p}{,}\PY{l+m+mi}{0}\PY{p}{:}\PY{l+m+mi}{2}\PY{p}{]}\PY{p}{,} \PY{n}{population}\PY{p}{[}\PY{p}{:}\PY{p}{,}\PY{l+m+mi}{2}\PY{p}{]}\PY{p}{)}
             \PY{n}{result} \PY{o}{=} \PY{n}{population}\PY{p}{[}\PY{n}{neigh}\PY{o}{.}\PY{n}{kneighbors}\PY{p}{(}\PY{n}{example}\PY{p}{,} \PY{n}{k}\PY{p}{,} \PY{n}{return\PYZus{}distance}\PY{o}{=}\PY{k+kc}{False}\PY{p}{)}\PY{p}{,}\PY{l+m+mi}{0}\PY{p}{:}\PY{l+m+mi}{2}\PY{p}{]}
             \PY{k}{return} \PY{n}{np}\PY{o}{.}\PY{n}{array\PYZus{}equal}\PY{p}{(}\PY{n}{get\PYZus{}neighbors}\PY{p}{(}\PY{n}{example}\PY{p}{,} \PY{n}{population}\PY{p}{,}\PY{n}{k}\PY{p}{)}\PY{p}{,}\PY{n}{result}\PY{p}{[}\PY{l+m+mi}{0}\PY{p}{]}\PY{p}{)}
         
         \PY{c+c1}{\PYZsh{} YOUR TESTS HERE}
         \PY{n}{test}\PY{p}{(}\PY{p}{[}\PY{p}{[}\PY{l+m+mi}{0}\PY{p}{,} \PY{l+m+mi}{0}\PY{p}{]}\PY{p}{]}\PY{p}{,} \PY{n}{population}\PY{p}{,} \PY{l+m+mi}{5}\PY{p}{)}
         \PY{n}{test}\PY{p}{(}\PY{p}{[}\PY{p}{[}\PY{l+m+mi}{2}\PY{p}{,} \PY{l+m+mi}{2}\PY{p}{]}\PY{p}{]}\PY{p}{,} \PY{n}{population}\PY{p}{,} \PY{l+m+mi}{5}\PY{p}{)}
\end{Verbatim}


\begin{Verbatim}[commandchars=\\\{\}]
{\color{outcolor}Out[{\color{outcolor}51}]:} True
\end{Verbatim}
            
    If your code is written properly, the two plots below should plot the
point of interest, example\_A and example\_B, (red) and the K nearest
neighbors (green). It should look SIMILAR to the two plots below.

 

    

    \begin{Verbatim}[commandchars=\\\{\}]
{\color{incolor}In [{\color{incolor}52}]:} \PY{n}{example\PYZus{}A} \PY{o}{=} \PY{p}{[}\PY{p}{[}\PY{l+m+mi}{0}\PY{p}{,} \PY{l+m+mi}{0}\PY{p}{]}\PY{p}{]}
         \PY{n}{example\PYZus{}B} \PY{o}{=} \PY{p}{[}\PY{p}{[}\PY{l+m+mi}{2}\PY{p}{,} \PY{l+m+mi}{2}\PY{p}{]}\PY{p}{]}
         
         \PY{n}{neighbors\PYZus{}A} \PY{o}{=} \PY{n}{get\PYZus{}neighbors}\PY{p}{(}\PY{n}{example\PYZus{}A}\PY{p}{,} \PY{n}{population}\PY{p}{,} \PY{n}{k}\PY{o}{=}\PY{l+m+mi}{10}\PY{p}{)}
         \PY{n}{plt}\PY{o}{.}\PY{n}{figure}\PY{p}{(}\PY{n}{figsize}\PY{o}{=}\PY{p}{(}\PY{l+m+mi}{10}\PY{p}{,}\PY{l+m+mi}{10}\PY{p}{)}\PY{p}{)}
         \PY{n}{plt}\PY{o}{.}\PY{n}{axis}\PY{p}{(}\PY{l+s+s1}{\PYZsq{}}\PY{l+s+s1}{equal}\PY{l+s+s1}{\PYZsq{}}\PY{p}{)}
         \PY{n}{plt}\PY{o}{.}\PY{n}{scatter}\PY{p}{(}\PY{n}{population}\PY{p}{[}\PY{p}{:}\PY{p}{,}\PY{l+m+mi}{0}\PY{p}{]}\PY{p}{,} \PY{n}{population}\PY{p}{[}\PY{p}{:}\PY{p}{,}\PY{l+m+mi}{1}\PY{p}{]}\PY{p}{,} \PY{n}{c}\PY{o}{=}\PY{n}{population}\PY{p}{[}\PY{p}{:}\PY{p}{,}\PY{l+m+mi}{2}\PY{p}{]}\PY{p}{)}
         \PY{n}{plt}\PY{o}{.}\PY{n}{scatter}\PY{p}{(}\PY{n}{neighbors\PYZus{}A}\PY{p}{[}\PY{p}{:}\PY{p}{,}\PY{l+m+mi}{0}\PY{p}{]}\PY{p}{,} \PY{n}{neighbors\PYZus{}A}\PY{p}{[}\PY{p}{:}\PY{p}{,}\PY{l+m+mi}{1}\PY{p}{]}\PY{p}{,} \PY{n}{c}\PY{o}{=}\PY{l+s+s1}{\PYZsq{}}\PY{l+s+s1}{g}\PY{l+s+s1}{\PYZsq{}}\PY{p}{)}
         \PY{n}{plt}\PY{o}{.}\PY{n}{scatter}\PY{p}{(}\PY{l+m+mi}{0}\PY{p}{,}\PY{l+m+mi}{0}\PY{p}{,}\PY{n}{c}\PY{o}{=}\PY{l+s+s1}{\PYZsq{}}\PY{l+s+s1}{r}\PY{l+s+s1}{\PYZsq{}}\PY{p}{)}
         \PY{n}{plt}\PY{o}{.}\PY{n}{title}\PY{p}{(}\PY{l+s+s1}{\PYZsq{}}\PY{l+s+s1}{Class A Example}\PY{l+s+s1}{\PYZsq{}}\PY{p}{)}
         \PY{n}{plt}\PY{o}{.}\PY{n}{show}\PY{p}{(}\PY{p}{)}
         
         \PY{n}{neighbors\PYZus{}B} \PY{o}{=} \PY{n}{get\PYZus{}neighbors}\PY{p}{(}\PY{n}{example\PYZus{}B}\PY{p}{,} \PY{n}{population}\PY{p}{,} \PY{n}{k}\PY{o}{=}\PY{l+m+mi}{10}\PY{p}{)}
         \PY{n}{plt}\PY{o}{.}\PY{n}{figure}\PY{p}{(}\PY{n}{figsize}\PY{o}{=}\PY{p}{(}\PY{l+m+mi}{10}\PY{p}{,}\PY{l+m+mi}{10}\PY{p}{)}\PY{p}{)}
         \PY{n}{plt}\PY{o}{.}\PY{n}{axis}\PY{p}{(}\PY{l+s+s1}{\PYZsq{}}\PY{l+s+s1}{equal}\PY{l+s+s1}{\PYZsq{}}\PY{p}{)}
         \PY{n}{plt}\PY{o}{.}\PY{n}{scatter}\PY{p}{(}\PY{n}{population}\PY{p}{[}\PY{p}{:}\PY{p}{,}\PY{l+m+mi}{0}\PY{p}{]}\PY{p}{,} \PY{n}{population}\PY{p}{[}\PY{p}{:}\PY{p}{,}\PY{l+m+mi}{1}\PY{p}{]}\PY{p}{,} \PY{n}{c}\PY{o}{=}\PY{n}{population}\PY{p}{[}\PY{p}{:}\PY{p}{,}\PY{l+m+mi}{2}\PY{p}{]}\PY{p}{)}
         \PY{n}{plt}\PY{o}{.}\PY{n}{scatter}\PY{p}{(}\PY{n}{neighbors\PYZus{}B}\PY{p}{[}\PY{p}{:}\PY{p}{,}\PY{l+m+mi}{0}\PY{p}{]}\PY{p}{,} \PY{n}{neighbors\PYZus{}B}\PY{p}{[}\PY{p}{:}\PY{p}{,}\PY{l+m+mi}{1}\PY{p}{]}\PY{p}{,} \PY{n}{c}\PY{o}{=}\PY{l+s+s1}{\PYZsq{}}\PY{l+s+s1}{g}\PY{l+s+s1}{\PYZsq{}}\PY{p}{)}
         \PY{n}{plt}\PY{o}{.}\PY{n}{scatter}\PY{p}{(}\PY{l+m+mi}{2}\PY{p}{,}\PY{l+m+mi}{2}\PY{p}{,}\PY{n}{c}\PY{o}{=}\PY{l+s+s1}{\PYZsq{}}\PY{l+s+s1}{r}\PY{l+s+s1}{\PYZsq{}}\PY{p}{)}
         \PY{n}{plt}\PY{o}{.}\PY{n}{title}\PY{p}{(}\PY{l+s+s1}{\PYZsq{}}\PY{l+s+s1}{Class B Example}\PY{l+s+s1}{\PYZsq{}}\PY{p}{)}
         \PY{n}{plt}\PY{o}{.}\PY{n}{show}\PY{p}{(}\PY{p}{)}
\end{Verbatim}


    \begin{center}
    \adjustimage{max size={0.9\linewidth}{0.9\paperheight}}{output_132_0.png}
    \end{center}
    { \hspace*{\fill} \\}
    
    \begin{center}
    \adjustimage{max size={0.9\linewidth}{0.9\paperheight}}{output_132_1.png}
    \end{center}
    { \hspace*{\fill} \\}
    
    \subsection{Submission}\label{submission}

You're done!

Before submitting this assignment, ensure to:

\begin{enumerate}
\def\labelenumi{\arabic{enumi}.}
\tightlist
\item
  Restart the Kernel (in the menubar, select
  Kernel-\textgreater{}Restart \& Run All)
\item
  Validate the notebook by clicking the "Validate" button
\end{enumerate}

Finally, make sure to \textbf{submit} the assignment via the Assignments
tab in Datahub


    % Add a bibliography block to the postdoc
    
    
    
    \end{document}
