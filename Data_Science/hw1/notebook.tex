
% Default to the notebook output style

    


% Inherit from the specified cell style.




    
\documentclass[11pt]{article}

    
    
    \usepackage[T1]{fontenc}
    % Nicer default font (+ math font) than Computer Modern for most use cases
    \usepackage{mathpazo}

    % Basic figure setup, for now with no caption control since it's done
    % automatically by Pandoc (which extracts ![](path) syntax from Markdown).
    \usepackage{graphicx}
    % We will generate all images so they have a width \maxwidth. This means
    % that they will get their normal width if they fit onto the page, but
    % are scaled down if they would overflow the margins.
    \makeatletter
    \def\maxwidth{\ifdim\Gin@nat@width>\linewidth\linewidth
    \else\Gin@nat@width\fi}
    \makeatother
    \let\Oldincludegraphics\includegraphics
    % Set max figure width to be 80% of text width, for now hardcoded.
    \renewcommand{\includegraphics}[1]{\Oldincludegraphics[width=.8\maxwidth]{#1}}
    % Ensure that by default, figures have no caption (until we provide a
    % proper Figure object with a Caption API and a way to capture that
    % in the conversion process - todo).
    \usepackage{caption}
    \DeclareCaptionLabelFormat{nolabel}{}
    \captionsetup{labelformat=nolabel}

    \usepackage{adjustbox} % Used to constrain images to a maximum size 
    \usepackage{xcolor} % Allow colors to be defined
    \usepackage{enumerate} % Needed for markdown enumerations to work
    \usepackage{geometry} % Used to adjust the document margins
    \usepackage{amsmath} % Equations
    \usepackage{amssymb} % Equations
    \usepackage{textcomp} % defines textquotesingle
    % Hack from http://tex.stackexchange.com/a/47451/13684:
    \AtBeginDocument{%
        \def\PYZsq{\textquotesingle}% Upright quotes in Pygmentized code
    }
    \usepackage{upquote} % Upright quotes for verbatim code
    \usepackage{eurosym} % defines \euro
    \usepackage[mathletters]{ucs} % Extended unicode (utf-8) support
    \usepackage[utf8x]{inputenc} % Allow utf-8 characters in the tex document
    \usepackage{fancyvrb} % verbatim replacement that allows latex
    \usepackage{grffile} % extends the file name processing of package graphics 
                         % to support a larger range 
    % The hyperref package gives us a pdf with properly built
    % internal navigation ('pdf bookmarks' for the table of contents,
    % internal cross-reference links, web links for URLs, etc.)
    \usepackage{hyperref}
    \usepackage{longtable} % longtable support required by pandoc >1.10
    \usepackage{booktabs}  % table support for pandoc > 1.12.2
    \usepackage[inline]{enumitem} % IRkernel/repr support (it uses the enumerate* environment)
    \usepackage[normalem]{ulem} % ulem is needed to support strikethroughs (\sout)
                                % normalem makes italics be italics, not underlines
    

    
    
    % Colors for the hyperref package
    \definecolor{urlcolor}{rgb}{0,.145,.698}
    \definecolor{linkcolor}{rgb}{.71,0.21,0.01}
    \definecolor{citecolor}{rgb}{.12,.54,.11}

    % ANSI colors
    \definecolor{ansi-black}{HTML}{3E424D}
    \definecolor{ansi-black-intense}{HTML}{282C36}
    \definecolor{ansi-red}{HTML}{E75C58}
    \definecolor{ansi-red-intense}{HTML}{B22B31}
    \definecolor{ansi-green}{HTML}{00A250}
    \definecolor{ansi-green-intense}{HTML}{007427}
    \definecolor{ansi-yellow}{HTML}{DDB62B}
    \definecolor{ansi-yellow-intense}{HTML}{B27D12}
    \definecolor{ansi-blue}{HTML}{208FFB}
    \definecolor{ansi-blue-intense}{HTML}{0065CA}
    \definecolor{ansi-magenta}{HTML}{D160C4}
    \definecolor{ansi-magenta-intense}{HTML}{A03196}
    \definecolor{ansi-cyan}{HTML}{60C6C8}
    \definecolor{ansi-cyan-intense}{HTML}{258F8F}
    \definecolor{ansi-white}{HTML}{C5C1B4}
    \definecolor{ansi-white-intense}{HTML}{A1A6B2}

    % commands and environments needed by pandoc snippets
    % extracted from the output of `pandoc -s`
    \providecommand{\tightlist}{%
      \setlength{\itemsep}{0pt}\setlength{\parskip}{0pt}}
    \DefineVerbatimEnvironment{Highlighting}{Verbatim}{commandchars=\\\{\}}
    % Add ',fontsize=\small' for more characters per line
    \newenvironment{Shaded}{}{}
    \newcommand{\KeywordTok}[1]{\textcolor[rgb]{0.00,0.44,0.13}{\textbf{{#1}}}}
    \newcommand{\DataTypeTok}[1]{\textcolor[rgb]{0.56,0.13,0.00}{{#1}}}
    \newcommand{\DecValTok}[1]{\textcolor[rgb]{0.25,0.63,0.44}{{#1}}}
    \newcommand{\BaseNTok}[1]{\textcolor[rgb]{0.25,0.63,0.44}{{#1}}}
    \newcommand{\FloatTok}[1]{\textcolor[rgb]{0.25,0.63,0.44}{{#1}}}
    \newcommand{\CharTok}[1]{\textcolor[rgb]{0.25,0.44,0.63}{{#1}}}
    \newcommand{\StringTok}[1]{\textcolor[rgb]{0.25,0.44,0.63}{{#1}}}
    \newcommand{\CommentTok}[1]{\textcolor[rgb]{0.38,0.63,0.69}{\textit{{#1}}}}
    \newcommand{\OtherTok}[1]{\textcolor[rgb]{0.00,0.44,0.13}{{#1}}}
    \newcommand{\AlertTok}[1]{\textcolor[rgb]{1.00,0.00,0.00}{\textbf{{#1}}}}
    \newcommand{\FunctionTok}[1]{\textcolor[rgb]{0.02,0.16,0.49}{{#1}}}
    \newcommand{\RegionMarkerTok}[1]{{#1}}
    \newcommand{\ErrorTok}[1]{\textcolor[rgb]{1.00,0.00,0.00}{\textbf{{#1}}}}
    \newcommand{\NormalTok}[1]{{#1}}
    
    % Additional commands for more recent versions of Pandoc
    \newcommand{\ConstantTok}[1]{\textcolor[rgb]{0.53,0.00,0.00}{{#1}}}
    \newcommand{\SpecialCharTok}[1]{\textcolor[rgb]{0.25,0.44,0.63}{{#1}}}
    \newcommand{\VerbatimStringTok}[1]{\textcolor[rgb]{0.25,0.44,0.63}{{#1}}}
    \newcommand{\SpecialStringTok}[1]{\textcolor[rgb]{0.73,0.40,0.53}{{#1}}}
    \newcommand{\ImportTok}[1]{{#1}}
    \newcommand{\DocumentationTok}[1]{\textcolor[rgb]{0.73,0.13,0.13}{\textit{{#1}}}}
    \newcommand{\AnnotationTok}[1]{\textcolor[rgb]{0.38,0.63,0.69}{\textbf{\textit{{#1}}}}}
    \newcommand{\CommentVarTok}[1]{\textcolor[rgb]{0.38,0.63,0.69}{\textbf{\textit{{#1}}}}}
    \newcommand{\VariableTok}[1]{\textcolor[rgb]{0.10,0.09,0.49}{{#1}}}
    \newcommand{\ControlFlowTok}[1]{\textcolor[rgb]{0.00,0.44,0.13}{\textbf{{#1}}}}
    \newcommand{\OperatorTok}[1]{\textcolor[rgb]{0.40,0.40,0.40}{{#1}}}
    \newcommand{\BuiltInTok}[1]{{#1}}
    \newcommand{\ExtensionTok}[1]{{#1}}
    \newcommand{\PreprocessorTok}[1]{\textcolor[rgb]{0.74,0.48,0.00}{{#1}}}
    \newcommand{\AttributeTok}[1]{\textcolor[rgb]{0.49,0.56,0.16}{{#1}}}
    \newcommand{\InformationTok}[1]{\textcolor[rgb]{0.38,0.63,0.69}{\textbf{\textit{{#1}}}}}
    \newcommand{\WarningTok}[1]{\textcolor[rgb]{0.38,0.63,0.69}{\textbf{\textit{{#1}}}}}
    
    
    % Define a nice break command that doesn't care if a line doesn't already
    % exist.
    \def\br{\hspace*{\fill} \\* }
    % Math Jax compatability definitions
    \def\gt{>}
    \def\lt{<}
    % Document parameters
    \title{hw1}
    
    
    

    % Pygments definitions
    
\makeatletter
\def\PY@reset{\let\PY@it=\relax \let\PY@bf=\relax%
    \let\PY@ul=\relax \let\PY@tc=\relax%
    \let\PY@bc=\relax \let\PY@ff=\relax}
\def\PY@tok#1{\csname PY@tok@#1\endcsname}
\def\PY@toks#1+{\ifx\relax#1\empty\else%
    \PY@tok{#1}\expandafter\PY@toks\fi}
\def\PY@do#1{\PY@bc{\PY@tc{\PY@ul{%
    \PY@it{\PY@bf{\PY@ff{#1}}}}}}}
\def\PY#1#2{\PY@reset\PY@toks#1+\relax+\PY@do{#2}}

\expandafter\def\csname PY@tok@w\endcsname{\def\PY@tc##1{\textcolor[rgb]{0.73,0.73,0.73}{##1}}}
\expandafter\def\csname PY@tok@c\endcsname{\let\PY@it=\textit\def\PY@tc##1{\textcolor[rgb]{0.25,0.50,0.50}{##1}}}
\expandafter\def\csname PY@tok@cp\endcsname{\def\PY@tc##1{\textcolor[rgb]{0.74,0.48,0.00}{##1}}}
\expandafter\def\csname PY@tok@k\endcsname{\let\PY@bf=\textbf\def\PY@tc##1{\textcolor[rgb]{0.00,0.50,0.00}{##1}}}
\expandafter\def\csname PY@tok@kp\endcsname{\def\PY@tc##1{\textcolor[rgb]{0.00,0.50,0.00}{##1}}}
\expandafter\def\csname PY@tok@kt\endcsname{\def\PY@tc##1{\textcolor[rgb]{0.69,0.00,0.25}{##1}}}
\expandafter\def\csname PY@tok@o\endcsname{\def\PY@tc##1{\textcolor[rgb]{0.40,0.40,0.40}{##1}}}
\expandafter\def\csname PY@tok@ow\endcsname{\let\PY@bf=\textbf\def\PY@tc##1{\textcolor[rgb]{0.67,0.13,1.00}{##1}}}
\expandafter\def\csname PY@tok@nb\endcsname{\def\PY@tc##1{\textcolor[rgb]{0.00,0.50,0.00}{##1}}}
\expandafter\def\csname PY@tok@nf\endcsname{\def\PY@tc##1{\textcolor[rgb]{0.00,0.00,1.00}{##1}}}
\expandafter\def\csname PY@tok@nc\endcsname{\let\PY@bf=\textbf\def\PY@tc##1{\textcolor[rgb]{0.00,0.00,1.00}{##1}}}
\expandafter\def\csname PY@tok@nn\endcsname{\let\PY@bf=\textbf\def\PY@tc##1{\textcolor[rgb]{0.00,0.00,1.00}{##1}}}
\expandafter\def\csname PY@tok@ne\endcsname{\let\PY@bf=\textbf\def\PY@tc##1{\textcolor[rgb]{0.82,0.25,0.23}{##1}}}
\expandafter\def\csname PY@tok@nv\endcsname{\def\PY@tc##1{\textcolor[rgb]{0.10,0.09,0.49}{##1}}}
\expandafter\def\csname PY@tok@no\endcsname{\def\PY@tc##1{\textcolor[rgb]{0.53,0.00,0.00}{##1}}}
\expandafter\def\csname PY@tok@nl\endcsname{\def\PY@tc##1{\textcolor[rgb]{0.63,0.63,0.00}{##1}}}
\expandafter\def\csname PY@tok@ni\endcsname{\let\PY@bf=\textbf\def\PY@tc##1{\textcolor[rgb]{0.60,0.60,0.60}{##1}}}
\expandafter\def\csname PY@tok@na\endcsname{\def\PY@tc##1{\textcolor[rgb]{0.49,0.56,0.16}{##1}}}
\expandafter\def\csname PY@tok@nt\endcsname{\let\PY@bf=\textbf\def\PY@tc##1{\textcolor[rgb]{0.00,0.50,0.00}{##1}}}
\expandafter\def\csname PY@tok@nd\endcsname{\def\PY@tc##1{\textcolor[rgb]{0.67,0.13,1.00}{##1}}}
\expandafter\def\csname PY@tok@s\endcsname{\def\PY@tc##1{\textcolor[rgb]{0.73,0.13,0.13}{##1}}}
\expandafter\def\csname PY@tok@sd\endcsname{\let\PY@it=\textit\def\PY@tc##1{\textcolor[rgb]{0.73,0.13,0.13}{##1}}}
\expandafter\def\csname PY@tok@si\endcsname{\let\PY@bf=\textbf\def\PY@tc##1{\textcolor[rgb]{0.73,0.40,0.53}{##1}}}
\expandafter\def\csname PY@tok@se\endcsname{\let\PY@bf=\textbf\def\PY@tc##1{\textcolor[rgb]{0.73,0.40,0.13}{##1}}}
\expandafter\def\csname PY@tok@sr\endcsname{\def\PY@tc##1{\textcolor[rgb]{0.73,0.40,0.53}{##1}}}
\expandafter\def\csname PY@tok@ss\endcsname{\def\PY@tc##1{\textcolor[rgb]{0.10,0.09,0.49}{##1}}}
\expandafter\def\csname PY@tok@sx\endcsname{\def\PY@tc##1{\textcolor[rgb]{0.00,0.50,0.00}{##1}}}
\expandafter\def\csname PY@tok@m\endcsname{\def\PY@tc##1{\textcolor[rgb]{0.40,0.40,0.40}{##1}}}
\expandafter\def\csname PY@tok@gh\endcsname{\let\PY@bf=\textbf\def\PY@tc##1{\textcolor[rgb]{0.00,0.00,0.50}{##1}}}
\expandafter\def\csname PY@tok@gu\endcsname{\let\PY@bf=\textbf\def\PY@tc##1{\textcolor[rgb]{0.50,0.00,0.50}{##1}}}
\expandafter\def\csname PY@tok@gd\endcsname{\def\PY@tc##1{\textcolor[rgb]{0.63,0.00,0.00}{##1}}}
\expandafter\def\csname PY@tok@gi\endcsname{\def\PY@tc##1{\textcolor[rgb]{0.00,0.63,0.00}{##1}}}
\expandafter\def\csname PY@tok@gr\endcsname{\def\PY@tc##1{\textcolor[rgb]{1.00,0.00,0.00}{##1}}}
\expandafter\def\csname PY@tok@ge\endcsname{\let\PY@it=\textit}
\expandafter\def\csname PY@tok@gs\endcsname{\let\PY@bf=\textbf}
\expandafter\def\csname PY@tok@gp\endcsname{\let\PY@bf=\textbf\def\PY@tc##1{\textcolor[rgb]{0.00,0.00,0.50}{##1}}}
\expandafter\def\csname PY@tok@go\endcsname{\def\PY@tc##1{\textcolor[rgb]{0.53,0.53,0.53}{##1}}}
\expandafter\def\csname PY@tok@gt\endcsname{\def\PY@tc##1{\textcolor[rgb]{0.00,0.27,0.87}{##1}}}
\expandafter\def\csname PY@tok@err\endcsname{\def\PY@bc##1{\setlength{\fboxsep}{0pt}\fcolorbox[rgb]{1.00,0.00,0.00}{1,1,1}{\strut ##1}}}
\expandafter\def\csname PY@tok@kc\endcsname{\let\PY@bf=\textbf\def\PY@tc##1{\textcolor[rgb]{0.00,0.50,0.00}{##1}}}
\expandafter\def\csname PY@tok@kd\endcsname{\let\PY@bf=\textbf\def\PY@tc##1{\textcolor[rgb]{0.00,0.50,0.00}{##1}}}
\expandafter\def\csname PY@tok@kn\endcsname{\let\PY@bf=\textbf\def\PY@tc##1{\textcolor[rgb]{0.00,0.50,0.00}{##1}}}
\expandafter\def\csname PY@tok@kr\endcsname{\let\PY@bf=\textbf\def\PY@tc##1{\textcolor[rgb]{0.00,0.50,0.00}{##1}}}
\expandafter\def\csname PY@tok@bp\endcsname{\def\PY@tc##1{\textcolor[rgb]{0.00,0.50,0.00}{##1}}}
\expandafter\def\csname PY@tok@fm\endcsname{\def\PY@tc##1{\textcolor[rgb]{0.00,0.00,1.00}{##1}}}
\expandafter\def\csname PY@tok@vc\endcsname{\def\PY@tc##1{\textcolor[rgb]{0.10,0.09,0.49}{##1}}}
\expandafter\def\csname PY@tok@vg\endcsname{\def\PY@tc##1{\textcolor[rgb]{0.10,0.09,0.49}{##1}}}
\expandafter\def\csname PY@tok@vi\endcsname{\def\PY@tc##1{\textcolor[rgb]{0.10,0.09,0.49}{##1}}}
\expandafter\def\csname PY@tok@vm\endcsname{\def\PY@tc##1{\textcolor[rgb]{0.10,0.09,0.49}{##1}}}
\expandafter\def\csname PY@tok@sa\endcsname{\def\PY@tc##1{\textcolor[rgb]{0.73,0.13,0.13}{##1}}}
\expandafter\def\csname PY@tok@sb\endcsname{\def\PY@tc##1{\textcolor[rgb]{0.73,0.13,0.13}{##1}}}
\expandafter\def\csname PY@tok@sc\endcsname{\def\PY@tc##1{\textcolor[rgb]{0.73,0.13,0.13}{##1}}}
\expandafter\def\csname PY@tok@dl\endcsname{\def\PY@tc##1{\textcolor[rgb]{0.73,0.13,0.13}{##1}}}
\expandafter\def\csname PY@tok@s2\endcsname{\def\PY@tc##1{\textcolor[rgb]{0.73,0.13,0.13}{##1}}}
\expandafter\def\csname PY@tok@sh\endcsname{\def\PY@tc##1{\textcolor[rgb]{0.73,0.13,0.13}{##1}}}
\expandafter\def\csname PY@tok@s1\endcsname{\def\PY@tc##1{\textcolor[rgb]{0.73,0.13,0.13}{##1}}}
\expandafter\def\csname PY@tok@mb\endcsname{\def\PY@tc##1{\textcolor[rgb]{0.40,0.40,0.40}{##1}}}
\expandafter\def\csname PY@tok@mf\endcsname{\def\PY@tc##1{\textcolor[rgb]{0.40,0.40,0.40}{##1}}}
\expandafter\def\csname PY@tok@mh\endcsname{\def\PY@tc##1{\textcolor[rgb]{0.40,0.40,0.40}{##1}}}
\expandafter\def\csname PY@tok@mi\endcsname{\def\PY@tc##1{\textcolor[rgb]{0.40,0.40,0.40}{##1}}}
\expandafter\def\csname PY@tok@il\endcsname{\def\PY@tc##1{\textcolor[rgb]{0.40,0.40,0.40}{##1}}}
\expandafter\def\csname PY@tok@mo\endcsname{\def\PY@tc##1{\textcolor[rgb]{0.40,0.40,0.40}{##1}}}
\expandafter\def\csname PY@tok@ch\endcsname{\let\PY@it=\textit\def\PY@tc##1{\textcolor[rgb]{0.25,0.50,0.50}{##1}}}
\expandafter\def\csname PY@tok@cm\endcsname{\let\PY@it=\textit\def\PY@tc##1{\textcolor[rgb]{0.25,0.50,0.50}{##1}}}
\expandafter\def\csname PY@tok@cpf\endcsname{\let\PY@it=\textit\def\PY@tc##1{\textcolor[rgb]{0.25,0.50,0.50}{##1}}}
\expandafter\def\csname PY@tok@c1\endcsname{\let\PY@it=\textit\def\PY@tc##1{\textcolor[rgb]{0.25,0.50,0.50}{##1}}}
\expandafter\def\csname PY@tok@cs\endcsname{\let\PY@it=\textit\def\PY@tc##1{\textcolor[rgb]{0.25,0.50,0.50}{##1}}}

\def\PYZbs{\char`\\}
\def\PYZus{\char`\_}
\def\PYZob{\char`\{}
\def\PYZcb{\char`\}}
\def\PYZca{\char`\^}
\def\PYZam{\char`\&}
\def\PYZlt{\char`\<}
\def\PYZgt{\char`\>}
\def\PYZsh{\char`\#}
\def\PYZpc{\char`\%}
\def\PYZdl{\char`\$}
\def\PYZhy{\char`\-}
\def\PYZsq{\char`\'}
\def\PYZdq{\char`\"}
\def\PYZti{\char`\~}
% for compatibility with earlier versions
\def\PYZat{@}
\def\PYZlb{[}
\def\PYZrb{]}
\makeatother


    % Exact colors from NB
    \definecolor{incolor}{rgb}{0.0, 0.0, 0.5}
    \definecolor{outcolor}{rgb}{0.545, 0.0, 0.0}



    
    % Prevent overflowing lines due to hard-to-break entities
    \sloppy 
    % Setup hyperref package
    \hypersetup{
      breaklinks=true,  % so long urls are correctly broken across lines
      colorlinks=true,
      urlcolor=urlcolor,
      linkcolor=linkcolor,
      citecolor=citecolor,
      }
    % Slightly bigger margins than the latex defaults
    
    \geometry{verbose,tmargin=1in,bmargin=1in,lmargin=1in,rmargin=1in}
    
    

    \begin{document}
    
    
    \maketitle
    
    

    
    Before you turn this problem in, make sure everything runs as expected.
First, \textbf{restart the kernel} (in the menubar, select
Kernel\(\rightarrow\)Restart) and then \textbf{run all cells} (in the
menubar, select Cell\(\rightarrow\)Run All).

Make sure you fill in any place that says \texttt{YOUR\ CODE\ HERE} or
"YOUR ANSWER HERE", as well as your name and collaborators below:

    \begin{Verbatim}[commandchars=\\\{\}]
{\color{incolor}In [{\color{incolor}1}]:} \PY{n}{NAME} \PY{o}{=} \PY{l+s+s2}{\PYZdq{}}\PY{l+s+s2}{Weijie Yuan}\PY{l+s+s2}{\PYZdq{}}
        \PY{n}{COLLABORATORS} \PY{o}{=} \PY{l+s+s2}{\PYZdq{}}\PY{l+s+s2}{N/A}\PY{l+s+s2}{\PYZdq{}}
\end{Verbatim}


    \section{Homework 1: Food Safety}\label{homework-1-food-safety}

\subsection{Cleaning and Exploring Data with
Pandas}\label{cleaning-and-exploring-data-with-pandas}

\subsection{Due Date: Thursday 9/13, 11:59
PM}\label{due-date-thursday-913-1159-pm}

\subsection{Course Policies}\label{course-policies}

Here are some important course policies. These are also located at
http://www.ds100.org/fa18/.

\textbf{Collaboration Policy}

Data science is a collaborative activity. While you may talk with others
about the homework, we ask that you \textbf{write your solutions
individually}. If you do discuss the assignments with others please
\textbf{include their names} at the top of your notebook.

\subsection{This assignment}\label{this-assignment}

In this homework, you will investigate restaurant food safety scores for
restaurants in San Francisco. Above is a sample score card for a
restaurant. The scores and violation information have been made
available by the San Francisco Department of Public Health and we have
made these data available to you via the DS 100 repository. The main
goal for this assignment is to understand how restaurants are scored. We
will walk through the various steps of exploratory data analysis to do
this. We will provide comments and insights along the way to give you a
sense of how we arrive at each discovery and what next steps it leads
to.

As we clean and explore these data, you will gain practice with: *
Reading simple csv files * Working with data at different levels of
granularity * Identifying the type of data collected, missing values,
anomalies, etc. * Exploring characteristics and distributions of
individual variables

\subsection{Score breakdown}\label{score-breakdown}

\begin{longtable}[]{@{}ll@{}}
\toprule
Question & Points\tabularnewline
\midrule
\endhead
1a & 1\tabularnewline
1b & 0\tabularnewline
1c & 0\tabularnewline
1d & 3\tabularnewline
1e & 1\tabularnewline
2a & 1\tabularnewline
2b & 2\tabularnewline
3a & 2\tabularnewline
3b & 0\tabularnewline
3c & 2\tabularnewline
3d & 1\tabularnewline
3e & 1\tabularnewline
4a & 2\tabularnewline
4b & 3\tabularnewline
5a & 1\tabularnewline
5b & 1\tabularnewline
5c & 1\tabularnewline
6a & 2\tabularnewline
6b & 3\tabularnewline
6c & 3\tabularnewline
7a & 2\tabularnewline
7b & 2\tabularnewline
7c & 6\tabularnewline
7d & 2\tabularnewline
7e & 3\tabularnewline
Total & 45\tabularnewline
\bottomrule
\end{longtable}

    To start the assignment, run the cell below to set up some imports and
the automatic tests that we will need for this assignment:

In many of these assignments (and your future adventures as a data
scientist) you will use \texttt{os}, \texttt{zipfile}, \texttt{pandas},
\texttt{numpy}, \texttt{matplotlib.pyplot}, and \texttt{seaborn}.

\begin{enumerate}
\def\labelenumi{\arabic{enumi}.}
\tightlist
\item
  Import each of these libraries \texttt{as} their commonly used
  abbreviations (e.g., \texttt{pd}, \texttt{np}, \texttt{plt}, and
  \texttt{sns}).\\
\item
  Don't forget to include \texttt{\%matplotlib\ inline} which enables
  \href{http://ipython.readthedocs.io/en/stable/interactive/magics.html\#magic-matplotlib}{inline
  matploblib plots}.
\item
  Add the line \texttt{sns.set()} to make your plots look nicer.
\end{enumerate}

    \begin{Verbatim}[commandchars=\\\{\}]
{\color{incolor}In [{\color{incolor}2}]:} \PY{k+kn}{import} \PY{n+nn}{os}
        \PY{k+kn}{import} \PY{n+nn}{zipfile}
        \PY{k+kn}{import} \PY{n+nn}{pandas} \PY{k}{as} \PY{n+nn}{pd}
        \PY{k+kn}{import} \PY{n+nn}{numpy} \PY{k}{as} \PY{n+nn}{np}
        \PY{k+kn}{import} \PY{n+nn}{matplotlib}\PY{n+nn}{.}\PY{n+nn}{pyplot} \PY{k}{as} \PY{n+nn}{plt}
        \PY{k+kn}{import} \PY{n+nn}{seaborn} \PY{k}{as} \PY{n+nn}{sns}
        \PY{o}{\PYZpc{}}\PY{k}{matplotlib} inline
        \PY{n}{sns}\PY{o}{.}\PY{n}{set}\PY{p}{(}\PY{p}{)}
\end{Verbatim}


    \begin{Verbatim}[commandchars=\\\{\}]
{\color{incolor}In [{\color{incolor}3}]:} \PY{k+kn}{import} \PY{n+nn}{sys}
        
        \PY{k}{assert} \PY{l+s+s1}{\PYZsq{}}\PY{l+s+s1}{zipfile}\PY{l+s+s1}{\PYZsq{}}\PY{o+ow}{in} \PY{n}{sys}\PY{o}{.}\PY{n}{modules}
        \PY{k}{assert} \PY{l+s+s1}{\PYZsq{}}\PY{l+s+s1}{pandas}\PY{l+s+s1}{\PYZsq{}}\PY{o+ow}{in} \PY{n}{sys}\PY{o}{.}\PY{n}{modules} \PY{o+ow}{and} \PY{n}{pd}
        \PY{k}{assert} \PY{l+s+s1}{\PYZsq{}}\PY{l+s+s1}{numpy}\PY{l+s+s1}{\PYZsq{}}\PY{o+ow}{in} \PY{n}{sys}\PY{o}{.}\PY{n}{modules} \PY{o+ow}{and} \PY{n}{np}
        \PY{k}{assert} \PY{l+s+s1}{\PYZsq{}}\PY{l+s+s1}{matplotlib}\PY{l+s+s1}{\PYZsq{}}\PY{o+ow}{in} \PY{n}{sys}\PY{o}{.}\PY{n}{modules} \PY{o+ow}{and} \PY{n}{plt}
        \PY{k}{assert} \PY{l+s+s1}{\PYZsq{}}\PY{l+s+s1}{seaborn}\PY{l+s+s1}{\PYZsq{}}\PY{o+ow}{in} \PY{n}{sys}\PY{o}{.}\PY{n}{modules} \PY{o+ow}{and} \PY{n}{sns}
\end{Verbatim}


    \subsection{Downloading the data}\label{downloading-the-data}

For this assignment, we need this data file:
http://www.ds100.org/fa18/assets/datasets/hw2-SFBusinesses.zip

We could write a few lines of code that are built to download this
specific data file, but it's a better idea to have a general function
that we can reuse for all of our assignments. Since this class isn't
really about the nuances of the Python file system libraries, we've
provided a function for you in ds100\_utils.py called
\texttt{fetch\_and\_cache} that can download files from the internet.

This function has the following arguments: - data\_url: the web address
to download - file: the file in which to save the results - data\_dir:
(default="data") the location to save the data - force: if true the file
is always re-downloaded

The way this function works is that it checks to see if
\texttt{data\_dir/file} already exists. If it does not exist already or
if \texttt{force=True}, the file at \texttt{data\_url} is downloaded and
placed at \texttt{data\_dir/file}. The process of storing a data file
for reuse later is called caching. If \texttt{data\_dir/file} already
and exists \texttt{force=False}, nothing is downloaded, and instead a
message is printed letting you know the date of the cached file.

The function returns a \texttt{pathlib.Path} object representing the
file. A \texttt{pathlib.Path} is an object that stores filepaths, e.g.
\texttt{\textasciitilde{}/Dropbox/ds100/horses.txt}.

    The code below uses \texttt{ds100\_utils.py} to download the data from
the following URL:
http://www.ds100.org/fa18/assets/datasets/hw2-SFBusinesses.zip

    \begin{Verbatim}[commandchars=\\\{\}]
{\color{incolor}In [{\color{incolor}4}]:} \PY{k+kn}{import} \PY{n+nn}{ds100\PYZus{}utils}
        \PY{n}{source\PYZus{}data\PYZus{}url} \PY{o}{=} \PY{l+s+s1}{\PYZsq{}}\PY{l+s+s1}{http://www.ds100.org/fa18/assets/datasets/hw2\PYZhy{}SFBusinesses.zip}\PY{l+s+s1}{\PYZsq{}}
        \PY{n}{target\PYZus{}file\PYZus{}name} \PY{o}{=} \PY{l+s+s1}{\PYZsq{}}\PY{l+s+s1}{data.zip}\PY{l+s+s1}{\PYZsq{}}
        \PY{n}{data\PYZus{}dir} \PY{o}{=} \PY{l+s+s1}{\PYZsq{}}\PY{l+s+s1}{.}\PY{l+s+s1}{\PYZsq{}}
        
        \PY{c+c1}{\PYZsh{} Change the force=False \PYZhy{}\PYZgt{} force=True in case you need to force redownload the data}
        \PY{n}{dest\PYZus{}path} \PY{o}{=} \PY{n}{ds100\PYZus{}utils}\PY{o}{.}\PY{n}{fetch\PYZus{}and\PYZus{}cache}\PY{p}{(}\PY{n}{data\PYZus{}url}\PY{o}{=}\PY{n}{source\PYZus{}data\PYZus{}url}\PY{p}{,} \PY{n}{data\PYZus{}dir}\PY{o}{=}\PY{n}{data\PYZus{}dir}\PY{p}{,} \PY{n}{file}\PY{o}{=}\PY{n}{target\PYZus{}file\PYZus{}name}\PY{p}{,} \PY{n}{force}\PY{o}{=}\PY{k+kc}{False}\PY{p}{)}
\end{Verbatim}


    \begin{Verbatim}[commandchars=\\\{\}]
Using cached version that was downloaded (UTC): Tue Sep  4 16:56:11 2018

    \end{Verbatim}

    After running the code, if you look at the directory containing
hw1.ipynb, you should see data.zip.

    \begin{center}\rule{0.5\linewidth}{\linethickness}\end{center}

\subsection{1: Loading Food Safety Data}\label{loading-food-safety-data}

Alright, great, now we have \texttt{data.zip}. We don't have any
specific questions yet, so let's focus on understanding the structure of
the data. Recall this involves answering questions such as

\begin{itemize}
\tightlist
\item
  Is the data in a standard format or encoding?
\item
  Is the data organized in records?
\item
  What are the fields in each record?
\end{itemize}

Let's start by looking at the contents of the zip file. We could in
theory do this by manually opening up the zip file on our computers or
using a shell command like \texttt{!unzip}, but on this homework we're
going to do almost everything in Python for maximum portability and
automation.

\textbf{Goal}: Fill in the code below so that \texttt{my\_zip} is a
\texttt{Zipfile.zipfile} object corresponding to the downloaded zip
file, and so that \texttt{list\_names} contains a list of the names of
all files inside the downloaded zip file.

Creating a \texttt{zipfile.Zipfile} object is a good start (the
\href{https://docs.python.org/3/library/zipfile.html}{Python docs} have
further details). You might also look back at the code from the case
study from lecture 2,
\href{http://www.ds100.org/fa18/assets/lectures/lec02/02-case-study.nbconvert.html}{02-case-study.ipynb}.
It's OK to copy and paste code from the 02-case-study file, though you
might get more out of this exercise if you type out an answer.

    \subsubsection{Question 1a: Looking Inside and Extracting the Zip
Files}\label{question-1a-looking-inside-and-extracting-the-zip-files}

    \begin{Verbatim}[commandchars=\\\{\}]
{\color{incolor}In [{\color{incolor}5}]:} \PY{c+c1}{\PYZsh{} Fill in the list\PYZus{}files variable with a list of all the names of the files in the zip file}
        \PY{n}{my\PYZus{}zip} \PY{o}{=} \PY{n}{zipfile}\PY{o}{.}\PY{n}{ZipFile}\PY{p}{(}\PY{n}{target\PYZus{}file\PYZus{}name}\PY{p}{,} \PY{n}{mode}\PY{o}{=}\PY{l+s+s1}{\PYZsq{}}\PY{l+s+s1}{r}\PY{l+s+s1}{\PYZsq{}}\PY{p}{)}
        \PY{n}{list\PYZus{}names} \PY{o}{=} \PY{n}{my\PYZus{}zip}\PY{o}{.}\PY{n}{namelist}\PY{p}{(}\PY{p}{)}
\end{Verbatim}


    The cell below will test that your code is correct.

    \begin{Verbatim}[commandchars=\\\{\}]
{\color{incolor}In [{\color{incolor}6}]:} \PY{k}{assert} \PY{n+nb}{isinstance}\PY{p}{(}\PY{n}{my\PYZus{}zip}\PY{p}{,} \PY{n}{zipfile}\PY{o}{.}\PY{n}{ZipFile}\PY{p}{)}
        \PY{k}{assert} \PY{n+nb}{isinstance}\PY{p}{(}\PY{n}{list\PYZus{}names}\PY{p}{,} \PY{n+nb}{list}\PY{p}{)}
        \PY{k}{assert} \PY{n+nb}{all}\PY{p}{(}\PY{p}{[}\PY{n+nb}{isinstance}\PY{p}{(}\PY{n}{file}\PY{p}{,} \PY{n+nb}{str}\PY{p}{)} \PY{k}{for} \PY{n}{file} \PY{o+ow}{in} \PY{n}{list\PYZus{}names}\PY{p}{]}\PY{p}{)} 
\end{Verbatim}


    In your answer above, if you see something like
\texttt{zipfile.Zipfile(\textquotesingle{}data.zip\textquotesingle{}...},
we suggest changing it to read \texttt{zipfile.Zipfile(dest\_path...} or
alternately \texttt{zipfile.Zipfile(target\_file\_name...}. In general,
we \textbf{strongly suggest having your filenames hard coded ONLY ONCE}
in any given iPython notebook. It is very dangerous to hard code things
twice, because if you change one but forget to change the other, you can
end up with very hard to find bugs.

    Now display the files' names and their sizes.

If you're not sure how to proceed, read about the attributes of a
\texttt{ZipFile} object in the Python docs linked above.

    \begin{Verbatim}[commandchars=\\\{\}]
{\color{incolor}In [{\color{incolor}7}]:} \PY{n}{my\PYZus{}zip}\PY{o}{.}\PY{n}{infolist}\PY{p}{(}\PY{p}{)}
\end{Verbatim}


\begin{Verbatim}[commandchars=\\\{\}]
{\color{outcolor}Out[{\color{outcolor}7}]:} [<ZipInfo filename='violations.csv' compress\_type=deflate external\_attr=0x20 file\_size=3726206 compress\_size=286253>,
         <ZipInfo filename='businesses.csv' compress\_type=deflate external\_attr=0x20 file\_size=660231 compress\_size=178549>,
         <ZipInfo filename='inspections.csv' compress\_type=deflate external\_attr=0x20 file\_size=466106 compress\_size=83198>,
         <ZipInfo filename='legend.csv' compress\_type=deflate external\_attr=0x20 file\_size=120 compress\_size=104>]
\end{Verbatim}
            
    Often when working with zipped data, we'll never unzip the actual
zipfile. This saves space on our local computer. However, for this HW,
the files are small, so we're just going to unzip everything. This has
the added benefit that you can look inside the csv files using a text
editor, which might be handy for more deeply understanding what's going
on. The cell below will unzip the csv files into a subdirectory called
"data". Try running the code below.

    \begin{Verbatim}[commandchars=\\\{\}]
{\color{incolor}In [{\color{incolor}8}]:} \PY{k+kn}{from} \PY{n+nn}{pathlib} \PY{k}{import} \PY{n}{Path}
        \PY{n}{data\PYZus{}dir} \PY{o}{=} \PY{n}{Path}\PY{p}{(}\PY{l+s+s1}{\PYZsq{}}\PY{l+s+s1}{data}\PY{l+s+s1}{\PYZsq{}}\PY{p}{)}
        \PY{n}{my\PYZus{}zip}\PY{o}{.}\PY{n}{extractall}\PY{p}{(}\PY{n}{data\PYZus{}dir}\PY{p}{)}
\end{Verbatim}


    When you ran the code above, nothing gets printed. However, this code
should have created a folder called "data", and in it should be the four
CSV files. Assuming you're using Datahub, use your web browser to verify
that these files were created, and try to open up \texttt{legend.csv} to
see what's inside. You should see something that looks like:

\begin{verbatim}
"Minimum_Score","Maximum_Score","Description"
0,70,"Poor"
71,85,"Needs Improvement"
86,90,"Adequate"
91,100,"Good"
\end{verbatim}

    \subsubsection{Question 1b: Programatically Looking Inside the
Files}\label{question-1b-programatically-looking-inside-the-files}

    What we see when we opened the file above is good news! It looks like
this file is indeed a csv file. Let's check the other three files. This
time, rather than opening up the files manually, let's use Python to
print out the first 5 lines of each. The \texttt{ds100\_utils} library
has a method called \texttt{head} that will allow you to retrieve the
first N lines of a file as a list. For example
\texttt{ds100\_utils.head(\textquotesingle{}data/legend.csv\textquotesingle{},\ 5)}
will return the first 5 lines of "data/legend.csv". Try using this
function to print out the first 5 lines of all four files that we just
extracted from the zipfile.

    \begin{Verbatim}[commandchars=\\\{\}]
{\color{incolor}In [{\color{incolor}9}]:} \PY{k}{for} \PY{n}{file} \PY{o+ow}{in} \PY{n}{my\PYZus{}zip}\PY{o}{.}\PY{n}{namelist}\PY{p}{(}\PY{p}{)}\PY{p}{:}
            \PY{n+nb}{print}\PY{p}{(}\PY{n}{ds100\PYZus{}utils}\PY{o}{.}\PY{n}{head}\PY{p}{(}\PY{n}{data\PYZus{}dir}\PY{o}{/}\PY{n}{file}\PY{p}{,} \PY{l+m+mi}{5}\PY{p}{)}\PY{p}{)}
\end{Verbatim}


    \begin{Verbatim}[commandchars=\\\{\}]
['"business\_id","date","description"\textbackslash{}n', '19,"20171211","Inadequate food safety knowledge or lack of certified food safety manager"\textbackslash{}n', '19,"20171211","Unapproved or unmaintained equipment or utensils"\textbackslash{}n', '19,"20160513","Unapproved or unmaintained equipment or utensils  [ date violation corrected: 12/11/2017 ]"\textbackslash{}n', '19,"20160513","Unclean or degraded floors walls or ceilings  [ date violation corrected: 12/11/2017 ]"\textbackslash{}n']
['"business\_id","name","address","city","state","postal\_code","latitude","longitude","phone\_number"\textbackslash{}n', '19,"NRGIZE LIFESTYLE CAFE","1200 VAN NESS AVE, 3RD FLOOR","San Francisco","CA","94109","37.786848","-122.421547","+14157763262"\textbackslash{}n', '24,"OMNI S.F. HOTEL - 2ND FLOOR PANTRY","500 CALIFORNIA ST, 2ND  FLOOR","San Francisco","CA","94104","37.792888","-122.403135","+14156779494"\textbackslash{}n', '31,"NORMAN\textbackslash{}'S ICE CREAM AND FREEZES","2801 LEAVENWORTH ST ","San Francisco","CA","94133","37.807155","-122.419004",""\textbackslash{}n', '45,"CHARLIE\textbackslash{}'S DELI CAFE","3202 FOLSOM ST ","San Francisco","CA","94110","37.747114","-122.413641","+14156415051"\textbackslash{}n']
['"business\_id","score","date","type"\textbackslash{}n', '19,"94","20160513","routine"\textbackslash{}n', '19,"94","20171211","routine"\textbackslash{}n', '24,"98","20171101","routine"\textbackslash{}n', '24,"98","20161005","routine"\textbackslash{}n']
['"Minimum\_Score","Maximum\_Score","Description"\textbackslash{}n', '0,70,"Poor"\textbackslash{}n', '71,85,"Needs Improvement"\textbackslash{}n', '86,90,"Adequate"\textbackslash{}n', '91,100,"Good"\textbackslash{}n']

    \end{Verbatim}

    \subsubsection{Question 1c: Reading in the
Files}\label{question-1c-reading-in-the-files}

Based on the above information, let's attempt to load
\texttt{businesses.csv}, \texttt{inspections.csv}, and
\texttt{violations.csv} into pandas data frames with the following
names: \texttt{bus}, \texttt{ins}, and \texttt{vio} respectively.

\emph{Note:} Because of character encoding issues one of the files
(\texttt{bus}) will require an additional argument
\texttt{encoding=\textquotesingle{}ISO-8859-1\textquotesingle{}} when
calling \texttt{pd.read\_csv}.

    \begin{Verbatim}[commandchars=\\\{\}]
{\color{incolor}In [{\color{incolor}10}]:} \PY{c+c1}{\PYZsh{} path to directory containing data}
         \PY{n}{dsDir} \PY{o}{=} \PY{n}{Path}\PY{p}{(}\PY{l+s+s1}{\PYZsq{}}\PY{l+s+s1}{data}\PY{l+s+s1}{\PYZsq{}}\PY{p}{)}
         
         \PY{c+c1}{\PYZsh{} Make sure to use these names}
         \PY{n}{bus} \PY{o}{=} \PY{n}{pd}\PY{o}{.}\PY{n}{read\PYZus{}csv}\PY{p}{(}\PY{n}{dsDir}\PY{o}{/}\PY{l+s+s1}{\PYZsq{}}\PY{l+s+s1}{businesses.csv}\PY{l+s+s1}{\PYZsq{}}\PY{p}{,}\PY{n}{encoding}\PY{o}{=}\PY{l+s+s1}{\PYZsq{}}\PY{l+s+s1}{ISO\PYZhy{}8859\PYZhy{}1}\PY{l+s+s1}{\PYZsq{}}\PY{p}{)}
         \PY{n}{ins} \PY{o}{=} \PY{n}{pd}\PY{o}{.}\PY{n}{read\PYZus{}csv}\PY{p}{(}\PY{n}{dsDir}\PY{o}{/}\PY{l+s+s1}{\PYZsq{}}\PY{l+s+s1}{inspections.csv}\PY{l+s+s1}{\PYZsq{}}\PY{p}{)}
         \PY{n}{vio} \PY{o}{=} \PY{n}{pd}\PY{o}{.}\PY{n}{read\PYZus{}csv}\PY{p}{(}\PY{n}{dsDir}\PY{o}{/}\PY{l+s+s1}{\PYZsq{}}\PY{l+s+s1}{violations.csv}\PY{l+s+s1}{\PYZsq{}}\PY{p}{)}
\end{Verbatim}


    Now that you've read in the files, let's try some \texttt{pd.DataFrame}
methods. Use the \texttt{DataFrame.head} command to show the top few
lines of the \texttt{bus}, \texttt{ins}, and \texttt{vio} dataframes.

    \begin{Verbatim}[commandchars=\\\{\}]
{\color{incolor}In [{\color{incolor}11}]:} \PY{n}{bus}\PY{o}{.}\PY{n}{head}\PY{p}{(}\PY{l+m+mi}{3}\PY{p}{)}
\end{Verbatim}


\begin{Verbatim}[commandchars=\\\{\}]
{\color{outcolor}Out[{\color{outcolor}11}]:}    business\_id                                name  \textbackslash{}
         0           19               NRGIZE LIFESTYLE CAFE   
         1           24  OMNI S.F. HOTEL - 2ND FLOOR PANTRY   
         2           31      NORMAN'S ICE CREAM AND FREEZES   
         
                                  address           city state postal\_code   latitude  \textbackslash{}
         0   1200 VAN NESS AVE, 3RD FLOOR  San Francisco    CA       94109  37.786848   
         1  500 CALIFORNIA ST, 2ND  FLOOR  San Francisco    CA       94104  37.792888   
         2           2801 LEAVENWORTH ST   San Francisco    CA       94133  37.807155   
         
             longitude  phone\_number  
         0 -122.421547  +14157763262  
         1 -122.403135  +14156779494  
         2 -122.419004           NaN  
\end{Verbatim}
            
    \begin{Verbatim}[commandchars=\\\{\}]
{\color{incolor}In [{\color{incolor}12}]:} \PY{n}{ins}\PY{o}{.}\PY{n}{head}\PY{p}{(}\PY{l+m+mi}{3}\PY{p}{)}
\end{Verbatim}


\begin{Verbatim}[commandchars=\\\{\}]
{\color{outcolor}Out[{\color{outcolor}12}]:}    business\_id  score      date     type
         0           19     94  20160513  routine
         1           19     94  20171211  routine
         2           24     98  20171101  routine
\end{Verbatim}
            
    \begin{Verbatim}[commandchars=\\\{\}]
{\color{incolor}In [{\color{incolor}13}]:} \PY{n}{vio}\PY{o}{.}\PY{n}{head}\PY{p}{(}\PY{l+m+mi}{3}\PY{p}{)}
\end{Verbatim}


\begin{Verbatim}[commandchars=\\\{\}]
{\color{outcolor}Out[{\color{outcolor}13}]:}    business\_id      date                                        description
         0           19  20171211  Inadequate food safety knowledge or lack of ce{\ldots}
         1           19  20171211   Unapproved or unmaintained equipment or utensils
         2           19  20160513  Unapproved or unmaintained equipment or utensi{\ldots}
\end{Verbatim}
            
    The \texttt{DataFrame.describe} method can also be handy for computing
summaries of various statistics of our dataframes. Try it out with each
of our 3 dataframes.

    \begin{Verbatim}[commandchars=\\\{\}]
{\color{incolor}In [{\color{incolor}14}]:} \PY{n}{bus}\PY{o}{.}\PY{n}{describe}\PY{p}{(}\PY{p}{)}
\end{Verbatim}


\begin{Verbatim}[commandchars=\\\{\}]
{\color{outcolor}Out[{\color{outcolor}14}]:}         business\_id     latitude    longitude
         count   6406.000000  3270.000000  3270.000000
         mean   53058.248049    37.773662  -122.425791
         std    34928.238762     0.022910     0.027762
         min       19.000000    37.668824  -122.510896
         25\%     7405.500000    37.760487  -122.436844
         50\%    68294.500000    37.780435  -122.418855
         75\%    83446.500000    37.789951  -122.406609
         max    94574.000000    37.824494  -122.368257
\end{Verbatim}
            
    \begin{Verbatim}[commandchars=\\\{\}]
{\color{incolor}In [{\color{incolor}15}]:} \PY{n}{ins}\PY{o}{.}\PY{n}{loc}\PY{p}{[}\PY{p}{:}\PY{p}{,}\PY{p}{[}\PY{l+s+s1}{\PYZsq{}}\PY{l+s+s1}{business\PYZus{}id}\PY{l+s+s1}{\PYZsq{}}\PY{p}{,}\PY{l+s+s1}{\PYZsq{}}\PY{l+s+s1}{score}\PY{l+s+s1}{\PYZsq{}}\PY{p}{]}\PY{p}{]}\PY{o}{.}\PY{n}{describe}\PY{p}{(}\PY{p}{)}
\end{Verbatim}


\begin{Verbatim}[commandchars=\\\{\}]
{\color{outcolor}Out[{\color{outcolor}15}]:}         business\_id         score
         count  14222.000000  14222.000000
         mean   45138.752637     90.697370
         std    34497.913056      8.088705
         min       19.000000     48.000000
         25\%     5634.000000     86.000000
         50\%    61462.000000     92.000000
         75\%    78074.000000     96.000000
         max    94231.000000    100.000000
\end{Verbatim}
            
    \begin{Verbatim}[commandchars=\\\{\}]
{\color{incolor}In [{\color{incolor}16}]:} \PY{n}{vio}\PY{o}{.}\PY{n}{loc}\PY{p}{[}\PY{p}{:}\PY{p}{,}\PY{p}{[}\PY{l+s+s1}{\PYZsq{}}\PY{l+s+s1}{business\PYZus{}id}\PY{l+s+s1}{\PYZsq{}}\PY{p}{]}\PY{p}{]}\PY{o}{.}\PY{n}{describe}\PY{p}{(}\PY{p}{)}
\end{Verbatim}


\begin{Verbatim}[commandchars=\\\{\}]
{\color{outcolor}Out[{\color{outcolor}16}]:}         business\_id
         count  39042.000000
         mean   45674.440244
         std    34172.433276
         min       19.000000
         25\%     4959.000000
         50\%    62060.000000
         75\%    77681.000000
         max    94231.000000
\end{Verbatim}
            
    \subsubsection{Question 1d: Verify Your Files were Read
Correctly}\label{question-1d-verify-your-files-were-read-correctly}

    Now, we perform some sanity checks for you to verify that you loaded the
data with the right structure. Run the following cells to load some
basic utilities (you do not need to change these at all):

    First, we check the basic structure of the data frames you created:

    \begin{Verbatim}[commandchars=\\\{\}]
{\color{incolor}In [{\color{incolor}17}]:} \PY{k}{assert} \PY{n+nb}{all}\PY{p}{(}\PY{n}{bus}\PY{o}{.}\PY{n}{columns} \PY{o}{==} \PY{p}{[}\PY{l+s+s1}{\PYZsq{}}\PY{l+s+s1}{business\PYZus{}id}\PY{l+s+s1}{\PYZsq{}}\PY{p}{,} \PY{l+s+s1}{\PYZsq{}}\PY{l+s+s1}{name}\PY{l+s+s1}{\PYZsq{}}\PY{p}{,} \PY{l+s+s1}{\PYZsq{}}\PY{l+s+s1}{address}\PY{l+s+s1}{\PYZsq{}}\PY{p}{,} \PY{l+s+s1}{\PYZsq{}}\PY{l+s+s1}{city}\PY{l+s+s1}{\PYZsq{}}\PY{p}{,} \PY{l+s+s1}{\PYZsq{}}\PY{l+s+s1}{state}\PY{l+s+s1}{\PYZsq{}}\PY{p}{,} \PY{l+s+s1}{\PYZsq{}}\PY{l+s+s1}{postal\PYZus{}code}\PY{l+s+s1}{\PYZsq{}}\PY{p}{,}
                                    \PY{l+s+s1}{\PYZsq{}}\PY{l+s+s1}{latitude}\PY{l+s+s1}{\PYZsq{}}\PY{p}{,} \PY{l+s+s1}{\PYZsq{}}\PY{l+s+s1}{longitude}\PY{l+s+s1}{\PYZsq{}}\PY{p}{,} \PY{l+s+s1}{\PYZsq{}}\PY{l+s+s1}{phone\PYZus{}number}\PY{l+s+s1}{\PYZsq{}}\PY{p}{]}\PY{p}{)}
         \PY{k}{assert} \PY{l+m+mi}{6400} \PY{o}{\PYZlt{}}\PY{o}{=} \PY{n+nb}{len}\PY{p}{(}\PY{n}{bus}\PY{p}{)} \PY{o}{\PYZlt{}}\PY{o}{=} \PY{l+m+mi}{6420}
         
         \PY{k}{assert} \PY{n+nb}{all}\PY{p}{(}\PY{n}{ins}\PY{o}{.}\PY{n}{columns} \PY{o}{==} \PY{p}{[}\PY{l+s+s1}{\PYZsq{}}\PY{l+s+s1}{business\PYZus{}id}\PY{l+s+s1}{\PYZsq{}}\PY{p}{,} \PY{l+s+s1}{\PYZsq{}}\PY{l+s+s1}{score}\PY{l+s+s1}{\PYZsq{}}\PY{p}{,} \PY{l+s+s1}{\PYZsq{}}\PY{l+s+s1}{date}\PY{l+s+s1}{\PYZsq{}}\PY{p}{,} \PY{l+s+s1}{\PYZsq{}}\PY{l+s+s1}{type}\PY{l+s+s1}{\PYZsq{}}\PY{p}{]}\PY{p}{)}
         \PY{k}{assert} \PY{l+m+mi}{14210} \PY{o}{\PYZlt{}}\PY{o}{=} \PY{n+nb}{len}\PY{p}{(}\PY{n}{ins}\PY{p}{)} \PY{o}{\PYZlt{}}\PY{o}{=} \PY{l+m+mi}{14250}
         
         \PY{k}{assert} \PY{n+nb}{all}\PY{p}{(}\PY{n}{vio}\PY{o}{.}\PY{n}{columns} \PY{o}{==} \PY{p}{[}\PY{l+s+s1}{\PYZsq{}}\PY{l+s+s1}{business\PYZus{}id}\PY{l+s+s1}{\PYZsq{}}\PY{p}{,} \PY{l+s+s1}{\PYZsq{}}\PY{l+s+s1}{date}\PY{l+s+s1}{\PYZsq{}}\PY{p}{,} \PY{l+s+s1}{\PYZsq{}}\PY{l+s+s1}{description}\PY{l+s+s1}{\PYZsq{}}\PY{p}{]}\PY{p}{)}
         \PY{k}{assert} \PY{l+m+mi}{39020} \PY{o}{\PYZlt{}}\PY{o}{=} \PY{n+nb}{len}\PY{p}{(}\PY{n}{vio}\PY{p}{)} \PY{o}{\PYZlt{}}\PY{o}{=} \PY{l+m+mi}{39080}
\end{Verbatim}


    Next we'll check that the statistics match what we expect. The following
are hard-coded statistical summaries of the correct data. .

    \begin{Verbatim}[commandchars=\\\{\}]
{\color{incolor}In [{\color{incolor}18}]:} \PY{n}{bus\PYZus{}summary} \PY{o}{=} \PY{n}{pd}\PY{o}{.}\PY{n}{DataFrame}\PY{p}{(}\PY{o}{*}\PY{o}{*}\PY{p}{\PYZob{}}\PY{l+s+s1}{\PYZsq{}}\PY{l+s+s1}{columns}\PY{l+s+s1}{\PYZsq{}}\PY{p}{:} \PY{p}{[}\PY{l+s+s1}{\PYZsq{}}\PY{l+s+s1}{business\PYZus{}id}\PY{l+s+s1}{\PYZsq{}}\PY{p}{,} \PY{l+s+s1}{\PYZsq{}}\PY{l+s+s1}{latitude}\PY{l+s+s1}{\PYZsq{}}\PY{p}{,} \PY{l+s+s1}{\PYZsq{}}\PY{l+s+s1}{longitude}\PY{l+s+s1}{\PYZsq{}}\PY{p}{]}\PY{p}{,}
          \PY{l+s+s1}{\PYZsq{}}\PY{l+s+s1}{data}\PY{l+s+s1}{\PYZsq{}}\PY{p}{:} \PY{p}{\PYZob{}}\PY{l+s+s1}{\PYZsq{}}\PY{l+s+s1}{business\PYZus{}id}\PY{l+s+s1}{\PYZsq{}}\PY{p}{:} \PY{p}{\PYZob{}}\PY{l+s+s1}{\PYZsq{}}\PY{l+s+s1}{50}\PY{l+s+s1}{\PYZpc{}}\PY{l+s+s1}{\PYZsq{}}\PY{p}{:} \PY{l+m+mf}{68294.5}\PY{p}{,} \PY{l+s+s1}{\PYZsq{}}\PY{l+s+s1}{max}\PY{l+s+s1}{\PYZsq{}}\PY{p}{:} \PY{l+m+mf}{94574.0}\PY{p}{,} \PY{l+s+s1}{\PYZsq{}}\PY{l+s+s1}{min}\PY{l+s+s1}{\PYZsq{}}\PY{p}{:} \PY{l+m+mf}{19.0}\PY{p}{\PYZcb{}}\PY{p}{,}
           \PY{l+s+s1}{\PYZsq{}}\PY{l+s+s1}{latitude}\PY{l+s+s1}{\PYZsq{}}\PY{p}{:} \PY{p}{\PYZob{}}\PY{l+s+s1}{\PYZsq{}}\PY{l+s+s1}{50}\PY{l+s+s1}{\PYZpc{}}\PY{l+s+s1}{\PYZsq{}}\PY{p}{:} \PY{l+m+mf}{37.780435}\PY{p}{,} \PY{l+s+s1}{\PYZsq{}}\PY{l+s+s1}{max}\PY{l+s+s1}{\PYZsq{}}\PY{p}{:} \PY{l+m+mf}{37.824494}\PY{p}{,} \PY{l+s+s1}{\PYZsq{}}\PY{l+s+s1}{min}\PY{l+s+s1}{\PYZsq{}}\PY{p}{:} \PY{l+m+mf}{37.668824}\PY{p}{\PYZcb{}}\PY{p}{,}
           \PY{l+s+s1}{\PYZsq{}}\PY{l+s+s1}{longitude}\PY{l+s+s1}{\PYZsq{}}\PY{p}{:} \PY{p}{\PYZob{}}\PY{l+s+s1}{\PYZsq{}}\PY{l+s+s1}{50}\PY{l+s+s1}{\PYZpc{}}\PY{l+s+s1}{\PYZsq{}}\PY{p}{:} \PY{o}{\PYZhy{}}\PY{l+m+mf}{122.41885450000001}\PY{p}{,}
            \PY{l+s+s1}{\PYZsq{}}\PY{l+s+s1}{max}\PY{l+s+s1}{\PYZsq{}}\PY{p}{:} \PY{o}{\PYZhy{}}\PY{l+m+mf}{122.368257}\PY{p}{,}
            \PY{l+s+s1}{\PYZsq{}}\PY{l+s+s1}{min}\PY{l+s+s1}{\PYZsq{}}\PY{p}{:} \PY{o}{\PYZhy{}}\PY{l+m+mf}{122.510896}\PY{p}{\PYZcb{}}\PY{p}{\PYZcb{}}\PY{p}{,}
          \PY{l+s+s1}{\PYZsq{}}\PY{l+s+s1}{index}\PY{l+s+s1}{\PYZsq{}}\PY{p}{:} \PY{p}{[}\PY{l+s+s1}{\PYZsq{}}\PY{l+s+s1}{min}\PY{l+s+s1}{\PYZsq{}}\PY{p}{,} \PY{l+s+s1}{\PYZsq{}}\PY{l+s+s1}{50}\PY{l+s+s1}{\PYZpc{}}\PY{l+s+s1}{\PYZsq{}}\PY{p}{,} \PY{l+s+s1}{\PYZsq{}}\PY{l+s+s1}{max}\PY{l+s+s1}{\PYZsq{}}\PY{p}{]}\PY{p}{\PYZcb{}}\PY{p}{)}
         
         \PY{n}{ins\PYZus{}summary} \PY{o}{=} \PY{n}{pd}\PY{o}{.}\PY{n}{DataFrame}\PY{p}{(}\PY{o}{*}\PY{o}{*}\PY{p}{\PYZob{}}\PY{l+s+s1}{\PYZsq{}}\PY{l+s+s1}{columns}\PY{l+s+s1}{\PYZsq{}}\PY{p}{:} \PY{p}{[}\PY{l+s+s1}{\PYZsq{}}\PY{l+s+s1}{business\PYZus{}id}\PY{l+s+s1}{\PYZsq{}}\PY{p}{,} \PY{l+s+s1}{\PYZsq{}}\PY{l+s+s1}{score}\PY{l+s+s1}{\PYZsq{}}\PY{p}{]}\PY{p}{,}
          \PY{l+s+s1}{\PYZsq{}}\PY{l+s+s1}{data}\PY{l+s+s1}{\PYZsq{}}\PY{p}{:} \PY{p}{\PYZob{}}\PY{l+s+s1}{\PYZsq{}}\PY{l+s+s1}{business\PYZus{}id}\PY{l+s+s1}{\PYZsq{}}\PY{p}{:} \PY{p}{\PYZob{}}\PY{l+s+s1}{\PYZsq{}}\PY{l+s+s1}{50}\PY{l+s+s1}{\PYZpc{}}\PY{l+s+s1}{\PYZsq{}}\PY{p}{:} \PY{l+m+mf}{61462.0}\PY{p}{,} \PY{l+s+s1}{\PYZsq{}}\PY{l+s+s1}{max}\PY{l+s+s1}{\PYZsq{}}\PY{p}{:} \PY{l+m+mf}{94231.0}\PY{p}{,} \PY{l+s+s1}{\PYZsq{}}\PY{l+s+s1}{min}\PY{l+s+s1}{\PYZsq{}}\PY{p}{:} \PY{l+m+mf}{19.0}\PY{p}{\PYZcb{}}\PY{p}{,}
           \PY{l+s+s1}{\PYZsq{}}\PY{l+s+s1}{score}\PY{l+s+s1}{\PYZsq{}}\PY{p}{:} \PY{p}{\PYZob{}}\PY{l+s+s1}{\PYZsq{}}\PY{l+s+s1}{50}\PY{l+s+s1}{\PYZpc{}}\PY{l+s+s1}{\PYZsq{}}\PY{p}{:} \PY{l+m+mf}{92.0}\PY{p}{,} \PY{l+s+s1}{\PYZsq{}}\PY{l+s+s1}{max}\PY{l+s+s1}{\PYZsq{}}\PY{p}{:} \PY{l+m+mf}{100.0}\PY{p}{,} \PY{l+s+s1}{\PYZsq{}}\PY{l+s+s1}{min}\PY{l+s+s1}{\PYZsq{}}\PY{p}{:} \PY{l+m+mf}{48.0}\PY{p}{\PYZcb{}}\PY{p}{\PYZcb{}}\PY{p}{,}
          \PY{l+s+s1}{\PYZsq{}}\PY{l+s+s1}{index}\PY{l+s+s1}{\PYZsq{}}\PY{p}{:} \PY{p}{[}\PY{l+s+s1}{\PYZsq{}}\PY{l+s+s1}{min}\PY{l+s+s1}{\PYZsq{}}\PY{p}{,} \PY{l+s+s1}{\PYZsq{}}\PY{l+s+s1}{50}\PY{l+s+s1}{\PYZpc{}}\PY{l+s+s1}{\PYZsq{}}\PY{p}{,} \PY{l+s+s1}{\PYZsq{}}\PY{l+s+s1}{max}\PY{l+s+s1}{\PYZsq{}}\PY{p}{]}\PY{p}{\PYZcb{}}\PY{p}{)}
         
         \PY{n}{vio\PYZus{}summary} \PY{o}{=} \PY{n}{pd}\PY{o}{.}\PY{n}{DataFrame}\PY{p}{(}\PY{o}{*}\PY{o}{*}\PY{p}{\PYZob{}}\PY{l+s+s1}{\PYZsq{}}\PY{l+s+s1}{columns}\PY{l+s+s1}{\PYZsq{}}\PY{p}{:} \PY{p}{[}\PY{l+s+s1}{\PYZsq{}}\PY{l+s+s1}{business\PYZus{}id}\PY{l+s+s1}{\PYZsq{}}\PY{p}{]}\PY{p}{,}
          \PY{l+s+s1}{\PYZsq{}}\PY{l+s+s1}{data}\PY{l+s+s1}{\PYZsq{}}\PY{p}{:} \PY{p}{\PYZob{}}\PY{l+s+s1}{\PYZsq{}}\PY{l+s+s1}{business\PYZus{}id}\PY{l+s+s1}{\PYZsq{}}\PY{p}{:} \PY{p}{\PYZob{}}\PY{l+s+s1}{\PYZsq{}}\PY{l+s+s1}{50}\PY{l+s+s1}{\PYZpc{}}\PY{l+s+s1}{\PYZsq{}}\PY{p}{:} \PY{l+m+mf}{62060.0}\PY{p}{,} \PY{l+s+s1}{\PYZsq{}}\PY{l+s+s1}{max}\PY{l+s+s1}{\PYZsq{}}\PY{p}{:} \PY{l+m+mf}{94231.0}\PY{p}{,} \PY{l+s+s1}{\PYZsq{}}\PY{l+s+s1}{min}\PY{l+s+s1}{\PYZsq{}}\PY{p}{:} \PY{l+m+mf}{19.0}\PY{p}{\PYZcb{}}\PY{p}{\PYZcb{}}\PY{p}{,}
          \PY{l+s+s1}{\PYZsq{}}\PY{l+s+s1}{index}\PY{l+s+s1}{\PYZsq{}}\PY{p}{:} \PY{p}{[}\PY{l+s+s1}{\PYZsq{}}\PY{l+s+s1}{min}\PY{l+s+s1}{\PYZsq{}}\PY{p}{,} \PY{l+s+s1}{\PYZsq{}}\PY{l+s+s1}{50}\PY{l+s+s1}{\PYZpc{}}\PY{l+s+s1}{\PYZsq{}}\PY{p}{,} \PY{l+s+s1}{\PYZsq{}}\PY{l+s+s1}{max}\PY{l+s+s1}{\PYZsq{}}\PY{p}{]}\PY{p}{\PYZcb{}}\PY{p}{)}
         
         \PY{k+kn}{from} \PY{n+nn}{IPython}\PY{n+nn}{.}\PY{n+nn}{display} \PY{k}{import} \PY{n}{display}
         
         \PY{n+nb}{print}\PY{p}{(}\PY{l+s+s1}{\PYZsq{}}\PY{l+s+s1}{What we expect from your Businesses dataframe:}\PY{l+s+s1}{\PYZsq{}}\PY{p}{)}
         \PY{n}{display}\PY{p}{(}\PY{n}{bus\PYZus{}summary}\PY{p}{)}
         \PY{n+nb}{print}\PY{p}{(}\PY{l+s+s1}{\PYZsq{}}\PY{l+s+s1}{What we expect from your Inspections dataframe:}\PY{l+s+s1}{\PYZsq{}}\PY{p}{)}
         \PY{n}{display}\PY{p}{(}\PY{n}{ins\PYZus{}summary}\PY{p}{)}
         \PY{n+nb}{print}\PY{p}{(}\PY{l+s+s1}{\PYZsq{}}\PY{l+s+s1}{What we expect from your Violations dataframe:}\PY{l+s+s1}{\PYZsq{}}\PY{p}{)}
         \PY{n}{display}\PY{p}{(}\PY{n}{vio\PYZus{}summary}\PY{p}{)}
\end{Verbatim}


    \begin{Verbatim}[commandchars=\\\{\}]
What we expect from your Businesses dataframe:

    \end{Verbatim}

    
    \begin{verbatim}
     business_id   latitude   longitude
min         19.0  37.668824 -122.510896
50%      68294.5  37.780435 -122.418855
max      94574.0  37.824494 -122.368257
    \end{verbatim}

    
    \begin{Verbatim}[commandchars=\\\{\}]
What we expect from your Inspections dataframe:

    \end{Verbatim}

    
    \begin{verbatim}
     business_id  score
min         19.0   48.0
50%      61462.0   92.0
max      94231.0  100.0
    \end{verbatim}

    
    \begin{Verbatim}[commandchars=\\\{\}]
What we expect from your Violations dataframe:

    \end{Verbatim}

    
    \begin{verbatim}
     business_id
min         19.0
50%      62060.0
max      94231.0
    \end{verbatim}

    
    The code below defines a testing function that we'll use to verify that
your data has the same statistics as what we expect. Run these cells to
define the function. The \texttt{df\_allclose} function has this name
because we are verifying that all of the statistics for your dataframe
are close to the expected values. Why not \texttt{df\_allequal}? It's a
bad idea in almost all cases to compare two floating point values like
37.780435, as rounding error can cause spurious failures.

    Do not delete the empty cell below!

    \begin{Verbatim}[commandchars=\\\{\}]
{\color{incolor}In [{\color{incolor}19}]:} \PY{l+s+sd}{\PYZdq{}\PYZdq{}\PYZdq{}Run this cell to load this utility comparison function that we will use in various}
         \PY{l+s+sd}{tests below (both tests you can see and those we run internally for grading).}
         
         \PY{l+s+sd}{Do not modify the function in any way.}
         \PY{l+s+sd}{\PYZdq{}\PYZdq{}\PYZdq{}}
         
         \PY{k}{def} \PY{n+nf}{df\PYZus{}allclose}\PY{p}{(}\PY{n}{actual}\PY{p}{,} \PY{n}{desired}\PY{p}{,} \PY{n}{columns}\PY{o}{=}\PY{k+kc}{None}\PY{p}{,} \PY{n}{rtol}\PY{o}{=}\PY{l+m+mf}{5e\PYZhy{}2}\PY{p}{)}\PY{p}{:}
             \PY{l+s+sd}{\PYZdq{}\PYZdq{}\PYZdq{}Compare selected columns of two dataframes on a few summary statistics.}
         \PY{l+s+sd}{    }
         \PY{l+s+sd}{    Compute the min, median and max of the two dataframes on the given columns, and compare}
         \PY{l+s+sd}{    that they match numerically to the given relative tolerance.}
         \PY{l+s+sd}{    }
         \PY{l+s+sd}{    If they don\PYZsq{}t match, an AssertionError is raised (by `numpy.testing`).}
         \PY{l+s+sd}{    \PYZdq{}\PYZdq{}\PYZdq{}}
             \PY{k+kn}{import} \PY{n+nn}{numpy}\PY{n+nn}{.}\PY{n+nn}{testing} \PY{k}{as} \PY{n+nn}{npt}
             
             \PY{c+c1}{\PYZsh{} summary statistics to compare on}
             \PY{n}{stats} \PY{o}{=} \PY{p}{[}\PY{l+s+s1}{\PYZsq{}}\PY{l+s+s1}{min}\PY{l+s+s1}{\PYZsq{}}\PY{p}{,} \PY{l+s+s1}{\PYZsq{}}\PY{l+s+s1}{50}\PY{l+s+s1}{\PYZpc{}}\PY{l+s+s1}{\PYZsq{}}\PY{p}{,} \PY{l+s+s1}{\PYZsq{}}\PY{l+s+s1}{max}\PY{l+s+s1}{\PYZsq{}}\PY{p}{]}
             
             \PY{c+c1}{\PYZsh{} For the desired values, we can provide a full DF with the same structure as}
             \PY{c+c1}{\PYZsh{} the actual data, or pre\PYZhy{}computed summary statistics.}
             \PY{c+c1}{\PYZsh{} We assume a pre\PYZhy{}computed summary was provided if columns is None. In that case, }
             \PY{c+c1}{\PYZsh{} `desired` *must* have the same structure as the actual\PYZsq{}s summary}
             \PY{k}{if} \PY{n}{columns} \PY{o+ow}{is} \PY{k+kc}{None}\PY{p}{:}
                 \PY{n}{des} \PY{o}{=} \PY{n}{desired}
                 \PY{n}{columns} \PY{o}{=} \PY{n}{desired}\PY{o}{.}\PY{n}{columns}
             \PY{k}{else}\PY{p}{:}
                 \PY{n}{des} \PY{o}{=} \PY{n}{desired}\PY{p}{[}\PY{n}{columns}\PY{p}{]}\PY{o}{.}\PY{n}{describe}\PY{p}{(}\PY{p}{)}\PY{o}{.}\PY{n}{loc}\PY{p}{[}\PY{n}{stats}\PY{p}{]}
         
             \PY{c+c1}{\PYZsh{} Extract summary stats from actual DF}
             \PY{n}{act} \PY{o}{=} \PY{n}{actual}\PY{p}{[}\PY{n}{columns}\PY{p}{]}\PY{o}{.}\PY{n}{describe}\PY{p}{(}\PY{p}{)}\PY{o}{.}\PY{n}{loc}\PY{p}{[}\PY{n}{stats}\PY{p}{]}
         
             \PY{n}{npt}\PY{o}{.}\PY{n}{assert\PYZus{}allclose}\PY{p}{(}\PY{n}{act}\PY{p}{,} \PY{n}{des}\PY{p}{,} \PY{n}{rtol}\PY{p}{)}
\end{Verbatim}


    Now let's run the automated tests. If your dataframes are correct, then
the following cell will seem to do nothing, which is a good thing!

    \begin{Verbatim}[commandchars=\\\{\}]
{\color{incolor}In [{\color{incolor}20}]:} \PY{c+c1}{\PYZsh{} These tests will raise an exception if your variables don\PYZsq{}t match numerically the correct}
         \PY{c+c1}{\PYZsh{} answers in the main summary statistics shown above.}
         \PY{n}{df\PYZus{}allclose}\PY{p}{(}\PY{n}{bus}\PY{p}{,} \PY{n}{bus\PYZus{}summary}\PY{p}{)}
         \PY{n}{df\PYZus{}allclose}\PY{p}{(}\PY{n}{ins}\PY{p}{,} \PY{n}{ins\PYZus{}summary}\PY{p}{)}
         \PY{n}{df\PYZus{}allclose}\PY{p}{(}\PY{n}{vio}\PY{p}{,} \PY{n}{vio\PYZus{}summary}\PY{p}{)}
\end{Verbatim}


    Do not edit the empty cell below. These are hidden tests!

    \subsubsection{Question 1e: Identifying Issues with the
Data}\label{question-1e-identifying-issues-with-the-data}

    Use the \texttt{head} command on your three files again. This time,
describe at least one potential problem with the data you see. Consider
issues with missing values and bad data.

    \begin{Verbatim}[commandchars=\\\{\}]
{\color{incolor}In [{\color{incolor}21}]:} \PY{n}{q1e\PYZus{}answer} \PY{o}{=} \PY{l+s+sa}{r}\PY{l+s+s2}{\PYZdq{}\PYZdq{}\PYZdq{}}
         
         \PY{l+s+s2}{The variable }\PY{l+s+s2}{\PYZdq{}}\PY{l+s+s2}{phone\PYZus{}number}\PY{l+s+s2}{\PYZdq{}}\PY{l+s+s2}{ in }\PY{l+s+s2}{\PYZdq{}}\PY{l+s+s2}{bus}\PY{l+s+s2}{\PYZdq{}}\PY{l+s+s2}{ dataframe has a missing value in first five rows. And business\PYZus{}id should }
         \PY{l+s+s2}{be regarded as qualitative variable instead of quantitative variable, so it is improper for id to have numeric}
         \PY{l+s+s2}{features.}
         
         \PY{l+s+s2}{\PYZdq{}\PYZdq{}\PYZdq{}}
         \PY{n+nb}{print}\PY{p}{(}\PY{n}{bus}\PY{o}{.}\PY{n}{head}\PY{p}{(}\PY{l+m+mi}{5}\PY{p}{)}\PY{p}{)}
         \PY{n+nb}{print}\PY{p}{(}\PY{n}{ins}\PY{o}{.}\PY{n}{head}\PY{p}{(}\PY{l+m+mi}{5}\PY{p}{)}\PY{p}{)}
         \PY{n+nb}{print}\PY{p}{(}\PY{n}{vio}\PY{o}{.}\PY{n}{head}\PY{p}{(}\PY{l+m+mi}{5}\PY{p}{)}\PY{p}{)}
         
         \PY{n+nb}{print}\PY{p}{(}\PY{n}{q1e\PYZus{}answer}\PY{p}{)}
\end{Verbatim}


    \begin{Verbatim}[commandchars=\\\{\}]
   business\_id                                name  \textbackslash{}
0           19               NRGIZE LIFESTYLE CAFE   
1           24  OMNI S.F. HOTEL - 2ND FLOOR PANTRY   
2           31      NORMAN'S ICE CREAM AND FREEZES   
3           45                 CHARLIE'S DELI CAFE   
4           48                          ART'S CAFE   

                         address           city state postal\_code   latitude  \textbackslash{}
0   1200 VAN NESS AVE, 3RD FLOOR  San Francisco    CA       94109  37.786848   
1  500 CALIFORNIA ST, 2ND  FLOOR  San Francisco    CA       94104  37.792888   
2           2801 LEAVENWORTH ST   San Francisco    CA       94133  37.807155   
3                3202 FOLSOM ST   San Francisco    CA       94110  37.747114   
4                 747 IRVING ST   San Francisco    CA       94122  37.764013   

    longitude  phone\_number  
0 -122.421547  +14157763262  
1 -122.403135  +14156779494  
2 -122.419004           NaN  
3 -122.413641  +14156415051  
4 -122.465749  +14156657440  
   business\_id  score      date     type
0           19     94  20160513  routine
1           19     94  20171211  routine
2           24     98  20171101  routine
3           24     98  20161005  routine
4           24     96  20160311  routine
   business\_id      date                                        description
0           19  20171211  Inadequate food safety knowledge or lack of ce{\ldots}
1           19  20171211   Unapproved or unmaintained equipment or utensils
2           19  20160513  Unapproved or unmaintained equipment or utensi{\ldots}
3           19  20160513  Unclean or degraded floors walls or ceilings  {\ldots}
4           19  20160513  Food safety certificate or food handler card n{\ldots}


The variable "phone\_number" in "bus" dataframe has a missing value in first five rows. And business\_id should 
be regarded as qualitative variable instead of quantitative variable, so it is improper for id to have numeric
features.



    \end{Verbatim}

    We will explore each file in turn, including determining its granularity
and primary keys and exploring many of the variables indivdually. Let's
begin with the businesses file, which has been read into the
\texttt{bus} dataframe.

    \begin{center}\rule{0.5\linewidth}{\linethickness}\end{center}

\subsection{2: Examining the Business
data}\label{examining-the-business-data}

From its name alone, we expect the \texttt{businesses.csv} file to
contain information about the restaurants. Let's investigate the
granularity of this dataset.

\textbf{Important note: From now on, the local autograder tests will not
be comprehensive. You can pass the automated tests in your notebook but
still fail tests in the autograder.} Please be sure to check your
results carefully.

    \subsubsection{Question 2a}\label{question-2a}

Examining the entries in \texttt{bus}, is the \texttt{business\_id}
unique for each record? Your code should compute the answer, i.e. don't
just hard code "True".

Hint: use \texttt{value\_counts()} or \texttt{unique()} to determine if
the \texttt{business\_id} series has any duplicates.

    \begin{Verbatim}[commandchars=\\\{\}]
{\color{incolor}In [{\color{incolor}22}]:} \PY{n}{is\PYZus{}business\PYZus{}id\PYZus{}unique} \PY{o}{=} \PY{n+nb}{all}\PY{p}{(}\PY{p}{[}\PY{n+nb}{id} \PY{k}{for} \PY{n+nb}{id} \PY{o+ow}{in} \PY{n}{bus}\PY{p}{[}\PY{l+s+s1}{\PYZsq{}}\PY{l+s+s1}{business\PYZus{}id}\PY{l+s+s1}{\PYZsq{}}\PY{p}{]}\PY{o}{.}\PY{n}{value\PYZus{}counts}\PY{p}{(}\PY{p}{)}\PY{o}{==}\PY{l+m+mi}{1}\PY{p}{]}\PY{p}{)}
\end{Verbatim}


    \begin{Verbatim}[commandchars=\\\{\}]
{\color{incolor}In [{\color{incolor}23}]:} \PY{k}{assert} \PY{n}{is\PYZus{}business\PYZus{}id\PYZus{}unique}
\end{Verbatim}


    \subsubsection{Question 2b}\label{question-2b}

With this information, you can address the question of granularity.
Answer the questions below.

\begin{enumerate}
\def\labelenumi{\arabic{enumi}.}
\tightlist
\item
  How many records are there?
\item
  What does each record represent (e.g., a store, a chain, a
  transaction)?\\
\item
  What is the primary key?
\end{enumerate}

Please write your answer in the \texttt{q2b\_answer} variable. You may
create new cells to run code as long as you don't delete the cell below.

    \begin{Verbatim}[commandchars=\\\{\}]
{\color{incolor}In [{\color{incolor}24}]:} \PY{c+c1}{\PYZsh{} use this cell for scratch work}
         \PY{c+c1}{\PYZsh{} consider using groupby or value\PYZus{}counts() on the \PYZsq{}name\PYZsq{} or \PYZsq{}business\PYZus{}id\PYZsq{} }
         \PY{n}{ans\PYZus{}1} \PY{o}{=} \PY{n+nb}{len}\PY{p}{(}\PY{n}{bus}\PY{o}{.}\PY{n}{groupby}\PY{p}{(}\PY{p}{[}\PY{l+s+s1}{\PYZsq{}}\PY{l+s+s1}{business\PYZus{}id}\PY{l+s+s1}{\PYZsq{}}\PY{p}{]}\PY{p}{)}\PY{p}{)}
         \PY{n}{ans\PYZus{}2} \PY{o}{=} \PY{l+s+s1}{\PYZsq{}}\PY{l+s+s1}{Each record represents a store.}\PY{l+s+s1}{\PYZsq{}}
         \PY{n}{ans\PYZus{}3} \PY{o}{=} \PY{l+s+s1}{\PYZsq{}}\PY{l+s+s1}{The primary key is business\PYZus{}id}\PY{l+s+s1}{\PYZsq{}}
\end{Verbatim}


    \begin{Verbatim}[commandchars=\\\{\}]
{\color{incolor}In [{\color{incolor}25}]:} \PY{n}{q2b\PYZus{}answer} \PY{o}{=} \PY{l+s+s2}{\PYZdq{}}\PY{l+s+s2}{1. There are }\PY{l+s+s2}{\PYZdq{}} \PY{o}{+}\PY{n+nb}{str}\PY{p}{(}\PY{n}{ans\PYZus{}1}\PY{p}{)} \PY{o}{+}\PY{l+s+s2}{\PYZdq{}}\PY{l+s+s2}{ records.}\PY{l+s+s2}{\PYZdq{}}\PY{o}{+}\PY{l+s+s2}{\PYZdq{}}\PY{l+s+se}{\PYZbs{}n}\PY{l+s+s2}{2. }\PY{l+s+s2}{\PYZdq{}}\PY{o}{+}\PY{n}{ans\PYZus{}2} \PY{o}{+}\PY{l+s+s2}{\PYZdq{}}\PY{l+s+se}{\PYZbs{}n}\PY{l+s+s2}{3. }\PY{l+s+s2}{\PYZdq{}}\PY{o}{+}\PY{n}{ans\PYZus{}3}\PY{o}{+}\PY{l+s+s2}{\PYZdq{}}\PY{l+s+s2}{.}\PY{l+s+s2}{\PYZdq{}}
         \PY{n+nb}{print}\PY{p}{(}\PY{n}{q2b\PYZus{}answer}\PY{p}{)}
\end{Verbatim}


    \begin{Verbatim}[commandchars=\\\{\}]
1. There are 6406 records.
2. Each record represents a store.
3. The primary key is business\_id.

    \end{Verbatim}

    \begin{center}\rule{0.5\linewidth}{\linethickness}\end{center}

\subsection{3: Zip code}\label{zip-code}

Next, let's explore some of the variables in the business table. We
begin by examining the postal code.

\subsubsection{Question 3a}\label{question-3a}

What kind of values are in the \texttt{postal\ code} column in the
\texttt{bus} data frame?\\
1. Are zip codes quantitative or qualitative? If qualitative, is it
ordinal or nominal? 1. How are the zip code values encoded in python:
ints, floats, strings, booleans ...?

To answer the second question you might want to examine a particular
entry using the Python \texttt{type} command.

    \begin{Verbatim}[commandchars=\\\{\}]
{\color{incolor}In [{\color{incolor}26}]:} \PY{n+nb}{type}\PY{p}{(}\PY{n}{bus}\PY{p}{[}\PY{l+s+s1}{\PYZsq{}}\PY{l+s+s1}{postal\PYZus{}code}\PY{l+s+s1}{\PYZsq{}}\PY{p}{]}\PY{p}{[}\PY{l+m+mi}{0}\PY{p}{]}\PY{p}{)}
\end{Verbatim}


\begin{Verbatim}[commandchars=\\\{\}]
{\color{outcolor}Out[{\color{outcolor}26}]:} str
\end{Verbatim}
            
    \begin{Verbatim}[commandchars=\\\{\}]
{\color{incolor}In [{\color{incolor}27}]:} \PY{c+c1}{\PYZsh{} Use this cell for your explorations.}
         \PY{n}{q3a\PYZus{}answer} \PY{o}{=} \PY{l+s+s2}{\PYZdq{}\PYZdq{}\PYZdq{}}\PY{l+s+s2}{1. Zip codes are qualitative and it is normial.}
         \PY{l+s+s2}{2. Zip code values are encoded in python as string.}\PY{l+s+s2}{\PYZdq{}\PYZdq{}\PYZdq{}}
         
         \PY{n+nb}{print}\PY{p}{(}\PY{n}{q3a\PYZus{}answer}\PY{p}{)}
\end{Verbatim}


    \begin{Verbatim}[commandchars=\\\{\}]
1. Zip codes are qualitative and it is normial.
2. Zip code values are encoded in python as string.

    \end{Verbatim}

    \subsubsection{Question 3b}\label{question-3b}

To explore the zip code values, it makes sense to examine counts, i.e.,
the number of records that have the same zip code value. This is
essentially answering the question: How many restaurants are in each zip
code?

    In the cell below, create a series where the index is the postal code
and the value is the number of businesses in that postal code. For
example, in 94110 (hey that's my old zip code!), there should be 596
businesses. Your series should be in descending order, i.e. 94110 should
be at the top.

For this answer, use \texttt{groupby}, \texttt{size}, and
\texttt{sort\_values}.

    \begin{Verbatim}[commandchars=\\\{\}]
{\color{incolor}In [{\color{incolor}28}]:} \PY{n}{zip\PYZus{}counts} \PY{o}{=} \PY{n}{bus}\PY{o}{.}\PY{n}{groupby}\PY{p}{(}\PY{l+s+s1}{\PYZsq{}}\PY{l+s+s1}{postal\PYZus{}code}\PY{l+s+s1}{\PYZsq{}}\PY{p}{)}\PY{o}{.}\PY{n}{size}\PY{p}{(}\PY{p}{)}\PY{o}{.}\PY{n}{sort\PYZus{}values}\PY{p}{(}\PY{n}{ascending} \PY{o}{=} \PY{k+kc}{False}\PY{p}{)}
\end{Verbatim}


    Unless you know pandas well already, your answer probably has one subtle
flaw in it: it fails to take into account businesses with missing zip
codes. Unfortunately, missing data is just a reality when we're working
with real data.

    There are a couple of ways to include null postal codes in the
zip\_counts series above. One approach is to use \texttt{fillna}, which
will replace all null (a.k.a. NaN) values with a string of our choosing.
In the example below, I picked "?????". When you run the code below, you
should see that there are 240 businesses with missing zip code.

    \begin{Verbatim}[commandchars=\\\{\}]
{\color{incolor}In [{\color{incolor}29}]:} \PY{n}{zip\PYZus{}counts} \PY{o}{=} \PY{n}{bus}\PY{o}{.}\PY{n}{fillna}\PY{p}{(}\PY{l+s+s2}{\PYZdq{}}\PY{l+s+s2}{?????}\PY{l+s+s2}{\PYZdq{}}\PY{p}{)}\PY{o}{.}\PY{n}{groupby}\PY{p}{(}\PY{l+s+s2}{\PYZdq{}}\PY{l+s+s2}{postal\PYZus{}code}\PY{l+s+s2}{\PYZdq{}}\PY{p}{)}\PY{o}{.}\PY{n}{size}\PY{p}{(}\PY{p}{)}\PY{o}{.}\PY{n}{sort\PYZus{}values}\PY{p}{(}\PY{n}{ascending}\PY{o}{=}\PY{k+kc}{False}\PY{p}{)}
         \PY{n}{zip\PYZus{}counts}\PY{o}{.}\PY{n}{head}\PY{p}{(}\PY{l+m+mi}{15}\PY{p}{)}
\end{Verbatim}


\begin{Verbatim}[commandchars=\\\{\}]
{\color{outcolor}Out[{\color{outcolor}29}]:} postal\_code
         94110    596
         94103    552
         94102    462
         94107    460
         94133    426
         94109    380
         94111    277
         94122    273
         94118    249
         94115    243
         ?????    240
         94105    232
         94108    228
         94114    223
         94117    204
         dtype: int64
\end{Verbatim}
            
    An alternate approach is to use the DataFrame \texttt{value\_counts}
method with the optional argument \texttt{dropna=False}, which will
ensure that null values are counted. In this case, the index will be
\texttt{NaN} for the row corresponding to a null postal code.

    \begin{Verbatim}[commandchars=\\\{\}]
{\color{incolor}In [{\color{incolor}30}]:} \PY{n}{bus}\PY{p}{[}\PY{l+s+s2}{\PYZdq{}}\PY{l+s+s2}{postal\PYZus{}code}\PY{l+s+s2}{\PYZdq{}}\PY{p}{]}\PY{o}{.}\PY{n}{value\PYZus{}counts}\PY{p}{(}\PY{n}{dropna}\PY{o}{=}\PY{k+kc}{False}\PY{p}{)}\PY{o}{.}\PY{n}{sort\PYZus{}values}\PY{p}{(}\PY{n}{ascending} \PY{o}{=} \PY{k+kc}{False}\PY{p}{)}\PY{o}{.}\PY{n}{head}\PY{p}{(}\PY{l+m+mi}{15}\PY{p}{)}
\end{Verbatim}


\begin{Verbatim}[commandchars=\\\{\}]
{\color{outcolor}Out[{\color{outcolor}30}]:} 94110    596
         94103    552
         94102    462
         94107    460
         94133    426
         94109    380
         94111    277
         94122    273
         94118    249
         94115    243
         NaN      240
         94105    232
         94108    228
         94114    223
         94117    204
         Name: postal\_code, dtype: int64
\end{Verbatim}
            
    Missing zip codes aren't our only problem. There is also some bad data
where the postal code got messed up, e.g., there are 3 'Ca' and 3 'CA'
values. Additionally, there are some extended postal codes that are 9
digits long, rather than the typical 5 digits.

Let's clean up the extended zip codes by dropping the digits beyond the
first 5. Rather than deleting replacing the old values in the
\texttt{postal\_code} columnm, we'll instead create a new column called
\texttt{postal\_code\_5}.

The reason we're making a new column is because it's typically good
practice to keep the original values when we are manipulating data. This
makes it easier to recover from mistakes, and also makes it more clear
that we are not working with the original raw data.

    \begin{Verbatim}[commandchars=\\\{\}]
{\color{incolor}In [{\color{incolor}31}]:} \PY{c+c1}{\PYZsh{} Run me}
         \PY{n}{bus}\PY{p}{[}\PY{l+s+s1}{\PYZsq{}}\PY{l+s+s1}{postal\PYZus{}code\PYZus{}5}\PY{l+s+s1}{\PYZsq{}}\PY{p}{]} \PY{o}{=} \PY{n}{bus}\PY{p}{[}\PY{l+s+s1}{\PYZsq{}}\PY{l+s+s1}{postal\PYZus{}code}\PY{l+s+s1}{\PYZsq{}}\PY{p}{]}\PY{o}{.}\PY{n}{str}\PY{p}{[}\PY{p}{:}\PY{l+m+mi}{5}\PY{p}{]}
         \PY{n}{bus}
\end{Verbatim}


\begin{Verbatim}[commandchars=\\\{\}]
{\color{outcolor}Out[{\color{outcolor}31}]:}       business\_id                                  name  \textbackslash{}
         0              19                 NRGIZE LIFESTYLE CAFE   
         1              24    OMNI S.F. HOTEL - 2ND FLOOR PANTRY   
         2              31        NORMAN'S ICE CREAM AND FREEZES   
         3              45                   CHARLIE'S DELI CAFE   
         4              48                            ART'S CAFE   
         5              54                   RHODA GOLDMAN PLAZA   
         6              56                            CAFE X + O   
         7              58                           OASIS GRILL   
         8              61                              CHOWDERS   
         9              66                      STARBUCKS COFFEE   
         10             67                       REVOLUTION CAFE   
         11             73                     DINO'S UNCLE VITO   
         12             76    OMNI S.F. HOTEL - 3RD FLOOR PANTRY   
         13             77  OMNI S.F. HOTEL - EMPLOYEE CAFETERIA   
         14             80                       LAW SCHOOL CAFE   
         15             81                   CLUB ED/BON APPETIT   
         16             88                          J.B.'S PLACE   
         17             95                                  VEGA   
         18             98                          XOX TRUFFLES   
         19             99         J \& M A-1 CAFE RESTAURANT LLC   
         20            101                      CABLE CAR CORNER   
         21            102                     AKIKO'S SUSHI BAR   
         22            108                             RUE LEPIC   
         23            116             THE WATERFRONT RESTAURANT   
         24            121                          AKIKOS SUSHI   
         25            125                           CENTERFOLDS   
         26            134                                  MINT   
         27            140                        CAFE MADELEINE   
         28            141          AFC SUSHI @ MOLLIE STONE'S 2   
         29            146                 DEJA VU PIZZA \& PASTA   
         {\ldots}           {\ldots}                                   {\ldots}   
         6376        94305               ROSAMUNDE SAUSAGE GRILL   
         6377        94310                         YOKAI EXPRESS   
         6378        94318                        YUANBAO JIAOZI   
         6379        94331                     MATCHA CAFE MAIKO   
         6380        94334              SUBWAY SANDWICHES \#53761   
         6381        94337              SUBWAY SANDWICHES \#61240   
         6382        94354               RAINBOW MARKET AND DELI   
         6383        94387                       FOUNDATION CAFE   
         6384        94388                       FOUNDATION CAFE   
         6385        94394                        KOKIO REPUBLIC   
         6386        94408                     SIZZLING POT KING   
         6387        94409                           AUGUST HALL   
         6388        94412                 NATIVE BAKING COMPANY   
         6389        94433                        GREEK TOWN LLC   
         6390        94442                           SIMPLY CAFE   
         6391        94456                UBER-ATG (BON APPETIT)   
         6392        94460                           DOBBS FERRY   
         6393        94465                        BEAUTIFULL LLC   
         6394        94468                             BAR CRENN   
         6395        94502                   NEW FORTUNE DIM SUM   
         6396        94521                JOE \& THE JUICE HOWARD   
         6397        94522                        CAFE JOSEPHINE   
         6398        94537         BON APPETIT @ USF- OUTTA HERE   
         6399        94540                          FOAM USA LLC   
         6400        94542                            OCEAN THAI   
         6401        94544                          D'MAIZE CAFE   
         6402        94555             EASY BREEZY FROZEN YOGURT   
         6403        94571                THE PHOENIX PASTIFICIO   
         6404        94572                 BROADWAY DIM SUM CAFE   
         6405        94574                           BINKA BITES   
         
                                     address           city state postal\_code  \textbackslash{}
         0      1200 VAN NESS AVE, 3RD FLOOR  San Francisco    CA       94109   
         1     500 CALIFORNIA ST, 2ND  FLOOR  San Francisco    CA       94104   
         2              2801 LEAVENWORTH ST   San Francisco    CA       94133   
         3                   3202 FOLSOM ST   San Francisco    CA       94110   
         4                    747 IRVING ST   San Francisco    CA       94122   
         5                     2180 POST ST   San Francisco    CA       94115   
         6                   1799 CHURCH ST   San Francisco    CA       94131   
         7                      91 DRUMM ST   San Francisco    CA       94111   
         8                 PIER 39  SPACE A3  San Francisco    CA       94133   
         9                   1800 IRVING ST   San Francisco    CA       94122   
         10                    3248 22ND ST   San Francisco    CA       94110   
         11                2101 FILLMORE ST   San Francisco    CA       94115   
         12     500 CALIFORNIA ST, 3RD FLOOR  San Francisco    CA       94104   
         13      500 CALIFORNIA ST, BASEMENT  San Francisco    CA       94104   
         14                  2199 FULTON ST   San Francisco    CA       94117   
         15                    2350 TURK ST   San Francisco    CA       94117   
         16                    1435 17TH ST   San Francisco    CA       94107   
         17                419 CORTLAND AVE   San Francisco    CA       94110   
         18                754 COLUMBUS AVE   San Francisco    CA       94133   
         19                     779 CLAY ST   San Francisco    CA       94108   
         20                  1099 POWELL ST   San Francisco    CA       94108   
         21                   542A MASON ST   San Francisco    CA       94102   
         22                      900 PINE ST  San Francisco    CA       94108   
         23               PIER 7 EMBARCADERO  San Francisco    CA       94111   
         24                     431 BUSH ST   San Francisco    CA       94108   
         25                 391 BROADWAY ST   San Francisco    CA       94133   
         26                400 MCALLISTER ST  San Francisco    CA       94102   
         27               300 CALIFORNIA ST   San Francisco    CA       94104   
         28              2435 CALIFORNIA ST   San Francisco    CA       94115   
         29                    3227 16TH ST   San Francisco    CA       94103   
         {\ldots}                             {\ldots}            {\ldots}   {\ldots}         {\ldots}   
         6376                 545 HAIGHT ST   San Francisco    CA       94117   
         6377                    135 4TH ST   San Francisco    CA       94103   
         6378                2110 IRVING ST   San Francisco    CA       94122   
         6379            1581 WEBSTER ST 175  San Francisco    CA       94115   
         6380               160 BROADWAY ST   San Francisco    CA       94111   
         6381              425 D BATTERY ST   San Francisco    CA       94111   
         6382                 684 LARKIN ST   San Francisco    CA       94109   
         6383                    645 5TH ST   San Francisco    CA       94107   
         6384                 335 KEARNY ST   San Francisco    CA       94108   
         6385                   428 11TH ST   San Francisco    CA       94109   
         6386                    139 8TH ST   San Francisco    CA       94103   
         6387                  420 MASON ST   San Francisco    CA         NaN   
         6388           1324 FITZGERALD AVE   San Francisco    CA       94124   
         6389                    88 02ND ST   San Francisco    CA       94105   
         6390                  340 GROVE ST   San Francisco    CA       94102   
         6391             581 20TH ST 2ND FL  San Francisco    CA       94107   
         6392                  409 GOUGH ST   San Francisco    CA       94102   
         6393            3401 CALIFORNIA ST   San Francisco    CA       94118   
         6394              3131 FILLMORE ST   San Francisco    CA       94123   
         6395               811 STOCKTON ST   San Francisco    CA       94108   
         6396                 301 HOWARD ST   San Francisco    CA       94105   
         6397                199 MUSEUM WAY   San Francisco    CA       94114   
         6398                2130 FULTON ST   San Francisco    CA       94117   
         6399               1745 TARAVAL ST   San Francisco    CA       94116   
         6400                2545 OCEAN AVE   San Francisco    CA       94132   
         6401                 50 PHELAN AVE   San Francisco    CA       94112   
         6402            44 WEST PORTAL AVE   San Francisco    CA       94127   
         6403                200 CLEMENT ST   San Francisco    CA       94118   
         6404               684 BROADWAY ST   San Francisco    CA       94133   
         6405               2241 GEARY BLVD   San Francisco    CA       94115   
         
                latitude   longitude  phone\_number postal\_code\_5  
         0     37.786848 -122.421547  +14157763262         94109  
         1     37.792888 -122.403135  +14156779494         94104  
         2     37.807155 -122.419004           NaN         94133  
         3     37.747114 -122.413641  +14156415051         94110  
         4     37.764013 -122.465749  +14156657440         94122  
         5     37.784626 -122.437734  +14153455060         94115  
         6     37.742325 -122.426476  +14158263535         94131  
         7     37.794483 -122.396584  +14158341942         94111  
         8     37.808240 -122.410189  +14153914737         94133  
         9     37.763578 -122.477461  +14152427970         94122  
         10    37.755419 -122.419542  +14156420474         94110  
         11    37.788932 -122.433895  +14159224700         94115  
         12    37.792888 -122.403135  +14156779494         94104  
         13    37.792888 -122.403135  +14156779494         94104  
         14    37.774941 -122.452797  +14154222268         94117  
         15    37.778468 -122.448484  +14154225849         94117  
         16    37.765003 -122.398084  +14155848446         94107  
         17    37.739207 -122.417447  +14152856000         94110  
         18    37.801665 -122.412104  +14154214814         94133  
         19    37.794293 -122.405967  +14156057219         94108  
         20    37.794615 -122.409705  +14153625925         94108  
         21    37.788484 -122.410045  +14159898218         94102  
         22    37.790868 -122.410854  +14154746070         94108  
         23    37.793874 -122.396464  +14153912696         94111  
         24    37.790643 -122.404676  +14153973218         94108  
         25    37.798233 -122.403637  +14158340662         94133  
         26    37.780247 -122.418974  +14155515942         94102  
         27    37.793268 -122.400323  +14153623332         94104  
         28    37.788773 -122.434697  +14155674902         94115  
         29    37.764713 -122.424709  +14152551600         94103  
         {\ldots}         {\ldots}         {\ldots}           {\ldots}           {\ldots}  
         6376        NaN         NaN  +14154376851         94117  
         6377        NaN         NaN  +14158234502         94103  
         6378        NaN         NaN  +14156013979         94122  
         6379        NaN         NaN  +14150009434         94115  
         6380        NaN         NaN  +14158861913         94111  
         6381        NaN         NaN  +14153991549         94111  
         6382        NaN         NaN  +14157664681         94109  
         6383        NaN         NaN  +14153503301         94107  
         6384        NaN         NaN           NaN         94108  
         6385        NaN         NaN  +14157996404         94109  
         6386        NaN         NaN  +14158028899         94103  
         6387        NaN         NaN           NaN           NaN  
         6388        NaN         NaN           NaN         94124  
         6389        NaN         NaN  +14152408032         94105  
         6390        NaN         NaN  +14156587659         94102  
         6391        NaN         NaN  +14158184997         94107  
         6392        NaN         NaN  +14155517709         94102  
         6393        NaN         NaN  +14157289080         94118  
         6394        NaN         NaN           NaN         94123  
         6395        NaN         NaN  +14153991511         94108  
         6396        NaN         NaN           NaN         94105  
         6397        NaN         NaN  +14153508976         94114  
         6398        NaN         NaN  +14153604802         94117  
         6399        NaN         NaN  +14156060018         94116  
         6400        NaN         NaN  +14155857251         94132  
         6401        NaN         NaN  +14154240604         94112  
         6402        NaN         NaN  +14155053351         94127  
         6403        NaN         NaN  +14154726100         94118  
         6404        NaN         NaN           NaN         94133  
         6405        NaN         NaN  +14157712907         94115  
         
         [6406 rows x 10 columns]
\end{Verbatim}
            
    \subsubsection{Question 3c : A Closer Look at Missing Zip
Codes}\label{question-3c-a-closer-look-at-missing-zip-codes}

Let's look more closely at businesses with missing zip codes. We'll see
that many zip codes are missing for a good reason. Examine the
businesses with missing zipcode values. Pay attention to their
addresses. Do you notice anything interesting? You might need to look at
a bunch of entries, i.e. don't just look at the first five.

\emph{Hint: You can use the series \texttt{isnull} method to create a
binary array, which can then be used to show only rows of the dataframe
that contain null values.}

    \begin{Verbatim}[commandchars=\\\{\}]
{\color{incolor}In [{\color{incolor}32}]:} \PY{c+c1}{\PYZsh{} Use this cell for your explorations.}
         \PY{n}{q3c\PYZus{}answer} \PY{o}{=} \PY{l+s+sa}{r}\PY{l+s+s2}{\PYZdq{}\PYZdq{}\PYZdq{}}
         \PY{l+s+s2}{The most addresses of businesses that have null post code are off the grid or approved private locations or approved locations.}\PY{l+s+s2}{\PYZdq{}\PYZdq{}\PYZdq{}}
         \PY{n+nb}{print}\PY{p}{(}\PY{n}{bus}\PY{p}{[}\PY{n}{bus}\PY{p}{[}\PY{l+s+s1}{\PYZsq{}}\PY{l+s+s1}{postal\PYZus{}code\PYZus{}5}\PY{l+s+s1}{\PYZsq{}}\PY{p}{]}\PY{o}{.}\PY{n}{isnull}\PY{p}{(}\PY{p}{)}\PY{p}{]}\PY{o}{.}\PY{n}{groupby}\PY{p}{(}\PY{l+s+s1}{\PYZsq{}}\PY{l+s+s1}{address}\PY{l+s+s1}{\PYZsq{}}\PY{p}{)}\PY{o}{.}\PY{n}{size}\PY{p}{(}\PY{p}{)}\PY{o}{.}\PY{n}{sort\PYZus{}values}\PY{p}{(}\PY{n}{ascending} \PY{o}{=} \PY{k+kc}{False}\PY{p}{)}\PY{o}{.}\PY{n}{head}\PY{p}{(}\PY{l+m+mi}{10}\PY{p}{)}\PY{p}{)}
         
         
         \PY{n+nb}{print}\PY{p}{(}\PY{n}{q3c\PYZus{}answer}\PY{p}{)}
\end{Verbatim}


    \begin{Verbatim}[commandchars=\\\{\}]
address
 OFF THE GRID                                  69
 APPROVED PRIVATE LOCATIONS                     6
 APPROVED LOCATIONS                             4
VARIOUS LOCATIONS                               2
 JUSTIN HERMAN PLAZA                            2
428 11TH ST                                     2
 OTG                                            2
OFF THE GRID                                    2
 FRONT, BETWEEN CALIFORNIA \& SACRAMENTO ST      1
 GOLDEN GATE PARK                               1
dtype: int64

The most addresses of businesses that have null post code are off the grid or approved private locations or approved locations.

    \end{Verbatim}

    \subsubsection{Question 3d: Incorrect Zip
Codes}\label{question-3d-incorrect-zip-codes}

    This dataset is supposed to be only about San Francisco, so let's set up
a list of all San Francisco zip codes.

    \begin{Verbatim}[commandchars=\\\{\}]
{\color{incolor}In [{\color{incolor}33}]:} \PY{n}{all\PYZus{}sf\PYZus{}zip\PYZus{}codes} \PY{o}{=} \PY{p}{[}\PY{l+s+s2}{\PYZdq{}}\PY{l+s+s2}{94102}\PY{l+s+s2}{\PYZdq{}}\PY{p}{,} \PY{l+s+s2}{\PYZdq{}}\PY{l+s+s2}{94103}\PY{l+s+s2}{\PYZdq{}}\PY{p}{,} \PY{l+s+s2}{\PYZdq{}}\PY{l+s+s2}{94104}\PY{l+s+s2}{\PYZdq{}}\PY{p}{,} \PY{l+s+s2}{\PYZdq{}}\PY{l+s+s2}{94105}\PY{l+s+s2}{\PYZdq{}}\PY{p}{,} \PY{l+s+s2}{\PYZdq{}}\PY{l+s+s2}{94107}\PY{l+s+s2}{\PYZdq{}}\PY{p}{,} \PY{l+s+s2}{\PYZdq{}}\PY{l+s+s2}{94108}\PY{l+s+s2}{\PYZdq{}}\PY{p}{,} \PY{l+s+s2}{\PYZdq{}}\PY{l+s+s2}{94109}\PY{l+s+s2}{\PYZdq{}}\PY{p}{,} \PY{l+s+s2}{\PYZdq{}}\PY{l+s+s2}{94110}\PY{l+s+s2}{\PYZdq{}}\PY{p}{,} \PY{l+s+s2}{\PYZdq{}}\PY{l+s+s2}{94111}\PY{l+s+s2}{\PYZdq{}}\PY{p}{,} \PY{l+s+s2}{\PYZdq{}}\PY{l+s+s2}{94112}\PY{l+s+s2}{\PYZdq{}}\PY{p}{,} \PY{l+s+s2}{\PYZdq{}}\PY{l+s+s2}{94114}\PY{l+s+s2}{\PYZdq{}}\PY{p}{,} \PY{l+s+s2}{\PYZdq{}}\PY{l+s+s2}{94115}\PY{l+s+s2}{\PYZdq{}}\PY{p}{,} \PY{l+s+s2}{\PYZdq{}}\PY{l+s+s2}{94116}\PY{l+s+s2}{\PYZdq{}}\PY{p}{,} \PY{l+s+s2}{\PYZdq{}}\PY{l+s+s2}{94117}\PY{l+s+s2}{\PYZdq{}}\PY{p}{,} \PY{l+s+s2}{\PYZdq{}}\PY{l+s+s2}{94118}\PY{l+s+s2}{\PYZdq{}}\PY{p}{,} \PY{l+s+s2}{\PYZdq{}}\PY{l+s+s2}{94119}\PY{l+s+s2}{\PYZdq{}}\PY{p}{,} \PY{l+s+s2}{\PYZdq{}}\PY{l+s+s2}{94120}\PY{l+s+s2}{\PYZdq{}}\PY{p}{,} \PY{l+s+s2}{\PYZdq{}}\PY{l+s+s2}{94121}\PY{l+s+s2}{\PYZdq{}}\PY{p}{,} \PY{l+s+s2}{\PYZdq{}}\PY{l+s+s2}{94122}\PY{l+s+s2}{\PYZdq{}}\PY{p}{,} \PY{l+s+s2}{\PYZdq{}}\PY{l+s+s2}{94123}\PY{l+s+s2}{\PYZdq{}}\PY{p}{,} \PY{l+s+s2}{\PYZdq{}}\PY{l+s+s2}{94124}\PY{l+s+s2}{\PYZdq{}}\PY{p}{,} \PY{l+s+s2}{\PYZdq{}}\PY{l+s+s2}{94125}\PY{l+s+s2}{\PYZdq{}}\PY{p}{,} \PY{l+s+s2}{\PYZdq{}}\PY{l+s+s2}{94126}\PY{l+s+s2}{\PYZdq{}}\PY{p}{,} \PY{l+s+s2}{\PYZdq{}}\PY{l+s+s2}{94127}\PY{l+s+s2}{\PYZdq{}}\PY{p}{,} \PY{l+s+s2}{\PYZdq{}}\PY{l+s+s2}{94128}\PY{l+s+s2}{\PYZdq{}}\PY{p}{,} \PY{l+s+s2}{\PYZdq{}}\PY{l+s+s2}{94129}\PY{l+s+s2}{\PYZdq{}}\PY{p}{,} \PY{l+s+s2}{\PYZdq{}}\PY{l+s+s2}{94130}\PY{l+s+s2}{\PYZdq{}}\PY{p}{,} \PY{l+s+s2}{\PYZdq{}}\PY{l+s+s2}{94131}\PY{l+s+s2}{\PYZdq{}}\PY{p}{,} \PY{l+s+s2}{\PYZdq{}}\PY{l+s+s2}{94132}\PY{l+s+s2}{\PYZdq{}}\PY{p}{,} \PY{l+s+s2}{\PYZdq{}}\PY{l+s+s2}{94133}\PY{l+s+s2}{\PYZdq{}}\PY{p}{,} \PY{l+s+s2}{\PYZdq{}}\PY{l+s+s2}{94134}\PY{l+s+s2}{\PYZdq{}}\PY{p}{,} \PY{l+s+s2}{\PYZdq{}}\PY{l+s+s2}{94137}\PY{l+s+s2}{\PYZdq{}}\PY{p}{,} \PY{l+s+s2}{\PYZdq{}}\PY{l+s+s2}{94139}\PY{l+s+s2}{\PYZdq{}}\PY{p}{,} \PY{l+s+s2}{\PYZdq{}}\PY{l+s+s2}{94140}\PY{l+s+s2}{\PYZdq{}}\PY{p}{,} \PY{l+s+s2}{\PYZdq{}}\PY{l+s+s2}{94141}\PY{l+s+s2}{\PYZdq{}}\PY{p}{,} \PY{l+s+s2}{\PYZdq{}}\PY{l+s+s2}{94142}\PY{l+s+s2}{\PYZdq{}}\PY{p}{,} \PY{l+s+s2}{\PYZdq{}}\PY{l+s+s2}{94143}\PY{l+s+s2}{\PYZdq{}}\PY{p}{,} \PY{l+s+s2}{\PYZdq{}}\PY{l+s+s2}{94144}\PY{l+s+s2}{\PYZdq{}}\PY{p}{,} \PY{l+s+s2}{\PYZdq{}}\PY{l+s+s2}{94145}\PY{l+s+s2}{\PYZdq{}}\PY{p}{,} \PY{l+s+s2}{\PYZdq{}}\PY{l+s+s2}{94146}\PY{l+s+s2}{\PYZdq{}}\PY{p}{,} \PY{l+s+s2}{\PYZdq{}}\PY{l+s+s2}{94147}\PY{l+s+s2}{\PYZdq{}}\PY{p}{,} \PY{l+s+s2}{\PYZdq{}}\PY{l+s+s2}{94151}\PY{l+s+s2}{\PYZdq{}}\PY{p}{,} \PY{l+s+s2}{\PYZdq{}}\PY{l+s+s2}{94158}\PY{l+s+s2}{\PYZdq{}}\PY{p}{,} \PY{l+s+s2}{\PYZdq{}}\PY{l+s+s2}{94159}\PY{l+s+s2}{\PYZdq{}}\PY{p}{,} \PY{l+s+s2}{\PYZdq{}}\PY{l+s+s2}{94160}\PY{l+s+s2}{\PYZdq{}}\PY{p}{,} \PY{l+s+s2}{\PYZdq{}}\PY{l+s+s2}{94161}\PY{l+s+s2}{\PYZdq{}}\PY{p}{,} \PY{l+s+s2}{\PYZdq{}}\PY{l+s+s2}{94163}\PY{l+s+s2}{\PYZdq{}}\PY{p}{,} \PY{l+s+s2}{\PYZdq{}}\PY{l+s+s2}{94164}\PY{l+s+s2}{\PYZdq{}}\PY{p}{,} \PY{l+s+s2}{\PYZdq{}}\PY{l+s+s2}{94172}\PY{l+s+s2}{\PYZdq{}}\PY{p}{,} \PY{l+s+s2}{\PYZdq{}}\PY{l+s+s2}{94177}\PY{l+s+s2}{\PYZdq{}}\PY{p}{,} \PY{l+s+s2}{\PYZdq{}}\PY{l+s+s2}{94188}\PY{l+s+s2}{\PYZdq{}}\PY{p}{]}
\end{Verbatim}


    Set \texttt{weird\_zip\_code\_businesses} equal to a new dataframe
showing only rows corresponding to zip codes that are not valid AND not
NaN. Use the \texttt{postal\_code\_5} field.

\emph{Hint: The \texttt{\textasciitilde{}} operator inverts a boolean
array. Use in conjunction with \texttt{isin}.}

\emph{Hint: The \texttt{notnull} method can be used to form a useful
boolean array for this problem.}

    \begin{Verbatim}[commandchars=\\\{\}]
{\color{incolor}In [{\color{incolor}34}]:} \PY{n}{weird\PYZus{}zip\PYZus{}code\PYZus{}businesses} \PY{o}{=} \PY{n}{bus}\PY{p}{[}\PY{o}{\PYZti{}}\PY{n}{bus}\PY{p}{[}\PY{l+s+s1}{\PYZsq{}}\PY{l+s+s1}{postal\PYZus{}code\PYZus{}5}\PY{l+s+s1}{\PYZsq{}}\PY{p}{]}\PY{o}{.}\PY{n}{isin}\PY{p}{(}\PY{n}{all\PYZus{}sf\PYZus{}zip\PYZus{}codes}\PY{p}{)} \PY{o}{\PYZam{}} \PY{n}{bus}\PY{p}{[}\PY{l+s+s1}{\PYZsq{}}\PY{l+s+s1}{postal\PYZus{}code\PYZus{}5}\PY{l+s+s1}{\PYZsq{}}\PY{p}{]}\PY{o}{.}\PY{n}{notnull}\PY{p}{(}\PY{p}{)}\PY{p}{]}
\end{Verbatim}


    \begin{Verbatim}[commandchars=\\\{\}]
{\color{incolor}In [{\color{incolor}35}]:} \PY{n}{weird\PYZus{}zip\PYZus{}code\PYZus{}businesses}
\end{Verbatim}


\begin{Verbatim}[commandchars=\\\{\}]
{\color{outcolor}Out[{\color{outcolor}35}]:}       business\_id                                               name  \textbackslash{}
         1211         5208                             GOLDEN GATE YACHT CLUB   
         1372         5755                                      J \& J VENDING   
         1373         5757                                  RICO VENDING, INC   
         2258        36547                                    EPIC ROASTHOUSE   
         2293        37167  INTERCONTINENTAL SAN FRANCISCO EMPLOYEE CAFETERIA   
         2295        37169     INTERCONTINENTAL SAN FRANCISCO 4TH FL. KITCHEN   
         2846        64540                                     LEO'S HOT DOGS   
         2852        64660                               HAIGHT STREET MARKET   
         2857        64738                                          JAPACURRY   
         2969        65856                                        BAMBOO ASIA   
         3142        67875                                 THE CHAIRMAN TRUCK   
         3665        72127                                   REVOLUTION FOODS   
         3758        74674                                     ELI'S HOT DOGS   
         4853        83744                                      LA FROMAGERIE   
         5060        85459                                         ORBIT ROOM   
         5325        87059                              COFFEE BAR-MONTGOMERY   
         5480        88139                                        TACOLICIOUS   
         5894        90733                                          JEEPSILOG   
         6002        91249                                          AN THE GO   
         6130        92141                                       ALFARO TRUCK   
         6300        93484                               CARDONA'S FOOD TRUCK   
         
                                    address           city state postal\_code  \textbackslash{}
         1211                   1 YACHT RD   San Francisco    CA         941   
         1372       VARIOUS LOACATIONS (17)  San Francisco    CA       94545   
         1373             VARIOUS LOCATIONS  San Francisco    CA       94066   
         2258       PIER 26 EMBARARCADERO    San Francisco    CA       95105   
         2293       888 HOWARD ST 2ND FLOOR  San Francisco    CA       94013   
         2295       888 HOWARD ST 4TH FLOOR  San Francisco    CA       94013   
         2846              2301 MISSION ST   San Francisco    CA          CA   
         2852               1530 HAIGHT ST   San Francisco    CA       92672   
         2857                      PUBLIC    San Francisco    CA          CA   
         2969             41 MONTGOMERY ST   San Francisco    CA       94101   
         3142                OFF THE GRID    San Francisco    CA       00000   
         3665                5383 CAPWELL    San Francisco    CA       94621   
         3758            101 BAYSHORE BLVD   San Francisco    CA       94014   
         4853            101 MONTGOMERY ST   San Francisco    CA       94101   
         5060               1900 MARKET ST   San Francisco    CA       94602   
         5325  101 MONTGOMERY ST SUITE 101C  San Francisco    CA       94014   
         5480             2250 CHESTNUT ST   San Francisco    CA          Ca   
         5894                2 MARINA BLVD   San Francisco    CA       94080   
         6002                OFF THE GRID    San Francisco    CA       00000   
         6130              332 VALENCIA ST   San Francisco    CA       64110   
         6300              2430 WHIPPLE RD   San Francisco    CA       94544   
         
                latitude   longitude  phone\_number postal\_code\_5  
         1211  37.807878 -122.442499  +14153462628           941  
         1372        NaN         NaN  +14156750910         94545  
         1373        NaN         NaN  +14155836723         94066  
         2258  37.788962 -122.387941  +14153699955         95105  
         2293  37.781664 -122.404778  +14156166532         94013  
         2295  37.781664 -122.404778  +14156166532         94013  
         2846  37.760054 -122.419166  +14152406434            CA  
         2852  37.769957 -122.447533  +14152550643         92672  
         2857  37.777122 -122.419639  +14152444785            CA  
         2969  37.774998 -122.418299  +14156246790         94101  
         3142  37.777122 -122.419639  +14158461711         00000  
         3665        NaN         NaN           NaN         94621  
         3758        NaN         NaN  +14158301168         94014  
         4853        NaN         NaN  +14153682943         94101  
         5060        NaN         NaN  +14153705584         94602  
         5325        NaN         NaN  +14158158774         94014  
         5480        NaN         NaN  +14156496077            Ca  
         5894        NaN         NaN  +14157035586         94080  
         6002        NaN         NaN  +14158192000         00000  
         6130        NaN         NaN  +14159409273         64110  
         6300        NaN         NaN  +14153365990         94544  
\end{Verbatim}
            
    If we were doing very serious data analysis, we might indivdually look
up every one of these strange records. Let's focus on just two of them:
zip codes 94545 and 94602. Use a search engine to identify what cities
these zip codes appear in. Try to explain why you think these two zip
codes appear in your dataframe. For the one with zip code 94602, try
searching for the business name and locate its real address.

    \begin{Verbatim}[commandchars=\\\{\}]
{\color{incolor}In [{\color{incolor}36}]:} \PY{n}{weird\PYZus{}zip\PYZus{}code\PYZus{}businesses}\PY{p}{[}\PY{n}{weird\PYZus{}zip\PYZus{}code\PYZus{}businesses}\PY{p}{[}\PY{l+s+s1}{\PYZsq{}}\PY{l+s+s1}{postal\PYZus{}code\PYZus{}5}\PY{l+s+s1}{\PYZsq{}}\PY{p}{]}\PY{o}{==}\PY{l+s+s1}{\PYZsq{}}\PY{l+s+s1}{94602}\PY{l+s+s1}{\PYZsq{}}\PY{p}{]}
\end{Verbatim}


\begin{Verbatim}[commandchars=\\\{\}]
{\color{outcolor}Out[{\color{outcolor}36}]:}       business\_id        name          address           city state  \textbackslash{}
         5060        85459  ORBIT ROOM  1900 MARKET ST   San Francisco    CA   
         
              postal\_code  latitude  longitude  phone\_number postal\_code\_5  
         5060       94602       NaN        NaN  +14153705584         94602  
\end{Verbatim}
            
    \begin{Verbatim}[commandchars=\\\{\}]
{\color{incolor}In [{\color{incolor}37}]:} \PY{n}{weird\PYZus{}zip\PYZus{}code\PYZus{}businesses}\PY{p}{[}\PY{n}{weird\PYZus{}zip\PYZus{}code\PYZus{}businesses}\PY{p}{[}\PY{l+s+s1}{\PYZsq{}}\PY{l+s+s1}{postal\PYZus{}code\PYZus{}5}\PY{l+s+s1}{\PYZsq{}}\PY{p}{]}\PY{o}{==}\PY{l+s+s1}{\PYZsq{}}\PY{l+s+s1}{94545}\PY{l+s+s1}{\PYZsq{}}\PY{p}{]}
\end{Verbatim}


\begin{Verbatim}[commandchars=\\\{\}]
{\color{outcolor}Out[{\color{outcolor}37}]:}       business\_id           name                  address           city  \textbackslash{}
         1372         5755  J \& J VENDING  VARIOUS LOACATIONS (17)  San Francisco   
         
              state postal\_code  latitude  longitude  phone\_number postal\_code\_5  
         1372    CA       94545       NaN        NaN  +14156750910         94545  
\end{Verbatim}
            
    \begin{Verbatim}[commandchars=\\\{\}]
{\color{incolor}In [{\color{incolor}38}]:} \PY{c+c1}{\PYZsh{} Use this cell for your explorations.}
         \PY{n}{q3d\PYZus{}answer} \PY{o}{=} \PY{l+s+sa}{r}\PY{l+s+s2}{\PYZdq{}\PYZdq{}\PYZdq{}}
         \PY{l+s+s2}{94545 \PYZhy{} The real address of J \PYZam{} J Vending is 33500 Western Ave, Union City, CA 94587. And there is another J And Amp J Vending located at 2700 Mccone Avenue, Hayward, CA 94545. These two addresses and zip codes may be mixed up. }
         
         \PY{l+s+s2}{94602 \PYZhy{} The real address of Orbit Room is 1900 Market St, San Francisco, CA 94102. Its wrong zip code appeared to be recorded wrongly with only one different digit.}
         \PY{l+s+s2}{\PYZdq{}\PYZdq{}\PYZdq{}}
         
         \PY{n+nb}{print}\PY{p}{(}\PY{n}{q3d\PYZus{}answer}\PY{p}{)}
\end{Verbatim}


    \begin{Verbatim}[commandchars=\\\{\}]

94545 - The real address of J \& J Vending is 33500 Western Ave, Union City, CA 94587. And there is another J And Amp J Vending located at 2700 Mccone Avenue, Hayward, CA 94545. These two addresses and zip codes may be mixed up. 

94602 - The real address of Orbit Room is 1900 Market St, San Francisco, CA 94102. Its wrong zip code appeared to be recorded wrongly with only one different digit.


    \end{Verbatim}

    \subsubsection{Question 3e}\label{question-3e}

We often want to clean the data to improve our analysis. This cleaning
might include changing values for a variable or dropping records.

Let's correct 94602 to the more likely value based on your analysis.
Let's modify the derived field \texttt{zip\_code} using
\texttt{bus{[}\textquotesingle{}zip\_code\textquotesingle{}{]}.str.replace}
to replace 94602 with the correct value based on this business's real
address that you learn by using a search engine.

    \begin{Verbatim}[commandchars=\\\{\}]
{\color{incolor}In [{\color{incolor}39}]:} \PY{c+c1}{\PYZsh{} WARNING: Be careful when uncommenting the line below, it will set the entire column to NaN unless you }
         \PY{c+c1}{\PYZsh{} put something to the right of the ellipses.}
         \PY{c+c1}{\PYZsh{} bus[\PYZsq{}postal\PYZus{}code\PYZus{}5\PYZsq{}] = ... }
         \PY{n}{bus}\PY{p}{[}\PY{l+s+s1}{\PYZsq{}}\PY{l+s+s1}{postal\PYZus{}code\PYZus{}5}\PY{l+s+s1}{\PYZsq{}}\PY{p}{]} \PY{o}{=} \PY{n}{bus}\PY{p}{[}\PY{l+s+s1}{\PYZsq{}}\PY{l+s+s1}{postal\PYZus{}code\PYZus{}5}\PY{l+s+s1}{\PYZsq{}}\PY{p}{]}\PY{o}{.}\PY{n}{str}\PY{o}{.}\PY{n}{replace}\PY{p}{(}\PY{l+s+s1}{\PYZsq{}}\PY{l+s+s1}{94602}\PY{l+s+s1}{\PYZsq{}}\PY{p}{,}\PY{l+s+s1}{\PYZsq{}}\PY{l+s+s1}{94102}\PY{l+s+s1}{\PYZsq{}}\PY{p}{)}
\end{Verbatim}


    \begin{Verbatim}[commandchars=\\\{\}]
{\color{incolor}In [{\color{incolor}40}]:} \PY{k}{assert} \PY{l+s+s2}{\PYZdq{}}\PY{l+s+s2}{94602}\PY{l+s+s2}{\PYZdq{}} \PY{o+ow}{not} \PY{o+ow}{in} \PY{n}{bus}\PY{p}{[}\PY{l+s+s1}{\PYZsq{}}\PY{l+s+s1}{postal\PYZus{}code\PYZus{}5}\PY{l+s+s1}{\PYZsq{}}\PY{p}{]}
\end{Verbatim}


    \begin{center}\rule{0.5\linewidth}{\linethickness}\end{center}

\subsection{4: Latitude and Longitude}\label{latitude-and-longitude}

Let's also consider latitude and longitude values and get a sense of how
many are missing.

\subsubsection{Question 4a}\label{question-4a}

How many businesses are missing longitude values?

\emph{Hint: Use isnull.}

    \begin{Verbatim}[commandchars=\\\{\}]
{\color{incolor}In [{\color{incolor}41}]:} \PY{n}{missing\PYZus{}latlongs} \PY{o}{=} \PY{n+nb}{sum}\PY{p}{(}\PY{n}{bus}\PY{p}{[}\PY{l+s+s1}{\PYZsq{}}\PY{l+s+s1}{longitude}\PY{l+s+s1}{\PYZsq{}}\PY{p}{]}\PY{o}{.}\PY{n}{isnull}\PY{p}{(}\PY{p}{)}\PY{p}{)}
\end{Verbatim}


    Do not delete the empty cell below!

    As a somewhat contrived exercise in data manipulation, let's try to
identify which zip codes are missing the most longitude values.

    Throughout problems 4a and 4b, let's focus on only the "dense" zip codes
of the city of San Francisco, listed below as \texttt{sf\_dense\_zip}.

    \begin{Verbatim}[commandchars=\\\{\}]
{\color{incolor}In [{\color{incolor}42}]:} \PY{n}{sf\PYZus{}dense\PYZus{}zip} \PY{o}{=} \PY{p}{[}\PY{l+s+s2}{\PYZdq{}}\PY{l+s+s2}{94102}\PY{l+s+s2}{\PYZdq{}}\PY{p}{,} \PY{l+s+s2}{\PYZdq{}}\PY{l+s+s2}{94103}\PY{l+s+s2}{\PYZdq{}}\PY{p}{,} \PY{l+s+s2}{\PYZdq{}}\PY{l+s+s2}{94104}\PY{l+s+s2}{\PYZdq{}}\PY{p}{,} \PY{l+s+s2}{\PYZdq{}}\PY{l+s+s2}{94105}\PY{l+s+s2}{\PYZdq{}}\PY{p}{,} \PY{l+s+s2}{\PYZdq{}}\PY{l+s+s2}{94107}\PY{l+s+s2}{\PYZdq{}}\PY{p}{,} \PY{l+s+s2}{\PYZdq{}}\PY{l+s+s2}{94108}\PY{l+s+s2}{\PYZdq{}}\PY{p}{,}
                     \PY{l+s+s2}{\PYZdq{}}\PY{l+s+s2}{94109}\PY{l+s+s2}{\PYZdq{}}\PY{p}{,} \PY{l+s+s2}{\PYZdq{}}\PY{l+s+s2}{94110}\PY{l+s+s2}{\PYZdq{}}\PY{p}{,} \PY{l+s+s2}{\PYZdq{}}\PY{l+s+s2}{94111}\PY{l+s+s2}{\PYZdq{}}\PY{p}{,} \PY{l+s+s2}{\PYZdq{}}\PY{l+s+s2}{94112}\PY{l+s+s2}{\PYZdq{}}\PY{p}{,} \PY{l+s+s2}{\PYZdq{}}\PY{l+s+s2}{94114}\PY{l+s+s2}{\PYZdq{}}\PY{p}{,} \PY{l+s+s2}{\PYZdq{}}\PY{l+s+s2}{94115}\PY{l+s+s2}{\PYZdq{}}\PY{p}{,}
                     \PY{l+s+s2}{\PYZdq{}}\PY{l+s+s2}{94116}\PY{l+s+s2}{\PYZdq{}}\PY{p}{,} \PY{l+s+s2}{\PYZdq{}}\PY{l+s+s2}{94117}\PY{l+s+s2}{\PYZdq{}}\PY{p}{,} \PY{l+s+s2}{\PYZdq{}}\PY{l+s+s2}{94118}\PY{l+s+s2}{\PYZdq{}}\PY{p}{,} \PY{l+s+s2}{\PYZdq{}}\PY{l+s+s2}{94121}\PY{l+s+s2}{\PYZdq{}}\PY{p}{,} \PY{l+s+s2}{\PYZdq{}}\PY{l+s+s2}{94122}\PY{l+s+s2}{\PYZdq{}}\PY{p}{,} \PY{l+s+s2}{\PYZdq{}}\PY{l+s+s2}{94123}\PY{l+s+s2}{\PYZdq{}}\PY{p}{,} 
                     \PY{l+s+s2}{\PYZdq{}}\PY{l+s+s2}{94124}\PY{l+s+s2}{\PYZdq{}}\PY{p}{,} \PY{l+s+s2}{\PYZdq{}}\PY{l+s+s2}{94127}\PY{l+s+s2}{\PYZdq{}}\PY{p}{,} \PY{l+s+s2}{\PYZdq{}}\PY{l+s+s2}{94131}\PY{l+s+s2}{\PYZdq{}}\PY{p}{,} \PY{l+s+s2}{\PYZdq{}}\PY{l+s+s2}{94132}\PY{l+s+s2}{\PYZdq{}}\PY{p}{,} \PY{l+s+s2}{\PYZdq{}}\PY{l+s+s2}{94133}\PY{l+s+s2}{\PYZdq{}}\PY{p}{,} \PY{l+s+s2}{\PYZdq{}}\PY{l+s+s2}{94134}\PY{l+s+s2}{\PYZdq{}}\PY{p}{]}
\end{Verbatim}


    In the cell below, create a series where the index is
\texttt{postal\_code\_5}, and the value is the number of businesses with
missing longitudes in that zip code. Your series should be in descending
order. Only businesses from \texttt{sf\_dense\_zip} should be included.

For example, 94110 should be at the top of the series, with the value
294.

*Hint: Start by making a new dataframe called \texttt{bus\_sf} that only
has businesses from \texttt{sf\_dense\_zip}.

\emph{Hint: Create a custom function to compute the number of null
entries in a series, and use this function with the \texttt{agg}
method.}

    \begin{Verbatim}[commandchars=\\\{\}]
{\color{incolor}In [{\color{incolor}43}]:} \PY{n}{bus\PYZus{}sf} \PY{o}{=} \PY{n}{bus}\PY{p}{[}\PY{n}{bus}\PY{p}{[}\PY{l+s+s1}{\PYZsq{}}\PY{l+s+s1}{postal\PYZus{}code\PYZus{}5}\PY{l+s+s1}{\PYZsq{}}\PY{p}{]}\PY{o}{.}\PY{n}{isin}\PY{p}{(}\PY{n}{sf\PYZus{}dense\PYZus{}zip}\PY{p}{)}\PY{p}{]}
         \PY{n}{num\PYZus{}missing\PYZus{}in\PYZus{}each\PYZus{}zip} \PY{o}{=} \PY{n}{bus\PYZus{}sf}\PY{o}{.}\PY{n}{groupby}\PY{p}{(}\PY{l+s+s1}{\PYZsq{}}\PY{l+s+s1}{postal\PYZus{}code\PYZus{}5}\PY{l+s+s1}{\PYZsq{}}\PY{p}{)}\PY{p}{[}\PY{l+s+s1}{\PYZsq{}}\PY{l+s+s1}{longitude}\PY{l+s+s1}{\PYZsq{}}\PY{p}{]}\PY{o}{.}\PY{n}{agg}\PY{p}{(}\PY{k}{lambda} \PY{n}{x}\PY{p}{:} \PY{n+nb}{sum}\PY{p}{(}\PY{n}{x}\PY{o}{.}\PY{n}{isnull}\PY{p}{(}\PY{p}{)}\PY{p}{)}\PY{p}{)}\PY{o}{.}\PY{n}{astype}\PY{p}{(}\PY{l+s+s1}{\PYZsq{}}\PY{l+s+s1}{int}\PY{l+s+s1}{\PYZsq{}}\PY{p}{)}\PY{o}{.}\PY{n}{sort\PYZus{}values}\PY{p}{(}\PY{n}{ascending} \PY{o}{=}\PY{k+kc}{False}\PY{p}{)}
\end{Verbatim}


    \begin{Verbatim}[commandchars=\\\{\}]
{\color{incolor}In [{\color{incolor}44}]:} \PY{n}{num\PYZus{}missing\PYZus{}in\PYZus{}each\PYZus{}zip}
\end{Verbatim}


\begin{Verbatim}[commandchars=\\\{\}]
{\color{outcolor}Out[{\color{outcolor}44}]:} postal\_code\_5
         94110    294
         94103    285
         94107    275
         94102    222
         94109    171
         94133    159
         94122    132
         94111    129
         94105    127
         94124    118
         94118    117
         94114    111
         94108     98
         94115     95
         94117     86
         94104     79
         94112     77
         94132     71
         94123     68
         94121     60
         94116     42
         94134     36
         94127     30
         94131     16
         Name: longitude, dtype: int64
\end{Verbatim}
            
    Do not edit the empty cell below.

    \subsubsection{Question 4b}\label{question-4b}

In question 4a, we counted the number of null values per zip code. Let's
now count the proportion of null values.

Create a new dataframe of counts of the null and proportion of null
values, storing the result in \texttt{bus\_sf\_latlong}. It should have
an index called \texttt{postal\_code\_5} and should also have 3 columns:

\begin{enumerate}
\def\labelenumi{\arabic{enumi}.}
\tightlist
\item
  \texttt{null\ count}: The number of missing values for the zip code.
\item
  \texttt{not\ null\ count}: The proportion of present values for the
  zip code.
\item
  \texttt{fraction\ null}: The fraction of values that are null for the
  zip code.
\end{enumerate}

Your data frame should be sorted by the fraction null in descending
order.

Recommended approach: Build three series with the appropriate names and
data and then combine them into a dataframe. This will require some new
syntax you may not have seen. You already have code from question 4a
that computes the \texttt{null\ count} series.

To pursue this recommended approach, you might find these two functions
useful:

\begin{itemize}
\tightlist
\item
  \texttt{rename}: Renames the values of a series.
\item
  \texttt{pd.concat}: Can be used to combine a list of Series into a
  dataframe. Example: \texttt{pd.concat({[}s1,\ s2,\ s3{]},\ axis=1)}
  will combine series 1, 2, and 3 into a dataframe.
\end{itemize}

\emph{Hint: You can use the divison operator to compute the ratio of two
series.}

\emph{Hint: The \textasciitilde{} operator can invert a binary array. Or
alternately, the \texttt{notnull} method can be used to create a binary
array from a series.}

\emph{Note: An alternate approach is to create three aggregation
functions as pass them in a list to the \texttt{agg} function.}

    \begin{Verbatim}[commandchars=\\\{\}]
{\color{incolor}In [{\color{incolor}45}]:} \PY{n}{null\PYZus{}count} \PY{o}{=} \PY{n}{bus\PYZus{}sf}\PY{o}{.}\PY{n}{groupby}\PY{p}{(}\PY{l+s+s1}{\PYZsq{}}\PY{l+s+s1}{postal\PYZus{}code\PYZus{}5}\PY{l+s+s1}{\PYZsq{}}\PY{p}{)}\PY{p}{[}\PY{l+s+s1}{\PYZsq{}}\PY{l+s+s1}{longitude}\PY{l+s+s1}{\PYZsq{}}\PY{p}{]}\PY{o}{.}\PY{n}{agg}\PY{p}{(}\PY{k}{lambda} \PY{n}{x}\PY{p}{:} \PY{n+nb}{sum}\PY{p}{(}\PY{n}{x}\PY{o}{.}\PY{n}{isnull}\PY{p}{(}\PY{p}{)}\PY{p}{)}\PY{p}{)}\PY{o}{.}\PY{n}{astype}\PY{p}{(}\PY{l+s+s1}{\PYZsq{}}\PY{l+s+s1}{int}\PY{l+s+s1}{\PYZsq{}}\PY{p}{)}\PY{o}{.}\PY{n}{rename}\PY{p}{(}\PY{l+s+s2}{\PYZdq{}}\PY{l+s+s2}{null count}\PY{l+s+s2}{\PYZdq{}}\PY{p}{)}
         \PY{n}{not\PYZus{}null\PYZus{}count} \PY{o}{=} \PY{n}{bus\PYZus{}sf}\PY{o}{.}\PY{n}{groupby}\PY{p}{(}\PY{l+s+s1}{\PYZsq{}}\PY{l+s+s1}{postal\PYZus{}code\PYZus{}5}\PY{l+s+s1}{\PYZsq{}}\PY{p}{)}\PY{p}{[}\PY{l+s+s1}{\PYZsq{}}\PY{l+s+s1}{longitude}\PY{l+s+s1}{\PYZsq{}}\PY{p}{]}\PY{o}{.}\PY{n}{agg}\PY{p}{(}\PY{k}{lambda} \PY{n}{x}\PY{p}{:} \PY{n+nb}{sum}\PY{p}{(}\PY{n}{x}\PY{o}{.}\PY{n}{notnull}\PY{p}{(}\PY{p}{)}\PY{p}{)}\PY{p}{)}\PY{o}{.}\PY{n}{astype}\PY{p}{(}\PY{l+s+s1}{\PYZsq{}}\PY{l+s+s1}{int}\PY{l+s+s1}{\PYZsq{}}\PY{p}{)}\PY{o}{.}\PY{n}{rename}\PY{p}{(}\PY{l+s+s2}{\PYZdq{}}\PY{l+s+s2}{not null count}\PY{l+s+s2}{\PYZdq{}}\PY{p}{)}
         \PY{n}{fraction\PYZus{}missing} \PY{o}{=} \PY{p}{(}\PY{n}{null\PYZus{}count} \PY{o}{/} \PY{p}{(}\PY{n}{not\PYZus{}null\PYZus{}count}\PY{o}{+}\PY{n}{null\PYZus{}count}\PY{p}{)}\PY{p}{)}\PY{o}{.}\PY{n}{rename}\PY{p}{(}\PY{l+s+s1}{\PYZsq{}}\PY{l+s+s1}{fraction null}\PY{l+s+s1}{\PYZsq{}}\PY{p}{)}
         \PY{n}{fraction\PYZus{}missing\PYZus{}df} \PY{o}{=} \PY{n}{pd}\PY{o}{.}\PY{n}{concat}\PY{p}{(}\PY{p}{[}\PY{n}{null\PYZus{}count}\PY{p}{,}\PY{n}{not\PYZus{}null\PYZus{}count}\PY{p}{,}\PY{n}{fraction\PYZus{}missing}\PY{p}{]}\PY{p}{,}\PY{n}{axis}\PY{o}{=}\PY{l+m+mi}{1}\PY{p}{)}
         \PY{n}{fraction\PYZus{}missing\PYZus{}df} \PY{o}{=} \PY{n}{fraction\PYZus{}missing\PYZus{}df}\PY{o}{.}\PY{n}{sort\PYZus{}values}\PY{p}{(}\PY{l+s+s1}{\PYZsq{}}\PY{l+s+s1}{fraction null}\PY{l+s+s1}{\PYZsq{}}\PY{p}{,}\PY{n}{ascending}\PY{o}{=}\PY{k+kc}{False}\PY{p}{)}
\end{Verbatim}


    Do not edit the empty cell below.

    \subsection{Summary of the Business
Data}\label{summary-of-the-business-data}

Before we move on to explore the other data, let's take stock of what we
have learned and the implications of our findings on future analysis.

\begin{itemize}
\tightlist
\item
  We found that the business id is unique across records and so we may
  be able to use it as a key in joining tables.
\item
  We found that there are many errors with the zip codes. As a result,
  we may want to drop the records with zip codes outside of San
  Francisco or to treat them differently. For some of the bad values, we
  could take the time to look up the restaurant address online and fix
  these errors.\\
\item
  We found that there are a huge number of missing longitude (and
  latitude) values. Fixing would require a lot of work, but could in
  principle be automated for business with well formed addresses.
\end{itemize}

    \begin{center}\rule{0.5\linewidth}{\linethickness}\end{center}

\subsection{5: Investigate the Inspection
Data}\label{investigate-the-inspection-data}

Let's now turn to the inspection DataFrame. Earlier, we found that
\texttt{ins} has 4 columns named \texttt{business\_id}, \texttt{score},
\texttt{date} and \texttt{type}. In this section, we determine the
granularity of \texttt{ins} and investigate the kinds of information
provided for the inspections.

    Let's start by looking again at the first 5 rows of \texttt{ins} to see
what we're working with.

    \begin{Verbatim}[commandchars=\\\{\}]
{\color{incolor}In [{\color{incolor}46}]:} \PY{n}{ins}\PY{o}{.}\PY{n}{head}\PY{p}{(}\PY{l+m+mi}{5}\PY{p}{)}
\end{Verbatim}


\begin{Verbatim}[commandchars=\\\{\}]
{\color{outcolor}Out[{\color{outcolor}46}]:}    business\_id  score      date     type
         0           19     94  20160513  routine
         1           19     94  20171211  routine
         2           24     98  20171101  routine
         3           24     98  20161005  routine
         4           24     96  20160311  routine
\end{Verbatim}
            
    \subsubsection{Question 5a}\label{question-5a}

From calling \texttt{head}, we know that each row in this table
corresponds to the inspection of a single business. Let's get a sense of
the total number of inspections conducted, as well as the total number
of unique businesses that occur in the dataset.

    \begin{Verbatim}[commandchars=\\\{\}]
{\color{incolor}In [{\color{incolor}47}]:} \PY{c+c1}{\PYZsh{} The number of rows in ins}
         \PY{n}{rows\PYZus{}in\PYZus{}table} \PY{o}{=} \PY{n+nb}{len}\PY{p}{(}\PY{n}{ins}\PY{p}{)}
         
         \PY{c+c1}{\PYZsh{} The number of unique business IDs in ins.}
         \PY{n}{unique\PYZus{}ins\PYZus{}ids} \PY{o}{=} \PY{n+nb}{len}\PY{p}{(}\PY{n}{ins}\PY{p}{[}\PY{l+s+s1}{\PYZsq{}}\PY{l+s+s1}{business\PYZus{}id}\PY{l+s+s1}{\PYZsq{}}\PY{p}{]}\PY{o}{.}\PY{n}{unique}\PY{p}{(}\PY{p}{)}\PY{p}{)}
\end{Verbatim}


    Do not delete the empty cell below!

    As you should have seen above, we have an average of roughly 3
inspections per business.

    \subsubsection{Question 5b}\label{question-5b}

Next, we examine the Series in the \texttt{ins} dataframe called
\texttt{type}. From examining the first few rows of \texttt{ins}, we see
that \texttt{type} is a string and one of its values is 'routine',
presumably for a routine inspection. What values does \texttt{type} take
on? How many occurrences of each value is in the DataFrame? What are the
implications for further analysis? For this problem, you need only fill
in the string with a description; there's no specific dataframe or
series that you need to create.

    \begin{Verbatim}[commandchars=\\\{\}]
{\color{incolor}In [{\color{incolor}48}]:} \PY{n}{ins}\PY{o}{.}\PY{n}{groupby}\PY{p}{(}\PY{l+s+s1}{\PYZsq{}}\PY{l+s+s1}{type}\PY{l+s+s1}{\PYZsq{}}\PY{p}{)}\PY{o}{.}\PY{n}{size}\PY{p}{(}\PY{p}{)}
\end{Verbatim}


\begin{Verbatim}[commandchars=\\\{\}]
{\color{outcolor}Out[{\color{outcolor}48}]:} type
         complaint        1
         routine      14221
         dtype: int64
\end{Verbatim}
            
    \begin{Verbatim}[commandchars=\\\{\}]
{\color{incolor}In [{\color{incolor}49}]:} \PY{n}{q5b\PYZus{}answer} \PY{o}{=} \PY{l+s+sa}{r}\PY{l+s+s2}{\PYZdq{}\PYZdq{}\PYZdq{}}
         \PY{l+s+s2}{Series }\PY{l+s+s2}{\PYZdq{}}\PY{l+s+s2}{type}\PY{l+s+s2}{\PYZdq{}}\PY{l+s+s2}{ only takes on }\PY{l+s+s2}{\PYZdq{}}\PY{l+s+s2}{complaint}\PY{l+s+s2}{\PYZdq{}}\PY{l+s+s2}{ with 1 occurrences and }\PY{l+s+s2}{\PYZdq{}}\PY{l+s+s2}{routine}\PY{l+s+s2}{\PYZdq{}}\PY{l+s+s2}{ with 14221 occurrences.}
         \PY{l+s+s2}{\PYZdq{}\PYZdq{}\PYZdq{}}
         \PY{n+nb}{print}\PY{p}{(}\PY{n}{q5b\PYZus{}answer}\PY{p}{)}
\end{Verbatim}


    \begin{Verbatim}[commandchars=\\\{\}]

Series "type" only takes on "complaint" with 1 occurrences and "routine" with 14221 occurrences.


    \end{Verbatim}

    \subsubsection{Question 5c}\label{question-5c}

In this question, we're going to try to figure out what years the data
spans. Unfortunately, the dates in our file are formatted as strings
such as \texttt{20160503}, which are a little tricky to interpret. The
ideal solution for this problem is to modify our dates so that they are
in an appropriate format for analysis.

In the cell below, we attempt to add a new column to \texttt{ins} called
\texttt{new\_date} which contains the \texttt{date} stored as a datetime
object. This calls the \texttt{pd.to\_datetime} method, which converts a
series of string representations of dates (and/or times) to a series
containing a datetime object.

    \begin{Verbatim}[commandchars=\\\{\}]
{\color{incolor}In [{\color{incolor}50}]:} \PY{n}{ins}\PY{p}{[}\PY{l+s+s1}{\PYZsq{}}\PY{l+s+s1}{new\PYZus{}date}\PY{l+s+s1}{\PYZsq{}}\PY{p}{]} \PY{o}{=} \PY{n}{pd}\PY{o}{.}\PY{n}{to\PYZus{}datetime}\PY{p}{(}\PY{n}{ins}\PY{p}{[}\PY{l+s+s1}{\PYZsq{}}\PY{l+s+s1}{date}\PY{l+s+s1}{\PYZsq{}}\PY{p}{]}\PY{p}{)}
         \PY{n}{ins}\PY{o}{.}\PY{n}{head}\PY{p}{(}\PY{l+m+mi}{5}\PY{p}{)}
\end{Verbatim}


\begin{Verbatim}[commandchars=\\\{\}]
{\color{outcolor}Out[{\color{outcolor}50}]:}    business\_id  score      date     type                      new\_date
         0           19     94  20160513  routine 1970-01-01 00:00:00.020160513
         1           19     94  20171211  routine 1970-01-01 00:00:00.020171211
         2           24     98  20171101  routine 1970-01-01 00:00:00.020171101
         3           24     98  20161005  routine 1970-01-01 00:00:00.020161005
         4           24     96  20160311  routine 1970-01-01 00:00:00.020160311
\end{Verbatim}
            
    As you'll see, the resulting \texttt{new\_date} column doesn't make any
sense. This is because the default behavior of the
\texttt{to\_datetime()} method does not properly process the passed
string. We can fix this by telling \texttt{to\_datetime} how to do its
job by providing a format string.

    \begin{Verbatim}[commandchars=\\\{\}]
{\color{incolor}In [{\color{incolor}51}]:} \PY{n}{ins}\PY{p}{[}\PY{l+s+s1}{\PYZsq{}}\PY{l+s+s1}{new\PYZus{}date}\PY{l+s+s1}{\PYZsq{}}\PY{p}{]} \PY{o}{=} \PY{n}{pd}\PY{o}{.}\PY{n}{to\PYZus{}datetime}\PY{p}{(}\PY{n}{ins}\PY{p}{[}\PY{l+s+s1}{\PYZsq{}}\PY{l+s+s1}{date}\PY{l+s+s1}{\PYZsq{}}\PY{p}{]}\PY{p}{,} \PY{n+nb}{format}\PY{o}{=}\PY{l+s+s1}{\PYZsq{}}\PY{l+s+s1}{\PYZpc{}}\PY{l+s+s1}{Y}\PY{l+s+s1}{\PYZpc{}}\PY{l+s+s1}{m}\PY{l+s+si}{\PYZpc{}d}\PY{l+s+s1}{\PYZsq{}}\PY{p}{)}
         \PY{n}{ins}\PY{o}{.}\PY{n}{head}\PY{p}{(}\PY{l+m+mi}{5}\PY{p}{)}
\end{Verbatim}


\begin{Verbatim}[commandchars=\\\{\}]
{\color{outcolor}Out[{\color{outcolor}51}]:}    business\_id  score      date     type   new\_date
         0           19     94  20160513  routine 2016-05-13
         1           19     94  20171211  routine 2017-12-11
         2           24     98  20171101  routine 2017-11-01
         3           24     98  20161005  routine 2016-10-05
         4           24     96  20160311  routine 2016-03-11
\end{Verbatim}
            
    This is still not ideal for our analysis, so we'll add one more column
that is just equal to the year by using the \texttt{dt.year} property of
the new series we just created.

    \begin{Verbatim}[commandchars=\\\{\}]
{\color{incolor}In [{\color{incolor}52}]:} \PY{n}{ins}\PY{p}{[}\PY{l+s+s1}{\PYZsq{}}\PY{l+s+s1}{year}\PY{l+s+s1}{\PYZsq{}}\PY{p}{]} \PY{o}{=} \PY{n}{ins}\PY{p}{[}\PY{l+s+s1}{\PYZsq{}}\PY{l+s+s1}{new\PYZus{}date}\PY{l+s+s1}{\PYZsq{}}\PY{p}{]}\PY{o}{.}\PY{n}{dt}\PY{o}{.}\PY{n}{year}
         \PY{n}{ins}\PY{o}{.}\PY{n}{head}\PY{p}{(}\PY{l+m+mi}{5}\PY{p}{)}
\end{Verbatim}


\begin{Verbatim}[commandchars=\\\{\}]
{\color{outcolor}Out[{\color{outcolor}52}]:}    business\_id  score      date     type   new\_date  year
         0           19     94  20160513  routine 2016-05-13  2016
         1           19     94  20171211  routine 2017-12-11  2017
         2           24     98  20171101  routine 2017-11-01  2017
         3           24     98  20161005  routine 2016-10-05  2016
         4           24     96  20160311  routine 2016-03-11  2016
\end{Verbatim}
            
    Now that we have this handy \texttt{year} column, we can try to
understand our data better.

What range of years is covered in this data set? Are there roughly the
same number of inspections each year? Provide your answer in text only.

    \begin{Verbatim}[commandchars=\\\{\}]
{\color{incolor}In [{\color{incolor}53}]:} \PY{n}{ins}\PY{o}{.}\PY{n}{groupby}\PY{p}{(}\PY{l+s+s1}{\PYZsq{}}\PY{l+s+s1}{year}\PY{l+s+s1}{\PYZsq{}}\PY{p}{)}\PY{o}{.}\PY{n}{size}\PY{p}{(}\PY{p}{)}
\end{Verbatim}


\begin{Verbatim}[commandchars=\\\{\}]
{\color{outcolor}Out[{\color{outcolor}53}]:} year
         2015    3305
         2016    5443
         2017    5166
         2018     308
         dtype: int64
\end{Verbatim}
            
    \begin{Verbatim}[commandchars=\\\{\}]
{\color{incolor}In [{\color{incolor}54}]:} \PY{n}{q5c\PYZus{}answer} \PY{o}{=} \PY{l+s+sa}{r}\PY{l+s+s2}{\PYZdq{}\PYZdq{}\PYZdq{}}
         \PY{l+s+s2}{The range of years from 2015\PYZhy{}2018 is covered in this data set and there are not roughly the same number of inspections each year. 2015 has 3305 inspections, while 2016, 2017, 2018 has 5443, 5166, 308 inspections respectively.}
         \PY{l+s+s2}{\PYZdq{}\PYZdq{}\PYZdq{}}
         \PY{n+nb}{print}\PY{p}{(}\PY{n}{q5c\PYZus{}answer}\PY{p}{)}
\end{Verbatim}


    \begin{Verbatim}[commandchars=\\\{\}]

The range of years from 2015-2018 is covered in this data set and there are not roughly the same number of inspections each year. 2015 has 3305 inspections, while 2016, 2017, 2018 has 5443, 5166, 308 inspections respectively.


    \end{Verbatim}

    \begin{center}\rule{0.5\linewidth}{\linethickness}\end{center}

\subsection{6: Explore inspection score}\label{explore-inspection-score}

    \subsubsection{Question 6a}\label{question-6a}

Let's look at the distribution of scores. As we saw before when we
called \texttt{head} on this data, inspection scores appear to be
integer values. The discreteness of this variable means that we can use
a barplot to visualize the distribution of the inspection score. Find
the counts of the number of inspections for each score. Specifically,
create a series in \texttt{scoreCts} where the index is the score and
the value is the number of times that score was given.

Then make a bar plot of these counts. It should look like the image
below. It does not need to look exactly the same, but it should be
fairly close.

    \begin{Verbatim}[commandchars=\\\{\}]
{\color{incolor}In [{\color{incolor}55}]:} \PY{n}{scoreCts} \PY{o}{=} \PY{n}{ins}\PY{o}{.}\PY{n}{groupby}\PY{p}{(}\PY{l+s+s1}{\PYZsq{}}\PY{l+s+s1}{score}\PY{l+s+s1}{\PYZsq{}}\PY{p}{)}\PY{o}{.}\PY{n}{size}\PY{p}{(}\PY{p}{)}
         \PY{n}{x} \PY{o}{=} \PY{n}{scoreCts}\PY{o}{.}\PY{n}{index}
         \PY{n}{y} \PY{o}{=} \PY{n}{scoreCts}\PY{o}{.}\PY{n}{values}
         \PY{n}{plt}\PY{o}{.}\PY{n}{bar}\PY{p}{(}\PY{n}{x}\PY{p}{,}\PY{n}{y}\PY{p}{)}
\end{Verbatim}


\begin{Verbatim}[commandchars=\\\{\}]
{\color{outcolor}Out[{\color{outcolor}55}]:} <Container object of 47 artists>
\end{Verbatim}
            
    \begin{center}
    \adjustimage{max size={0.9\linewidth}{0.9\paperheight}}{output_120_1.png}
    \end{center}
    { \hspace*{\fill} \\}
    
    \subsubsection{Question 6b}\label{question-6b}

Describe the qualities of the distribution of the inspections scores
based on your bar plot. Consider the mode(s), symmetry, tails, gaps, and
anamolous values. Are there any unusual features of this distribution?
What do your observations imply about the scores?

    \begin{Verbatim}[commandchars=\\\{\}]
{\color{incolor}In [{\color{incolor}56}]:} \PY{n}{q6b\PYZus{}answer} \PY{o}{=} \PY{l+s+sa}{r}\PY{l+s+s2}{\PYZdq{}\PYZdq{}\PYZdq{}}
         \PY{l+s+s2}{The mode is 100. And the distribution of the inspections scores is asymmetry including more high score and few low score. It implies that people tend to give higher score if the experience can be acceptable.}
         
         \PY{l+s+s2}{The tails including some extreme low scores like 55, 57 and so on. They maybe caused by some extrme unhappy experience of limited customers.}
         
         \PY{l+s+s2}{The score gaps happen a lot extremely at 91, 93. They maybe caused by preference to certain scores like 90 because it is a perfectly round number.}
         
         \PY{l+s+s2}{Anomalous values are mostly extreme low scores caused by review in bad faith or extreme horrible experience of some customers.}
         \PY{l+s+s2}{\PYZdq{}\PYZdq{}\PYZdq{}}
         \PY{n+nb}{print}\PY{p}{(}\PY{n}{q6b\PYZus{}answer}\PY{p}{)}
\end{Verbatim}


    \begin{Verbatim}[commandchars=\\\{\}]

The mode is 100. And the distribution of the inspections scores is asymmetry including more high score and few low score. It implies that people tend to give higher score if the experience can be acceptable.

The tails including some extreme low scores like 55, 57 and so on. They maybe caused by some extrme unhappy experience of limited customers.

The score gaps happen a lot extremely at 91, 93. They maybe caused by preference to certain scores like 90 because it is a perfectly round number.

Anomalous values are mostly extreme low scores caused by review in bad faith or extreme horrible experience of some customers.


    \end{Verbatim}

    \subsubsection{Question 6c}\label{question-6c}

    Let's figure out which restaurants had the worst scores ever. Let's
start by creating a new dataframe called \texttt{ins\_named}. It should
be exactly the same as \texttt{ins}, except that it should have the name
and address of every business, as determined by the \texttt{bus}
dataframe. If a \texttt{business\_id} in \texttt{ins} does not exist in
\texttt{bus}, the name and address should be given as NaN.

\emph{Hint: Use the merge method to join the \texttt{ins} dataframe with
the appropriate portion of the \texttt{bus} dataframe.}

    \begin{Verbatim}[commandchars=\\\{\}]
{\color{incolor}In [{\color{incolor}57}]:} \PY{n}{ins\PYZus{}named} \PY{o}{=} \PY{n}{pd}\PY{o}{.}\PY{n}{merge}\PY{p}{(}\PY{n}{ins}\PY{p}{,}\PY{n}{bus}\PY{o}{.}\PY{n}{loc}\PY{p}{[}\PY{p}{:}\PY{p}{,}\PY{p}{[}\PY{l+s+s1}{\PYZsq{}}\PY{l+s+s1}{business\PYZus{}id}\PY{l+s+s1}{\PYZsq{}}\PY{p}{,}\PY{l+s+s1}{\PYZsq{}}\PY{l+s+s1}{name}\PY{l+s+s1}{\PYZsq{}}\PY{p}{,}\PY{l+s+s1}{\PYZsq{}}\PY{l+s+s1}{address}\PY{l+s+s1}{\PYZsq{}}\PY{p}{]}\PY{p}{]}\PY{p}{,}\PY{n}{right\PYZus{}on}\PY{o}{=}\PY{l+s+s2}{\PYZdq{}}\PY{l+s+s2}{business\PYZus{}id}\PY{l+s+s2}{\PYZdq{}}\PY{p}{,}\PY{n}{left\PYZus{}on}\PY{o}{=}\PY{l+s+s2}{\PYZdq{}}\PY{l+s+s2}{business\PYZus{}id}\PY{l+s+s2}{\PYZdq{}}\PY{p}{,}\PY{n}{how} \PY{o}{=} \PY{l+s+s1}{\PYZsq{}}\PY{l+s+s1}{left}\PY{l+s+s1}{\PYZsq{}}\PY{p}{)}
\end{Verbatim}


    Using this data frame, identify the restaurant with the lowest
inspection scores ever. Optionally: head to yelp.com and look up the
reviews page for this restaurant. Copy and paste anything interesting
you want to share.

    \begin{Verbatim}[commandchars=\\\{\}]
{\color{incolor}In [{\color{incolor}58}]:} \PY{n}{ins\PYZus{}named}\PY{o}{.}\PY{n}{sort\PYZus{}values}\PY{p}{(}\PY{l+s+s1}{\PYZsq{}}\PY{l+s+s1}{score}\PY{l+s+s1}{\PYZsq{}}\PY{p}{)}\PY{o}{.}\PY{n}{iloc}\PY{p}{[}\PY{l+m+mi}{0}\PY{p}{]}
\end{Verbatim}


\begin{Verbatim}[commandchars=\\\{\}]
{\color{outcolor}Out[{\color{outcolor}58}]:} business\_id                  86647
         score                           48
         date                      20160907
         type                       routine
         new\_date       2016-09-07 00:00:00
         year                          2016
         name                       DA CAFE
         address            407 CLEMENT ST 
         Name: 13179, dtype: object
\end{Verbatim}
            
    \begin{Verbatim}[commandchars=\\\{\}]
{\color{incolor}In [{\color{incolor}59}]:} \PY{n}{q6c\PYZus{}answer} \PY{o}{=} \PY{l+s+sa}{r}\PY{l+s+s2}{\PYZdq{}\PYZdq{}\PYZdq{}}
         \PY{l+s+s2}{The restaurant with the lowest inspection scores ever is DA CAFE located at 407 Clement Street.}
         \PY{l+s+s2}{Interesting Review: Entomophagy, otherwise known as the consumption of insects, has been around for thousands of years in some cultures. Not only do these insects apparently taste good, but they}\PY{l+s+s2}{\PYZsq{}}\PY{l+s+s2}{re an inexpensive and nutritious food source.}
         \PY{l+s+s2}{No kidding?}\PY{l+s+s2}{\PYZdq{}\PYZdq{}\PYZdq{}}
         
         \PY{n+nb}{print}\PY{p}{(}\PY{n}{q6c\PYZus{}answer}\PY{p}{)}
\end{Verbatim}


    \begin{Verbatim}[commandchars=\\\{\}]

The restaurant with the lowest inspection scores ever is DA CAFE located at 407 Clement Street.
Interesting Review: Entomophagy, otherwise known as the consumption of insects, has been around for thousands of years in some cultures. Not only do these insects apparently taste good, but they're an inexpensive and nutritious food source.
No kidding?

    \end{Verbatim}

    Just for fun you can also look up the restaurants with the best scores.
You'll see that lots of them aren't restaurants at all!

    \begin{center}\rule{0.5\linewidth}{\linethickness}\end{center}

\subsection{7: Restaurant Ratings Over
Time}\label{restaurant-ratings-over-time}

    Let's consider various scenarios involving restaurants with multiple
ratings over time.

    \subsubsection{Question 7a}\label{question-7a}

    Let's see which restaurant has had the most extreme change in their
ratings. Let the "swing" of a restaurant be defined as the difference
between its lowest and highest ever rating. If a restaurant has been
reviewed fewer than two times, its swing is zero. Using whatever
technique you want to use, identify the three restaurants that are tied
for the maximum swing value.

    \begin{Verbatim}[commandchars=\\\{\}]
{\color{incolor}In [{\color{incolor}60}]:} \PY{n}{score\PYZus{}group\PYZus{}by\PYZus{}id} \PY{o}{=} \PY{n}{ins\PYZus{}named}\PY{o}{.}\PY{n}{groupby}\PY{p}{(}\PY{l+s+s1}{\PYZsq{}}\PY{l+s+s1}{business\PYZus{}id}\PY{l+s+s1}{\PYZsq{}}\PY{p}{)}
         \PY{n}{score\PYZus{}group\PYZus{}by\PYZus{}id}\PY{o}{.}\PY{n}{agg}\PY{p}{(}\PY{k}{lambda} \PY{n}{x}\PY{p}{:} \PY{n+nb}{max}\PY{p}{(}\PY{n}{x}\PY{p}{)}\PY{o}{\PYZhy{}}\PY{n+nb}{min}\PY{p}{(}\PY{n}{x}\PY{p}{)}\PY{p}{)}\PY{p}{[}\PY{l+s+s1}{\PYZsq{}}\PY{l+s+s1}{score}\PY{l+s+s1}{\PYZsq{}}\PY{p}{]}\PY{o}{.}\PY{n}{sort\PYZus{}values}\PY{p}{(}\PY{n}{ascending}\PY{o}{=}\PY{k+kc}{False}\PY{p}{)}\PY{o}{.}\PY{n}{head}\PY{p}{(}\PY{l+m+mi}{3}\PY{p}{)}
\end{Verbatim}


\begin{Verbatim}[commandchars=\\\{\}]
{\color{outcolor}Out[{\color{outcolor}60}]:} business\_id
         71440    39
         2044     39
         73978    39
         Name: score, dtype: int64
\end{Verbatim}
            
    \begin{Verbatim}[commandchars=\\\{\}]
{\color{incolor}In [{\color{incolor}61}]:} \PY{n}{ins\PYZus{}named}\PY{p}{[}\PY{n}{ins\PYZus{}named}\PY{p}{[}\PY{l+s+s1}{\PYZsq{}}\PY{l+s+s1}{business\PYZus{}id}\PY{l+s+s1}{\PYZsq{}}\PY{p}{]}\PY{o}{.}\PY{n}{isin}\PY{p}{(}\PY{p}{[}\PY{l+s+s1}{\PYZsq{}}\PY{l+s+s1}{71440}\PY{l+s+s1}{\PYZsq{}}\PY{p}{,}\PY{l+s+s1}{\PYZsq{}}\PY{l+s+s1}{2044}\PY{l+s+s1}{\PYZsq{}}\PY{p}{,}\PY{l+s+s1}{\PYZsq{}}\PY{l+s+s1}{73978}\PY{l+s+s1}{\PYZsq{}}\PY{p}{]}\PY{p}{)}\PY{p}{]}
\end{Verbatim}


\begin{Verbatim}[commandchars=\\\{\}]
{\color{outcolor}Out[{\color{outcolor}61}]:}       business\_id  score      date     type   new\_date  year  \textbackslash{}
         1207         2044     98  20170130  routine 2017-01-30  2017   
         1208         2044     93  20160328  routine 2016-03-28  2016   
         1209         2044     59  20160204  routine 2016-02-04  2016   
         9491        71440     57  20161121  routine 2016-11-21  2016   
         9492        71440     96  20150817  routine 2015-08-17  2015   
         9713        73978     61  20150316  routine 2015-03-16  2015   
         9714        73978    100  20160913  routine 2016-09-13  2016   
         
                                      name             address  
         1207          JOANIE'S DINER INC.  1329 COLUMBUS AVE   
         1208          JOANIE'S DINER INC.  1329 COLUMBUS AVE   
         1209          JOANIE'S DINER INC.  1329 COLUMBUS AVE   
         9491  NEW GARDEN RESTAURANT, INC.      716 KEARNY ST   
         9492  NEW GARDEN RESTAURANT, INC.      716 KEARNY ST   
         9713                     THE CREW    1330 NORIEGA ST   
         9714                     THE CREW    1330 NORIEGA ST   
\end{Verbatim}
            
    \begin{Verbatim}[commandchars=\\\{\}]
{\color{incolor}In [{\color{incolor}62}]:} \PY{n}{q7a\PYZus{}answer} \PY{o}{=} \PY{l+s+sa}{r}\PY{l+s+s2}{\PYZdq{}\PYZdq{}\PYZdq{}}
         \PY{l+s+s2}{The three restaurants that are tied for the maximum swing value 39 is JOANIE}\PY{l+s+s2}{\PYZsq{}}\PY{l+s+s2}{S DINER INC., NEW GARDEN RESTAURANT, INC. and THE CREW.}
         \PY{l+s+s2}{\PYZdq{}\PYZdq{}\PYZdq{}}
         \PY{n+nb}{print}\PY{p}{(}\PY{n}{q7a\PYZus{}answer}\PY{p}{)}
\end{Verbatim}


    \begin{Verbatim}[commandchars=\\\{\}]

The three restaurants that are tied for the maximum swing value 39 is JOANIE'S DINER INC., NEW GARDEN RESTAURANT, INC. and THE CREW.


    \end{Verbatim}

    \subsubsection{Question 7b}\label{question-7b}

To get a sense of the number of times each restaurant has been
inspected, create a multi-indexed dataframe called
\texttt{inspections\_by\_id\_and\_year} where each row corresponds to
data about a given business in a single year, and there is a single data
column named \texttt{count} that represents the number of inspections
for that business in that year. The first index in the MultiIndex should
be on \texttt{business\_id}, and the second should be on \texttt{year}.

An example row in this dataframe might look tell you that business\_id
is 573, year is 2017, and count is 4.

\emph{Hint: Use groupby to group based on both the \texttt{business\_id}
and the \texttt{year}.}

\emph{Hint: Use rename to change the name of the column to
\texttt{count}.}

    \begin{Verbatim}[commandchars=\\\{\}]
{\color{incolor}In [{\color{incolor}63}]:} \PY{n}{inspections\PYZus{}by\PYZus{}id\PYZus{}and\PYZus{}year} \PY{o}{=} \PY{n}{ins\PYZus{}named}\PY{o}{.}\PY{n}{groupby}\PY{p}{(}\PY{p}{[}\PY{n}{ins\PYZus{}named}\PY{p}{[}\PY{l+s+s1}{\PYZsq{}}\PY{l+s+s1}{business\PYZus{}id}\PY{l+s+s1}{\PYZsq{}}\PY{p}{]}\PY{p}{,}\PY{n}{ins\PYZus{}named}\PY{p}{[}\PY{l+s+s1}{\PYZsq{}}\PY{l+s+s1}{year}\PY{l+s+s1}{\PYZsq{}}\PY{p}{]}\PY{p}{]}\PY{p}{)}\PY{o}{.}\PY{n}{size}\PY{p}{(}\PY{p}{)}\PY{o}{.}\PY{n}{to\PYZus{}frame}\PY{p}{(}\PY{p}{)}
         \PY{n}{inspections\PYZus{}by\PYZus{}id\PYZus{}and\PYZus{}year} \PY{o}{=} \PY{n}{inspections\PYZus{}by\PYZus{}id\PYZus{}and\PYZus{}year}\PY{o}{.}\PY{n}{rename}\PY{p}{(}\PY{n}{columns}\PY{o}{=}\PY{p}{\PYZob{}}\PY{l+m+mi}{0}\PY{p}{:} \PY{l+s+s2}{\PYZdq{}}\PY{l+s+s2}{count}\PY{l+s+s2}{\PYZdq{}}\PY{p}{\PYZcb{}}\PY{p}{)}
\end{Verbatim}


    \begin{Verbatim}[commandchars=\\\{\}]
{\color{incolor}In [{\color{incolor}64}]:} \PY{n}{inspections\PYZus{}by\PYZus{}id\PYZus{}and\PYZus{}year}\PY{o}{.}\PY{n}{loc}\PY{p}{[}\PY{l+m+mi}{19}\PY{p}{]}\PY{o}{.}\PY{n}{loc}\PY{p}{[}\PY{l+m+mi}{2016}\PY{p}{]}\PY{o}{==}\PY{l+m+mi}{1}
\end{Verbatim}


\begin{Verbatim}[commandchars=\\\{\}]
{\color{outcolor}Out[{\color{outcolor}64}]:} count    True
         Name: 2016, dtype: bool
\end{Verbatim}
            
    Do not edit the empty cell below!

    You should see that some businesses are inspected many times in a single
year. Let's get a sense of the distribution of the counts of the number
of inspections by calling \texttt{value\_counts}. There are quite a lot
of businesses with 2 inspections in the same year, so it seems like it
might be interesting to see what we can learn from such businesses.

    \begin{Verbatim}[commandchars=\\\{\}]
{\color{incolor}In [{\color{incolor}65}]:} \PY{n}{inspections\PYZus{}by\PYZus{}id\PYZus{}and\PYZus{}year}\PY{p}{[}\PY{l+s+s1}{\PYZsq{}}\PY{l+s+s1}{count}\PY{l+s+s1}{\PYZsq{}}\PY{p}{]}\PY{o}{.}\PY{n}{value\PYZus{}counts}\PY{p}{(}\PY{p}{)}
\end{Verbatim}


\begin{Verbatim}[commandchars=\\\{\}]
{\color{outcolor}Out[{\color{outcolor}65}]:} 1    9531
         2    2175
         3     111
         4       2
         Name: count, dtype: int64
\end{Verbatim}
            
    \subsubsection{Question 7c}\label{question-7c}

What's the relationship between the first and second scores for the
businesses with 2 inspections in a year? Do they typically improve? For
simplicity, let's focus on only 2016 for this problem.

First, make a dataframe called \texttt{scores\_pairs\_by\_business}
indexed by \texttt{business\_id} (containing only businesses with
exactly 2 inspections in 2016). This dataframe contains the field
\texttt{score\_pair} consisting of the score pairs ordered
chronologically \texttt{{[}first\_score,\ second\_score{]}}.

Plot these scores. That is, make a scatter plot to display these pairs
of scores. Include on the plot a reference line with slope 1.

You may find the functions \texttt{sort\_values}, \texttt{groupby},
\texttt{filter} and \texttt{agg} helpful, though not all necessary.

The first few rows of the resulting table should look something like:

\begin{verbatim}
<tr style="text-align: right;">
  <th></th>
  <th>score_pair</th>
</tr>
<tr>
  <th>business_id</th>
  <th></th>
</tr>
\end{verbatim}

\begin{verbatim}
<tr>
  <th>24</th>
  <td>[96, 98]</td>
</tr>
<tr>
  <th>45</th>
  <td>[78, 84]</td>
</tr>
<tr>
  <th>66</th>
  <td>[98, 100]</td>
</tr>
<tr>
  <th>67</th>
  <td>[87, 94]</td>
</tr>
<tr>
  <th>76</th>
  <td>[100, 98]</td>
</tr>
\end{verbatim}

The scatter plot shoud look like this:

 \emph{Note: Each score pair must be a list type; numpy arrays will not
pass the autograder.}

\emph{Hint: Use the \texttt{filter} method from lecture 3 to create a
new dataframe that only contains restaurants that received exactly 2
inspections.}

\emph{Hint: Our answer is a single line of code that uses
\texttt{sort\_values}, \texttt{groupby}, \texttt{filter},
\texttt{groupby}, \texttt{agg}, and \texttt{rename} in that order. Your
answer does not need to use these exact methods.}

    \begin{Verbatim}[commandchars=\\\{\}]
{\color{incolor}In [{\color{incolor}66}]:} \PY{n}{scores\PYZus{}pairs\PYZus{}by\PYZus{}business}\PY{o}{=}\PY{n}{ins}\PY{o}{.}\PY{n}{sort\PYZus{}values}\PY{p}{(}\PY{l+s+s1}{\PYZsq{}}\PY{l+s+s1}{new\PYZus{}date}\PY{l+s+s1}{\PYZsq{}}\PY{p}{)}\PY{o}{.}\PY{n}{loc}\PY{p}{[}\PY{n}{ins}\PY{p}{[}\PY{l+s+s1}{\PYZsq{}}\PY{l+s+s1}{year}\PY{l+s+s1}{\PYZsq{}}\PY{p}{]}\PY{o}{==}\PY{l+m+mi}{2016}\PY{p}{]}\PY{o}{.}\PY{n}{groupby}\PY{p}{(}\PY{l+s+s1}{\PYZsq{}}\PY{l+s+s1}{business\PYZus{}id}\PY{l+s+s1}{\PYZsq{}}\PY{p}{)}\PY{o}{.}\PY{n}{filter}\PY{p}{(}\PY{k}{lambda} \PY{n}{x}\PY{p}{:} \PY{n+nb}{len}\PY{p}{(}\PY{n}{x}\PY{p}{)} \PY{o}{==}\PY{l+m+mi}{2}\PY{p}{)}\PY{p}{[}\PY{p}{[}\PY{l+s+s1}{\PYZsq{}}\PY{l+s+s1}{business\PYZus{}id}\PY{l+s+s1}{\PYZsq{}}\PY{p}{,}\PY{l+s+s1}{\PYZsq{}}\PY{l+s+s1}{score}\PY{l+s+s1}{\PYZsq{}}\PY{p}{]}\PY{p}{]}\PY{o}{.}\PY{n}{groupby}\PY{p}{(}\PY{l+s+s1}{\PYZsq{}}\PY{l+s+s1}{business\PYZus{}id}\PY{l+s+s1}{\PYZsq{}}\PY{p}{)}\PY{o}{.}\PY{n}{agg}\PY{p}{(}\PY{k}{lambda} \PY{n}{x}\PY{p}{:}\PY{p}{[}\PY{n}{i} \PY{k}{for} \PY{n}{i} \PY{o+ow}{in} \PY{n}{x}\PY{p}{]}\PY{p}{)}\PY{o}{.}\PY{n}{rename}\PY{p}{(}\PY{n}{columns}\PY{o}{=}\PY{p}{\PYZob{}}\PY{l+s+s1}{\PYZsq{}}\PY{l+s+s1}{score}\PY{l+s+s1}{\PYZsq{}}\PY{p}{:} \PY{l+s+s2}{\PYZdq{}}\PY{l+s+s2}{score\PYZus{}pair}\PY{l+s+s2}{\PYZdq{}}\PY{p}{\PYZcb{}}\PY{p}{)}
         \PY{n+nb}{type}\PY{p}{(}\PY{n}{scores\PYZus{}pairs\PYZus{}by\PYZus{}business}\PY{o}{.}\PY{n}{loc}\PY{p}{[}\PY{l+m+mi}{24}\PY{p}{]}\PY{p}{[}\PY{l+m+mi}{0}\PY{p}{]}\PY{p}{)}
\end{Verbatim}


\begin{Verbatim}[commandchars=\\\{\}]
{\color{outcolor}Out[{\color{outcolor}66}]:} list
\end{Verbatim}
            
    \begin{Verbatim}[commandchars=\\\{\}]
{\color{incolor}In [{\color{incolor}67}]:} \PY{n}{scores\PYZus{}pairs\PYZus{}by\PYZus{}business}\PY{o}{.}\PY{n}{head}\PY{p}{(}\PY{l+m+mi}{5}\PY{p}{)}
\end{Verbatim}


\begin{Verbatim}[commandchars=\\\{\}]
{\color{outcolor}Out[{\color{outcolor}67}]:}             score\_pair
         business\_id           
         24            [96, 98]
         45            [78, 84]
         66           [98, 100]
         67            [87, 94]
         76           [100, 98]
\end{Verbatim}
            
    \begin{Verbatim}[commandchars=\\\{\}]
{\color{incolor}In [{\color{incolor}68}]:} \PY{n}{scores\PYZus{}pairs\PYZus{}by\PYZus{}business} \PY{o}{=} \PY{n}{ins}\PY{o}{.}\PY{n}{sort\PYZus{}values}\PY{p}{(}\PY{l+s+s1}{\PYZsq{}}\PY{l+s+s1}{new\PYZus{}date}\PY{l+s+s1}{\PYZsq{}}\PY{p}{)}\PY{o}{.}\PY{n}{loc}\PY{p}{[}\PY{n}{ins}\PY{p}{[}\PY{l+s+s1}{\PYZsq{}}\PY{l+s+s1}{year}\PY{l+s+s1}{\PYZsq{}}\PY{p}{]}\PY{o}{==}\PY{l+m+mi}{2016}\PY{p}{]}\PY{o}{.}\PY{n}{groupby}\PY{p}{(}\PY{l+s+s1}{\PYZsq{}}\PY{l+s+s1}{business\PYZus{}id}\PY{l+s+s1}{\PYZsq{}}\PY{p}{)}\PY{o}{.}\PY{n}{filter}\PY{p}{(}\PY{k}{lambda} \PY{n}{x}\PY{p}{:} \PY{n+nb}{len}\PY{p}{(}\PY{n}{x}\PY{p}{)} \PY{o}{==}\PY{l+m+mi}{2}\PY{p}{)}\PY{p}{[}\PY{p}{[}\PY{l+s+s1}{\PYZsq{}}\PY{l+s+s1}{business\PYZus{}id}\PY{l+s+s1}{\PYZsq{}}\PY{p}{,}\PY{l+s+s1}{\PYZsq{}}\PY{l+s+s1}{score}\PY{l+s+s1}{\PYZsq{}}\PY{p}{]}\PY{p}{]}\PY{o}{.}\PY{n}{groupby}\PY{p}{(}\PY{l+s+s1}{\PYZsq{}}\PY{l+s+s1}{business\PYZus{}id}\PY{l+s+s1}{\PYZsq{}}\PY{p}{)}\PY{o}{.}\PY{n}{agg}\PY{p}{(}\PY{k}{lambda} \PY{n}{x}\PY{p}{:}\PY{p}{[}\PY{n}{i} \PY{k}{for} \PY{n}{i} \PY{o+ow}{in} \PY{n}{x}\PY{p}{]}\PY{p}{)}\PY{o}{.}\PY{n}{rename}\PY{p}{(}\PY{n}{columns}\PY{o}{=}\PY{p}{\PYZob{}}\PY{l+s+s1}{\PYZsq{}}\PY{l+s+s1}{score}\PY{l+s+s1}{\PYZsq{}}\PY{p}{:} \PY{l+s+s2}{\PYZdq{}}\PY{l+s+s2}{score\PYZus{}pair}\PY{l+s+s2}{\PYZdq{}}\PY{p}{\PYZcb{}}\PY{p}{)}
         
         \PY{c+c1}{\PYZsh{} For some odd reason, we can\PYZsq{}t just pass `list` into `.agg` so we define this function:}
         \PY{c+c1}{\PYZsh{} You may or may not use it}
         
         \PY{k}{def} \PY{n+nf}{group\PYZus{}to\PYZus{}list}\PY{p}{(}\PY{n}{group}\PY{p}{)}\PY{p}{:}
             \PY{k}{return} \PY{n+nb}{list}\PY{p}{(}\PY{n}{group}\PY{p}{)}
\end{Verbatim}


    \begin{Verbatim}[commandchars=\\\{\}]
{\color{incolor}In [{\color{incolor}69}]:} \PY{k}{assert} \PY{n+nb}{isinstance}\PY{p}{(}\PY{n}{scores\PYZus{}pairs\PYZus{}by\PYZus{}business}\PY{p}{,} \PY{n}{pd}\PY{o}{.}\PY{n}{DataFrame}\PY{p}{)}
         \PY{k}{assert} \PY{n}{scores\PYZus{}pairs\PYZus{}by\PYZus{}business}\PY{o}{.}\PY{n}{columns} \PY{o}{==} \PY{p}{[}\PY{l+s+s1}{\PYZsq{}}\PY{l+s+s1}{score\PYZus{}pair}\PY{l+s+s1}{\PYZsq{}}\PY{p}{]}
\end{Verbatim}


    \begin{Verbatim}[commandchars=\\\{\}]
{\color{incolor}In [{\color{incolor}70}]:} \PY{c+c1}{\PYZsh{} Create scatter plot here.}
         \PY{n}{x}\PY{o}{=} \PY{n}{np}\PY{o}{.}\PY{n}{array}\PY{p}{(}\PY{p}{[}\PY{n}{scores\PYZus{}pairs\PYZus{}by\PYZus{}business}\PY{o}{.}\PY{n}{values}\PY{p}{[}\PY{p}{:}\PY{p}{,}\PY{l+m+mi}{0}\PY{p}{]}\PY{p}{[}\PY{n}{i}\PY{p}{]}\PY{p}{[}\PY{l+m+mi}{0}\PY{p}{]} \PY{k}{for} \PY{n}{i} \PY{o+ow}{in} \PY{n+nb}{range}\PY{p}{(}\PY{n+nb}{len}\PY{p}{(}\PY{n}{scores\PYZus{}pairs\PYZus{}by\PYZus{}business}\PY{o}{.}\PY{n}{values}\PY{p}{)}\PY{p}{)}\PY{p}{]}\PY{p}{)}
         \PY{n}{y}\PY{o}{=} \PY{n}{np}\PY{o}{.}\PY{n}{array}\PY{p}{(}\PY{p}{[}\PY{n}{scores\PYZus{}pairs\PYZus{}by\PYZus{}business}\PY{o}{.}\PY{n}{values}\PY{p}{[}\PY{p}{:}\PY{p}{,}\PY{l+m+mi}{0}\PY{p}{]}\PY{p}{[}\PY{n}{i}\PY{p}{]}\PY{p}{[}\PY{l+m+mi}{1}\PY{p}{]} \PY{k}{for} \PY{n}{i} \PY{o+ow}{in} \PY{n+nb}{range}\PY{p}{(}\PY{n+nb}{len}\PY{p}{(}\PY{n}{scores\PYZus{}pairs\PYZus{}by\PYZus{}business}\PY{o}{.}\PY{n}{values}\PY{p}{)}\PY{p}{)}\PY{p}{]}\PY{p}{)}
         \PY{n}{plt}\PY{o}{.}\PY{n}{scatter}\PY{p}{(}\PY{n}{x}\PY{p}{,} \PY{n}{y}\PY{p}{,} \PY{n}{color}\PY{o}{=}\PY{l+s+s1}{\PYZsq{}}\PY{l+s+s1}{\PYZsq{}}\PY{p}{,}\PY{n}{marker}\PY{o}{=}\PY{l+s+s1}{\PYZsq{}}\PY{l+s+s1}{o}\PY{l+s+s1}{\PYZsq{}}\PY{p}{,}\PY{n}{edgecolors}\PY{o}{=}\PY{l+s+s1}{\PYZsq{}}\PY{l+s+s1}{b}\PY{l+s+s1}{\PYZsq{}}\PY{p}{)}
         \PY{n}{plt}\PY{o}{.}\PY{n}{xlabel}\PY{p}{(}\PY{l+s+s2}{\PYZdq{}}\PY{l+s+s2}{first score}\PY{l+s+s2}{\PYZdq{}}\PY{p}{)}
         \PY{n}{plt}\PY{o}{.}\PY{n}{ylabel}\PY{p}{(}\PY{l+s+s2}{\PYZdq{}}\PY{l+s+s2}{second score}\PY{l+s+s2}{\PYZdq{}}\PY{p}{)}
         \PY{n}{plt}\PY{o}{.}\PY{n}{xlim}\PY{p}{(}\PY{l+m+mi}{55}\PY{p}{,} \PY{l+m+mi}{100}\PY{p}{)}
         \PY{n}{plt}\PY{o}{.}\PY{n}{ylim}\PY{p}{(}\PY{l+m+mi}{55}\PY{p}{,} \PY{l+m+mi}{100}\PY{p}{)}
         \PY{n}{x2}\PY{o}{=}\PY{n}{np}\PY{o}{.}\PY{n}{arange}\PY{p}{(}\PY{l+m+mi}{55}\PY{p}{,}\PY{l+m+mi}{100}\PY{p}{,}\PY{l+m+mf}{0.01}\PY{p}{)}
         \PY{n}{y2}\PY{o}{=}\PY{n}{x2}
         \PY{n}{plt}\PY{o}{.}\PY{n}{plot}\PY{p}{(}\PY{n}{x2}\PY{p}{,}\PY{n}{y2}\PY{p}{,}\PY{l+s+s2}{\PYZdq{}}\PY{l+s+s2}{red}\PY{l+s+s2}{\PYZdq{}}\PY{p}{,}\PY{n}{linewidth}\PY{o}{=}\PY{l+m+mi}{2}\PY{p}{)}
         \PY{n}{plt}\PY{o}{.}\PY{n}{show}\PY{p}{(}\PY{p}{)}
\end{Verbatim}


    \begin{center}
    \adjustimage{max size={0.9\linewidth}{0.9\paperheight}}{output_148_0.png}
    \end{center}
    { \hspace*{\fill} \\}
    
    \subsubsection{Question 7d}\label{question-7d}

Another way to compare the scores from the two inspections is to examine
the difference in scores. Subtract the first score from the second in
\texttt{scores\_pairs\_by\_business}. Make a histogram of these
differences in the scores. We might expect these differences to be
positive, indicating an improvement from the first to the second
inspection.

The histogram should look like this:

\emph{Hint: Use \texttt{second\_score} and \texttt{first\_score} created
in the scatter plot code above.}

\emph{Hint: Convert the scores into numpy arrays to make them easier to
deal with.}

\emph{Hint: Try changing the number of bins when you call plt.hist.}

    \begin{Verbatim}[commandchars=\\\{\}]
{\color{incolor}In [{\color{incolor}71}]:} \PY{c+c1}{\PYZsh{} Create histogram here}
         \PY{n}{subtract} \PY{o}{=} \PY{n}{y}\PY{o}{\PYZhy{}}\PY{n}{x}
         \PY{n}{plt}\PY{o}{.}\PY{n}{hist}\PY{p}{(}\PY{n}{subtract}\PY{p}{,}\PY{n}{np}\PY{o}{.}\PY{n}{arange}\PY{p}{(}\PY{o}{\PYZhy{}}\PY{l+m+mi}{28}\PY{p}{,}\PY{l+m+mi}{36}\PY{p}{,}\PY{l+m+mi}{2}\PY{p}{)}\PY{p}{)}
\end{Verbatim}


\begin{Verbatim}[commandchars=\\\{\}]
{\color{outcolor}Out[{\color{outcolor}71}]:} (array([   1.,    0.,    3.,    2.,    4.,    4.,    8.,    7.,   17.,
                   15.,   34.,   66.,   90.,  135.,  218.,  151.,  115.,   63.,
                   44.,   38.,   20.,   11.,    5.,    5.,    5.,    5.,    4.,
                    3.,    0.,    2.,    1.]),
          array([-28, -26, -24, -22, -20, -18, -16, -14, -12, -10,  -8,  -6,  -4,
                  -2,   0,   2,   4,   6,   8,  10,  12,  14,  16,  18,  20,  22,
                  24,  26,  28,  30,  32,  34]),
          <a list of 31 Patch objects>)
\end{Verbatim}
            
    \begin{center}
    \adjustimage{max size={0.9\linewidth}{0.9\paperheight}}{output_150_1.png}
    \end{center}
    { \hspace*{\fill} \\}
    
    \subsubsection{Question 7e}\label{question-7e}

If a restaurant's score improves from the first to the second
inspection, what do you expect to see in the scatter plot that you made
in question 7c? What do you see?

If a restaurant's score improves from the first to the second
inspection, how would this be reflected in the histogram of the
difference in the scores that you made in question 7d? What do you see?

    \begin{Verbatim}[commandchars=\\\{\}]
{\color{incolor}In [{\color{incolor}72}]:} \PY{n}{q7c\PYZus{}answer} \PY{o}{=} \PY{l+s+sa}{r}\PY{l+s+s2}{\PYZdq{}\PYZdq{}\PYZdq{}}
         \PY{l+s+s2}{1. If a restaurant}\PY{l+s+s2}{\PYZsq{}}\PY{l+s+s2}{s score improves from the first to the second inspection, then the corresponding point in the scatter plot are above the reference line. And I see more restaurants}\PY{l+s+s2}{\PYZsq{}}\PY{l+s+s2}{ scores improve from the first to the second inspection.}
         
         \PY{l+s+s2}{2. If a restaurant}\PY{l+s+s2}{\PYZsq{}}\PY{l+s+s2}{s score improves from the first to the second inspection, then the corresponding number will be added to the bar at the right side of 0. And I see more restaurants}\PY{l+s+s2}{\PYZsq{}}\PY{l+s+s2}{ scores improve from the first to the second inspection and restaurants that get worse scores are mostly at the left neighborhood of 0.}
         \PY{l+s+s2}{\PYZdq{}\PYZdq{}\PYZdq{}}
         \PY{n+nb}{print}\PY{p}{(}\PY{n}{q7c\PYZus{}answer}\PY{p}{)}
\end{Verbatim}


    \begin{Verbatim}[commandchars=\\\{\}]

1. If a restaurant's score improves from the first to the second inspection, then the corresponding point in the scatter plot are above the reference line. And I see more restaurants' scores improve from the first to the second inspection.

2. If a restaurant's score improves from the first to the second inspection, then the corresponding number will be added to the bar at the right side of 0. And I see more restaurants' scores improve from the first to the second inspection and restaurants that get worse scores are mostly at the left neighborhood of 0.


    \end{Verbatim}

    \subsection{Summary of the Inspections
Data}\label{summary-of-the-inspections-data}

What we have learned about the inspections data? What might be some next
steps in our investigation?

\begin{itemize}
\tightlist
\item
  We found that the records are at the inspection level and that we have
  inspections for multiple years.\\
\item
  We also found that many restaurants have more than one inspection a
  year.
\item
  By joining the business and inspection data, we identified the name of
  the restaurant with the worst rating and optionally the names of the
  restaurants with the best rating.
\item
  We identified the restaurants that have had the largest swing in
  rating over time.
\item
  We also examined the relationship between the scores when a restaurant
  has multiple inspections in a year. Our findings were a bit
  counterintuitive and may warrant further investigation.
\end{itemize}

    \subsection{Congrats !}\label{congrats}

Congrats! You are finished with HW1.

    \subsection{Submission}\label{submission}

You're done!

Before submitting this assignment, ensure to:

\begin{enumerate}
\def\labelenumi{\arabic{enumi}.}
\tightlist
\item
  Restart the Kernel (in the menubar, select
  Kernel-\textgreater{}Restart \& Run All)
\item
  Validate the notebook by clicking the "Validate" button
\end{enumerate}

Finally, make sure to \textbf{submit} the assignment via the Assignments
tab in Datahub


    % Add a bibliography block to the postdoc
    
    
    
    \end{document}
